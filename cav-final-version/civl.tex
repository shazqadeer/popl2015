\section{Verification}
\label{sec:verification}

In this section, we present our verification method on a core concurrent programming language called \civl (Figure~\ref{fig:syntax}).
Due to lack of space, we can only provide an overview of the design of the \civl verifier.
The full formalization of the language and detailed rules for all verification judgments is available in a technical report~\cite{gc-techreport}.

\begin{wrapfigure}[10]{l}{0.5\textwidth}
\setlength{\tabcolsep}{3pt}
\scriptsize{
\begin{tabular}{rclcl}
$\stmt \in \Stmt$ &::= & $\skipstmt \mid \yield{e} \mid \call{A} \mid$ \\
                  & & $\call{P} \mid \async{P} \mid$ \\
                  & & $\ablock{e}{\stmt} \mid \stmt;\;\stmt \mid$\\
                  & & $\ite{\locExpr}{\stmt}{\stmt} \mid$ \\
                  & & $\while{e}{\locExpr}{\stmt}$ \\
$F \in \mathit{Frame} $ &::= & $(P, \varsL, \stmt)$ \\
$T \in \Thread$ &::= & $(\varsTL, \Frames)$ \\
$\Prog \in \Program$ &::= & $(\procs, \actions, \varsG, \TS)$
\end{tabular}
}
\caption{Syntax}
\label{fig:syntax}
\end{wrapfigure}
A \civl program $\Prog$ contains procedures $\procs$, atomic actions $\actions$, 
global state $G$, and threads $\TS$.
Each thread $T$ in $\TS$ contains thread-local state $\varsTL$ and stack frames $\Frames$.
Each stack frame $\Frame$ in $\Frames$ contains a procedure name $P$, procedure-local state $\varsL$, and a statement $\stmt$
representing the code in $P$ that remains to be executed.
Thus, $\Prog$ contains all information to represent not only the static program 
written by the programmer but also the entire state of the program as it executes.
The statements in \civl contain the usual constructs such as sequencing, conditional control flow, and looping.
In addition, it contains invocation of procedures ($\call{P}$), execution of atomic actions ($\call{A})$, and thread creation ($\async{P}$).
Each atomic action has a single-state {\em gate predicate\/} and a two-state {\em transition relation\/}.
If a thread executes an atomic action in a state (disjoint union of global, thread-local, and procedure-local state) 
where its gate predicate does not hold, the program fails;
otherwise, the state is modified according to its transition relation.
The execution of $\Prog$ is modeled as the usual {\em preemptive\/} semantics in which a
nondeterministically chosen thread may execute any number of steps.
$\Prog$ is {\em unsafe\/} if some execution fails the gate of an atomic action;
otherwise, $\Prog$ is {\em safe\/}.

\newcommand{\hi}{\mathit{hi}}
\newcommand{\lo}{\mathit{lo}}

Suppose a program $\Prog^\hi$ has been proved to be safe.
However, it is implemented using atomic actions that are too coarse to be directly implementable.  
To carry over the safety of $\Prog^\hi$ to a realizable implementation $\Prog^\lo$, 
these coarse atomic actions must be refined down to lower-level actions.
During refinement, a high-level atomic action $A$ is implemented by a procedure $P$, which is itself implemented using lower-level atomic actions.
In \civl, the programmer can simultaneously refine many atomic actions by specifying a partial function $\Refines$ from procedures to atomic actions;
$\Prog^\hi$ is obtained from $\Prog^\lo$ by replacing each occurrence of $\call{P}$ for $P \in \dom(\Refines)$ with $\call{\Refines(P)}$.
The main contribution of this paper is a verification method that allows us to validate such a refinement from 
$\Prog^\hi$ to $\Prog^\lo$ (or abstraction from $\Prog^\lo$ to $\Prog^\hi$) so that 
safety of $\Prog^\hi$ implies the safety of $\Prog^\lo$ as well.

While abstracting $\Prog^\lo$ to $\Prog^\hi$, it is often inconvenient to 
reason about $\Prog^\lo$ using its preemptive semantics, which allows potential interference 
at {\em every\/} control location in a thread from concurrently-executing threads.
To make reasoning more convenient, \civl provides the statement $\yield{e}$, an annotation 
used to specify a cooperative semantics for the program.
In this semantics, a thread executes continuously until it reaches a yield statement, 
at which point a different thread may be scheduled.
To ensure that any reasoning performed on cooperative semantics is also sound for preemptive semantics,
\civl exploits commutativity reasoning.
It allows the programmer to specify 
the commutativity type of atomic actions in the program---$B$ for both mover, $R$ for right mover, $L$ for left mover, 
and $N$ for non mover~\cite{FlanaganFLQ08}. 
The \civl verifier checks the correctness of these commutativity types by verifying each atomic action pairwise against 
every atomic action in the program.
While it is sound to put a yield statement before and after every atomic action,
the programmer may omit certain yield statements, e.g., a yield after a right mover or a yield before a left mover.
In general, the {\em Yield Sufficiency Automaton\/} from Figure~\ref{fig:ysa} encodes 
all sequences of atomic actions and yield statements for which reasoning about cooperative semantics is sound.
Given the commutativity types of atomic actions and the program code annotated with yield statements,
the \civl verifier checks modularly for each procedure that its implementation is connected to $\YSA$
via a simulation relation~\cite{HenzingerHK95}.

In addition to introducing a control location where interference is allowed to occur, 
a yield statement $\yield{e}$ also provides an invariant $e$ 
to constrain the environment interference.
The invariant $e$ is similar to the location invariant in the method of Owicki and Gries~\cite{OwickiG76}.
It is expected to hold when the executing thread reaches the yield statement (sequential correctness) 
and also be preserved by concurrently-executing threads (non-interference).
Each procedure is equipped with a precondition, a postcondition,
and a set of (potentially) modified thread-local variables.
\civl uses these procedure annotations to verify the sequential correctness of location invariants for each
procedure separately.
To verify non-interference, it would suffice to check that each location invariant is preserved by each atomic action in the program.
\civl increases the precision of this check by allowing each location invariant to be preserved across 
an atomic block, introduced as the statement $\ablock{e}{\stmt}$.
The invariant $e$ annotating the atomic block is expected to hold when this statement begins execution and is verified as part of sequential correctness.
The \civl type checker checks that the statement $\stmt$ inside this atomic block does not have any yield statement or other atomic blocks inside it.
Thus, non-interference of a location invariant $e'$ against $\ablock{e}{\stmt}$ is achieved by proving the Floyd-Hoare triple $\{e \wedge e'\}\stmt\{e'\}$.

Having verified sequential correctness and non-interference for location invariants, it remains to verify refinement,
i.e., if $\Refines(P) = A$, then the atomic action $A$ is correctly refined by the procedure $P$.
This requirement means that any path from entry to exit of $P$ must contain exactly one atomic block that implements the action $A$;
all other atomic blocks on the path must leave global and thread-local variables unchanged.
To perform this check, the \civl verifier introduces the following fresh local variables in $P$:
(1)~a $\Boolean$ variable $b$ initialized to $\false$ to track whether an atomic block along the current execution has modified a global or thread-local variable,
(2)~variables to capture snapshot of global and thread-local variables at the beginning of each atomic block.
By updating these auxiliary variables appropriately, the refinement check is reduced to a collection of assertions introduced into the body of $P$
at the end of atomic blocks and at the exit of $P$.

Often, commutativity and non-interference checks require knowledge about distinctness of local program variables in different threads.
For example, in Figure~\ref{fig:reft}, to prove that {\tt AcquireLock} commutes to the right of {\tt ReleaseLock}, the verifier must know
that the input parameter {\tt tid} to these atomic actions is different if they are being executed by different threads.
A similar situation arises in Figure~\ref{fig:ex5}, when attempting to prove that the location invariant {\tt t == a[tid]} is preserved by the atomic 
action {\tt a[tid] := t + 1}. 
Information about distinctness of program variables in different threads is difficult to provide as a location invariant 
whose scope is local to the context of the unique executing thread.
As an alternative, we exploit reasoning based on a linear type system~\cite{Wadler90lineartypes}.
The programmer declares certain variables as linear at input and output interfaces of procedures and actions.
Using this interface information, the \civl type system computes a set of {\em available\/} linear variables at each control location 
in a procedure.
The availability of a variable may change at an assignment or a procedure call, e.g., 
if {\tt y} is available just before {\tt x := y}, then {\tt y} is not available and {\tt x} is available just afterwards.
The \civl type checker guarantees that the values contained in available linear variables, across all threads at their respective control locations, are 
distinct from each other.
This fact is introduced as a logical assumption by the verifier when performing commutativity and non-interference checks.

The interaction between the linear type system and logical reasoning in \civl is more general than the description above.
In \civl, the programmer may specify an arbitrary function $\Perm$ from a value to a set of values;
the set $\Perm(v)$ is the set of {\em permissions\/} associated with $v$.
The example described in the previous paragraph corresponds to the special case when $\Perm(v) = \{v\}$.
The \civl type checker enforces a generalization of the distinctness invariant that 
the permission sets corresponding to the values in available variables across all threads are mutually disjoint.

\subsection{Safety guarantee}

We can combine the verification techniques described above to verify the safety of a program $\Prog^\lo$.
Specifically, we can guarantee that $\Prog^\lo$ is safe (i.e., all atomic actions will satisfy their gates when run)
if the following conditions hold:

\begin{enumerate}
\item
$\Prog^\hi$ is safe when executed with preemptive semantics.
\item
$\Prog^\lo$ is a valid refinement of $\Prog^\hi$, according to the rules for refinement in \civl.
Specifically, for any atomic action $A$ in $\Prog^\hi$ implemented by a procedure $P$ in $\Prog^\lo$,
any path from entry to exit of $P$ must contain exactly one atomic block that implements the action $A$;
all other atomic blocks on the path must leave the global and thread-local state unchanged.
Furthermore, all calls to $A$ in $\Prog^\hi$ are replaced by calls to $P$ in $\Prog^\lo$.
\item
The invariants of $\Prog^\lo$ satisfy sequential correctness and non-interference with respect to cooperative semantics.
\item
$\Prog^\lo$ is {\em well-typed} with respect to linearity.
Specifically, $\Prog^\lo$ does not try to duplicate any linear variables,
and linear variables passed to procedures calls and atomic actions are available
as expected by the type checker.
\item
The atomic actions in $\Prog^\lo$ satisfy the pairwise commutativity checks.
\item
The yield statements in $\Prog^\lo$ are sufficient, according to the yield sufficiency automaton in Figure~\ref{fig:ysa}.
\item
Any infinite execution of $\Prog^\lo$ must visit a yield statement infinitely often.
\end{enumerate}

By themselves, conditions~1-4 guarantee that $\Prog^\lo$ will be safe when executed with cooperative semantics.
Conditions~5-7 then additionally ensure that $\Prog^\lo$ will be safe when executed with preemptive semantics.
The technical report~\cite{gc-techreport},
which includes formal definitions of all the conditions for an extension of the language in Figure~\ref{fig:syntax},
formalizes this safety guarantee into a soundness theorem by establishing a simulation relation between $\Prog^\lo$ and $\Prog^\hi$.
Since the theorem connects the safety of one program's preemptive semantics to another program's preemptive semantics,
multiple applications of the theorem can be chained together to establish the safety of a low-level program:
the lowest level $\Prog^0$ is safe because $\Prog^1$ is safe, $\Prog^1$ is safe because $\Prog^2$ is safe, and so on.

