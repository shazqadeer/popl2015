\section{Experience}
\label{sec:experience}

The \civl verifier has been under development for around two years.  
Over that period, we have developed a collection of 32 benchmarks, 
ranging in size from 17 to 539 LOC, to illustrate various features of
\civl and for regression testing as we evolved the verifier.
In addition to microbenchmarks, this collection also includes
standard benchmarks from the literature such as a multiset implementation~\cite{ElmasTQ05}, 
the ticket algorithm~\cite{FarzanKP14}, 
Treiber stack~\cite{Herlihy2008}, work-stealing queue~\cite{Blumofe1999},
device cache~\cite{ElmasQT09}, and lock-protected increment~\cite{FlanaganQ03}. 
The \civl verifier is fast; the entire benchmark set verifies in 20 seconds on a standard 4-core Windows PC (2.8GHz, 8GB)
with no benchmark requiring more than a few seconds.

\subsection{Garbage collector}
We have used \civl to design and verify a realistic concurrent mark-sweep garbage collection (GC) algorithm (available at~\cite{GC}).  
In particular, although our algorithm is based on an earlier algorithm by Dijkstra et al~\cite{dijk78}, 
it extends the earlier algorithm with various modern optimizations and embellishments to improve generality and performance.  
These extensions include lower write barrier overhead, phase-based synchronization and handshaking, 
and coordination between the GC and mutator threads during root scanning; our use of linearity aids the proof of root scanning, 
while our rely-guarantee encoding aids management of colors inside the write barrier
(which is similar to the barrier in Section~\ref{sec:overview}).
Furthermore, our encoding of the algorithm in \civl spans a wide range of abstraction, 
from low-level memory operations all the way up to high-level specifications; 
we used six layers of refinement to help hide low-level details from the high-level portions of the verification.

We believe that \civl's combination of features makes practical, for the first time, verification across such a wide range of abstraction:

\begin{itemize}
\item The GC's lowest layers relied primarily on reduction to
prove that synchronization primitives and operations on
concurrent data structures and synchronization appear atomic
to higher layers.

\item The GC's higher layers relied primarily on invariant-based non-interference reasoning.
This reasoning was simplified because reduction already made lower-layer operations atomic,
reducing the amount of interference between higher-layer operations.
In addition, the use of location invariants made certain layers of the proof more
manageable compared to an earlier effort verifying the same GC where we used
rely-guarantee reasoning and auxiliary variables to reason about
non-interference.  

\item Linear variables were used throughout the proof to model the distinct
thread identifiers for the garbage collector thread and mutator
threads, but were most instrumental in expressing mutual exclusion
during initialization and during root scanning.
In initialization and root scanning,
the mutator threads temporarily donate a fraction of their linear permissions to the GC thread.
The distinctness invariant from Section~\ref{sec:verification} guarantees
that the mutator threads and GC threads cannot simultaneously possess the same linear permissions;
we leverage this guarantee to prove non-interference of mutator and GC actions during initialization and root scanning.

%Variable hiding was used throughout the proof.
\end{itemize}

\civl's support for refinement also enabled concise specifications of the GC's correctness:
a correct GC must implement Allocate, ReadField, and WriteField actions that appear to act atomically,
even though the implementation of these operations actually executes concurrently with the GC thread and with other program threads.
The specification states that Allocate atomically adds new objects to the heap,
while ReadField and WriteField read and write heap object fields.
Although the GC's Mark and Sweep code constitutes most of the GC code,
they are hidden in the high-level specification;
they have detailed correctness specifications in the middle layers of the proof,
but the most important point at the high level is that their work not interfere with Allocate, ReadField, and WriteField.
In particular, Mark must coordinate with WriteField's write barrier,
and Sweep must not remove objects reachable by ReadField and WriteField.

Overall, our GC implementation consists of about 2100 lines of Boogie code.
The verification takes 60 seconds on the same PC used for microbenchmarks.
The bulk of this time, 54 seconds, is taken by the verification of
sequential correctness and non-interference.
The checks for linear variables, yield sufficiency, and commutativity take the rest of the time and are insignificant in comparison.

%\subsection{Discussion}
%The GC verification demonstrates that stepwise refinement using \civl is a
%powerful verification approach. 
%Another important lesson learned is that a collection
%of techniques is needed to support stepwise refinement, and a different blend of
%techniques best serves the verification goals at different layers of refinement. 

%Our experience with the GC verification indicates that 
%\civl's capability to use multiple verification techniques, soundly and cooperatively, is very valuable. 

