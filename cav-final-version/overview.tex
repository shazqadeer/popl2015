\section{Overview}
\label{sec:overview}

We present an overview of our approach to refinement on the program in Figure~\ref{fig:reft}, a simplified version of the write barrier in a concurrent garbage collector (GC).
In a concurrent GC, a color (\exC{UNALLOC}, \exC{WHITE}, \exC{GRAY}, or \exC{BLACK}) is associated with each object on the heap.  
Before writing to an address \exC{addr}, a mutator executes a write barrier. 
It checks \exC{addr} has color \exC{WHITE} and sets it to \exC{GRAY}, indicating that the object at \exC{addr} should not be garbage collected.
\begin{wrapfigure}[26]{l}{0.475\textwidth}
\vspace*{-1cm}
{\scriptsize
\begin{verbatim}
var Color: int; // UNALLOC=0, WHITE=1,
                // GRAY=2, BLACK=3

procedure WB(linear tid:Tid)
atomic [if (Color == WHITE) Color := GRAY];
requires Color >= WHITE;
ensures Color >= GRAY;
{
  var cNoLock:int;
  cNoLock := GetColorNoLock(tid);
  yield Color >= cNoLock;
  if (cNoLock <= WHITE) 
    call WBSlow(tid);
}

procedure WBSlow(linear tid:Tid)
atomic [if (Color <= WHITE) Color := GRAY];
{
  var cLock:int;
  call AcquireLock(tid);
  cLock := GetColorLocked(tid);
  if (cLock <= WHITE) 
    call SetColorLocked(tid, GRAY);
  call ReleaseLock(tid);
}

procedure GetColorNoLock(linear tid:Tid) 
  returns (cl:int) atomic [...];
procedure AcquireLock(linear tid:Tid) 
  right [...];
procedure ReleaseLock(linear tid:Tid) 
  left [...];
procedure GetColorLocked(linear tid:Tid) 
  returns (cl:int) both [...];
procedure SetColorLocked(linear tid:Tid, 
  cl: int) atomic [...];
\end{verbatim}
}
\vspace*{-0.3cm}
\caption{Write barrier}
\label{fig:reft}
\end{wrapfigure}
\exC{WB} implements the write barrier.
The write barrier is only invoked on allocated objects, thus, colors cannot be \exC{UNALLOC} when \exC{WB} is called.
To simplify exposition, we consider a single object whose color is stored in the shared variable \exC{Color}. 
\exC{WB} first reads \exC{Color} without holding a lock, to avoid when possible, the cost of acquiring and releasing a lock for each address encountered by a mutator. 
If \exC{Color <= WHITE}, \exC{WB} calls the more expensive procedure \exC{WBSlow} to re-examine and possibly update \exC{Color} while holding the lock. 
The annotation \exC{yield} \exC{Color >= cNoLock} is a local invariant expected to be preserved by the environment of \exC{WB}. 
\civl simplifies reasoning about \exC{WBSlow} by allowing us to express its specification as the following atomic action:\\
\begin{footnotesize}
\begin{tt}
[if (Color <= WHITE)  Color := GRAY]
\end{tt}
\end{footnotesize}\\
This specification indicates that regardless of how the environment interferes with its execution, to its caller it appears as if \exC{WBSlow} atomically executes the code above. 

{\bf Per-procedure simulation, non-interference via invariants.}
The verification of \exC{WB} illustrates a combination of techniques. 
We first explain how \exC{WB}'s post-condition is verified. 
To see that this task is not trivial, consider a scenario in which \exC{WB}, not holding a lock, reads {\tt Color} and sets \exC{cNoLock} to \exC{GRAY} and then yields. 
Another thread sets {\tt Color} to {\tt WHITE}. 
\exC{WB} resumes, but because the local variable \exC{cNoLock} is \exC{GRAY}, does nothing and exits with {\tt Color} being {\tt WHITE}, violating \exC{WB}'s postcondition. 
But, in the GC this scenario is not possible. 
The yield predicate (location invariant) {\tt Color >= cNoLock} expresses the fact that other threads can only modify {\tt Color} to a higher (darker) value. 
\civl verifies the correctness of this location invariant and rules out this undesirable scenario. 
Using this location invariant, \exC{WB}'s pre-condition, and \exC{WBSlow}'s atomic specification, \civl is able to verify \exC{WB}'s post-condition. 

To verify that the implementation of \exC{WB} refines its atomic specification, \civl critically relies on the fact that {\tt Color} is never {\tt UNALLOC} during \exC{WB}'s execution. 
This is important because otherwise \exC{WBSlow} would set {\tt Color} to {\tt GRAY} whereas \exC{WB} would leave it unmodified, leading to a refinement violation. 
\exC{WB}'s precondition {\tt Color >=  WHITE} and the location invariant {\tt Color >= cNoLock} imply that {\tt Color} is never {\tt UNALLOC} during the execution of \exC{WB}. 
In the environment thus constrained, \civl checks atomicity refinement for {\tt WB} by verifying the existence of a particular simulation-relation. 
Each control path through \exC{WB} is analyzed as a sequence of code fragments, from one {\tt yield} statement to the next\footnote{Procedure entry and exit points are treated as {\tt yield} statements annotated by the procedure pre- and post-conditions.}.
For each control path through a procedure, exactly one code fragment must be simulated by the atomic action specification while others do not modify global state.
This refinement proof for \exC{WB} makes use of (1) correct modeling of environment interference by the pre- and post-conditions, and the yield predicate, and (2) the atomic action specification for the called procedure \exC{WBSlow}. 
The \civl verifier automatically computes a logical verification condition capturing the proof obligations from the body and specification of \exC{WB}.

Just as the verification of \exC{WB} builds on the specification of \exC{WBSlow}, the verification of \exC{WBSlow} builds on other refinement proofs (not shown) of the procedures called in \exC{WBSlow}; these procedures are shown at the bottom of the figure. 
This example shows only one procedure at this layer. In programs with many procedures with atomic specifications at each layer, \civl combines the per-procedure refinement proofs soundly into a whole-program refinement proof. 

{\bf Preemptive vs cooperative semantics.}
The verification of \exC{WBSlow} highlights another important feature in \civl. 
Refinement checking is performed on cooperative semantics in which a 
{\tt yield}-to-{\tt yield} execution fragment of code is executed atomically.
However, in a real execution, control can switch between threads at any point in the code. 
A naive modeling of a real execution would put a yield statement before every instruction in the code.
The absence of a yield statement before every instruction is justified by reasoning about mover types~\cite{FlanaganFLQ08}. 
The procedures called in \exC{WBSlow} have the mover types claimed in their
declarations and verified by \civl. 
For example, the mover type of \exC{Acquirelock} is {\tt right} which indicates 
that it commutes later in time against concurrently executing
environment actions. 
% \exC{ReleaseLock} has mover type {\tt left} and it commutes earlier in time;
% \exC{SetColorLocked} has mover type {\tt atomic} and it does not commute;
% \exC{GetColorLocked} has mover type {\tt both} and it commutes both earlier and later in time.
These mover types are checked by constructing verification conditions from each pair of atomic actions.

Given verified mover types for actions, \civl verifies the correctness of the placement of {\tt yield} statements using a novel approach.
\begin{wrapfigure}[7]{l}{0.5\textwidth}
\vspace*{-1cm}
\begin{center}
\includegraphics[scale=0.25]{YieldTypeCheckingAutomaton.pdf}
\end{center} 
\vspace*{-0.3cm}
\caption{Yield sufficiency automaton}
\label{fig:ysa}
\end{wrapfigure}
A {\em yield sufficiency automaton\/} (Figure~\ref{fig:ysa})
encodes all sequences of atomic actions (of {\bf R}ight, 
{\bf L}eft,
{\bf B}oth and
{\bf N}on-mover types)  and yields for which safety of cooperative semantics is sufficient 
for safety of preemptive semantics. 
Each ``transaction'' starts with a sequence of right movers (or both movers) and ends with a sequence of left movers (or both movers).
In the middle, it can have at most one non mover. Transactions must be
separated by {\tt yield} statements.
\civl then interprets the control-flow graph of each procedure as an automaton with mover types as edge labels. 
This abstraction for \exC{WBSlow} is shown in Figure~\ref{fig:midwb}.
\civl verifies that this automaton is simulated by the yield sufficiency automaton using an existing algorithm for computing simulation relations~\cite{HenzingerHK95}.

The use of commutativity reasoning is optional in \civl, but beneficial in our experience. 
Commutativity reasoning may be avoided by annotating atomic action specifications with the mover type {\tt atomic}
\begin{wrapfigure}[7]{l}{0.5\textwidth}
\vspace*{-1cm}
\begin{center}
\includegraphics[scale=0.25]{WBSlow.pdf}
\end{center}
\vspace*{-0.8cm}
\caption{Abstraction of \exC{WBSlow}}
\label{fig:midwb}
\end{wrapfigure}
and inserting a yield statement before every invocation of an atomic action.
In our experience with \civl, using more yield statements,
each with an accompanying location invariant, can make proofs difficult in two ways.
First, the annotation burden goes up because sophisticated ghost variables may need to be introduced in the 
program semantics.\footnote{Location invariants that cannot refer to the state of other threads are known to be incomplete, 
both in theory and in practice.}
Second, the computational cost of the pairwise mover reasoning is replaced by the cost of pairwise non-interference checks between yield predicates 
and concurrently executing atomic actions. 

{\bf Linear variables.}
In Figure~\ref{fig:reft}, thread identifier (\exC{tid}) variables are declared \exC{linear} 
to indicate that two threads cannot possess the same thread identifier simultaneously.
We now explain this feature of \civl in more details using 
\begin{wrapfigure}[17]{l}{0.5\textwidth}
\vspace*{-1cm}
{\scriptsize
\begin{verbatim}
type Tid;
procedure Allocate() 
  returns (linear tid:Tid);

var a:[Tid]int;

procedure main()
{
  while (true) {
    var linear tid:Tid := Allocate();
    async call P(tid);
    yield true;
  }
}
procedure P(linear tid: Tid)
  ensures a[tid] == old(a)[tid] + 1;
{
  var t:int := a[tid];
  yield t == a[tid];
  a[tid] := t + 1;
}
\end{verbatim}
}
\vspace*{-0.3cm}
 \caption{Encoding thread identifiers}
\label{fig:ex5}
\end{wrapfigure}
the program in Figure~\ref{fig:ex5}.
This example contains a shared global array {\tt a} indexed by an uninterpreted type {\tt Tid} 
representing the set of thread identifiers.
A collection of threads are executing procedure {\tt P} concurrently.
The identifier of the thread executing {\tt P} is passed in as the parameter {\tt tid}.
A thread with identifier {\tt tid} owns {\tt a[tid]} and can increment it without danger of interference.
The yield predicate {\tt t == a[tid]} in {\tt P} indicates this expectation, yet it is not possible to prove it 
unless the reasoning engine knows that the value of {\tt tid} in one thread is distinct 
its value in a different thread.

Instead of building a notion of thread identifiers into \civl, we provide a more 
primitive and general notion of linear variables.
The \civl type system ensures that values contained in linear variables cannot be duplicated~\cite{Wadler90lineartypes}.
Consequently, the parameter {\tt tid} of distinct concurrent calls to {\tt P} are known to be distinct;
the \civl verifier exploits this invariant while checking for non-interference and commutativity.
Linearity is general enough to support much more than just fixed thread identifiers:
\civl also uses it to express separation of memory (as is done
commonly in separation logic proofs~\cite{Reynolds02}; see \cite{LahiriQW11})
and to express permissions~\cite{boyland:03fractions} that may be transferred but not duplicated between threads.
Our verified GC, for example, expresses mutual exclusion during initialization and root scanning by temporarily transferring permissions from mutator threads to the GC thread.

{\bf Variable hiding.}
The atomic action specification of \exC{WBSlow} makes no reference to the lock variable, although its implementation involves a lock. 
When verifying refinement for \exC{WBSlow}, the lock variable has been hidden. 
\civl allows the programmer to both introduce and hide variables in
each refinement step, thereby providing the capability to perform data refinement.
The ability to introduce and hide variables and write yield predicates specific to each refinement step 
facilitates proofs spanning a large range of abstraction.

