\section{Modules}
\label{sec:modules}

The technical report\cite{gc-techreport} describes a simple module system built on \civl that allows separate verification of modules,
allowing programmers to check a large program by breaking it into smaller pieces and checking the pieces independently.
A key challenge for modular verification in \civl is the checking of non-interference and commutativity.
Naively, these are whole-program judgments,
quadratically checking all pairs of actions or all pairs of yields and atomic blocks from an entire program.
To check these judgments on a per-module basis rather for a whole program,
we observe that commutativity and non-interference are trivially satisfied for operations that act on disjoint sets of global variables.
If an atomic block modifies only variables $g_1$ and $g_2$, it will not interfere with a location invariant that refers only to variables $g_3$ and $g_4$.
More generally, let each module $M$ own a set of global variables, such that each global variable is owned by exactly one module,
and decree that only $M$'s procedures and actions can access $M$'s global variables.
Statements in $M$'s procedures can only read and write $M$'s own global variables,
and $M$'s actions and location invariants can only refer to $M$'s own global variables.
(On the other hand, procedure assertions that are not checked for non-interference,
such as the $e$ in $\ablock{e}{\stmt}$,
may mention global variables from other modules,
since these assertions can neither interfere with other modules' location invariants nor be interfered with by other modules' statements.)

Note that ownership can change across refinement layers.
For example, a library module implementing locks may define a variable to represent the abstract state of a lock;
after the lock module is verified at a low layer,
another module can take ownership of the lock variable in a higher layer
(see \cite{gc-techreport} for a detailed example of ownership transfer across three layers, from a lock module to a datatype module to a client module).
