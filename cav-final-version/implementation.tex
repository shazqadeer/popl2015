\section{Implementation}
\label{sec:implementation}

We have implemented the method described in this section as a conservative extension 
of the Boogie~\cite{BarnettCDJL05} language and verifier.
Our implementation provides new language primitives for linear variables, asynchronous and parallel procedure calls, 
yields, atomic actions as procedure specifications, expressing refinement layers, and hiding of global variables and procedures.
At its core, Boogie is an unstructured language comprising code blocks and goto statements.
Our implementation handles the complexity of unstructured control flow.
To simplify the exposition, our formalization uses Floyd-Hoare triples to present sequential correctness and 
annotated atomic code blocks to present refinement and non-interference checks.
However, our implementation is considerably more automated.  
All the annotations, except those at yields, loops, and procedure boundaries, are automatically generated 
using the technique of verification conditions~\cite{BL05}.
Annotated atomic code blocks are also inferred automatically.
Non-interference checks are collected as inlined procedures
invoked at appropriate places within the code of a procedure for increased precision.

We automated the simulation relation check used for yield sufficiency in Section~\ref{sec:verification} 
by adapting an algorithm by Henzinger et al.\cite{HenzingerHK95} for computing the similarity relation of 
labeled graphs.
The complexity of the algorithm is $O(n*m)$, where $n$ and $m$ are the number of control-flow graph nodes and edges.
In practice, this part of the verification is fast.

A large proof usually comprises multiple layers of refinement chained together.
Our implementation allows the specification of multiple views of a program in a single file by using the mechanism of {\em layers}.
The programmer may attach a positive layer number to each annotation and procedure; 
version $i$ of the program is constructed from annotations labeled $i$ and procedures labeled at least $i$.
We have implemented a type checker to make sure that layer numbers are used appropriately, e.g., 
it is illegal for a procedure with layer $i$ to call a procedure with layer $j$ greater than $i$.
