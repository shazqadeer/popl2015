\section{Operational semantics}
\label{sec:operational-semantics}

\begin{figure}
\scriptsize{
\medskip
%%%%%%%%%%%%%%%%%%%%
$
\inferrule
{
\ProgCtxt[\varsG][\varsTL][\varsL][s] = (\_, \actions, \_, \_, \_) \\
\actions \vdash (\MakeStore{\varsG}{\varsTL}{\varsL}, s) \trans (\MakeStore{\varsG'}{\varsTL'}{\varsL'}, s')
}
{\ProgCtxt[\varsG][\varsTL][\varsL][s] \trans \ProgCtxt[\varsG'][\varsTL'][\varsL'][s']}
\;(\textsc{Step})
$
\medskip
%%%%%%%%%%%%%%%%%%%%
$
\inferrule
{
\ProgCtxt[\varsG][\varsTL][\varsL][s] = (\_, \actions, \_, \_, \_) \\
\actions \vdash (\MakeStore{\varsG}{\varsTL}{\varsL}, s) \fails
}
{\ProgCtxt[\varsG][\varsTL][\varsL][s]) \fails}
\;(\textsc{Fail})
$
\medskip
%%%%%%%%%%%%%%%%%%%%
$
\inferrule
{
\procs(P) = (\phi, \mods, \psi,\stmt) \\
\ProcLins(P) = (\lins, \lins') \\
T' = (\varsTL, (\varsL, \yield{\phi,\lins};\stmt)) \\
\TS = \ThreadCtxt[\varsTL][\varsL][\async{P}] \\
\TS' = \ThreadCtxt[\varsTL][\varsL][\skipstmt]
}
{
(\procs, \actions, \ProcLins, \varsG, \TS)
\trans
(\procs, \actions, \ProcLins, \varsG, \TS' \cdot T')
}
\;(\textsc{Async})
$
\medskip
%%%%%%%%%%%%%%%%%%%%
$
\inferrule
{
T = (\varsTL, (\varsL,\skipstmt)) 
}
{(\procs, \actions, \ProcLins, \varsG, \YieldingThreads \cdot T \cdot \YieldingThreads') \trans (\procs, \actions, \ProcLins, \varsG, \YieldingThreads \cdot \YieldingThreads')}
\;(\textsc{End})
$
\medskip
%%%%%%%%%%%%%%%%%%%%
$
\inferrule
{
\procs(P) = (\phi, \mods, \psi,\stmt) \\
\ProcLins(P) = (\lins, \lins') \\
\StmtStack = \yield{\phi,\lins};\stmt;\yield{\psi,\lins'}
}
{\ProgCtxt[\varsG][\varsTL][\varsL][\call{P}] \trans \ProgCtxt[\varsG][\varsTL][\varsL][\Frame{L}{\StmtStack}]}
\;{(\textsc{Call})}
$
\medskip
%%%%%%%%%%%%%%%%%%%%
$
\inferrule
{
\\
}
{\ProgCtxt[\varsG][\varsTL][\varsL][\Frame{\varsL'}{\skipstmt}] \trans \ProgCtxt[\varsG][\varsTL][\varsL][\skipstmt]}
\;{(\textsc{Return})}
$
%%%%%%%%%%%%%%%%%%%%
}
\caption{Operational semantics for program}
\label{fig:operational-semantics1}
\end{figure}


\begin{figure}
\scriptsize{
\medskip
%%%%%%%%%%%%%%%%%%%%
$
\inferrule
{
\vars = \MakeStore{\varsG}{\varsTL}{\varsL} \\
\varsL \in \locExpr
}
{\actions \vdash (\vars, \assert{\locExpr}) \trans (\vars, \skipstmt)}
\;{(\textsc{Assert-True})}
$
\medskip
%%%%%%%%%%%%%%%%%%%%
$
\inferrule
{
\vars = \MakeStore{\varsG}{\varsTL}{\varsL} \\
\varsL \not \in \locExpr
}
{\actions \vdash (\vars, \assert{\locExpr}) \fails}
\;{(\textsc{Assert-False})}
$
\medskip\\
%%%%%%%%%%%%%%%%%%%%
$
\inferrule
{
\actions(A) = (\rho, \alpha, m) \\
(\vars, \vars') \in \alpha \\
}
{
\actions \vdash (\vars, \call{A}) \trans (\vars',\skipstmt)
}
\;{(\textsc{Atomic})}
$
\medskip
%%%%%%%%%%%%%%%%%%%%
$
\inferrule
{
\\
}
{\actions \vdash (\vars, \yield{e,\lins}) \trans (\vars, \skipstmt)}
\;{(\textsc{Yield})}
$
\medskip
%%%%%%%%%%%%%%%%%%%%
$
\inferrule
{
}
{\actions \vdash (\vars, \ablock{e,\lins}{\stmt}) \trans (\vars, \stmt)}
\;{(\textsc{AtomicBlock})}
$
\medskip
%%%%%%%%%%%%%%%%%%%%
$
\inferrule
{
\\
}
{\actions \vdash (\vars, \skipstmt;\stmt) \trans (\vars, \stmt)}
\;{\;\;\;\;\;\;\;\;\;\;\;\;(\textsc{Seq})}
$
\medskip
%%%%%%%%%%%%%%%%%%%%
$
\inferrule
{
\vars = \MakeStore{\varsG}{\varsTL}{\varsL} \\
\varsL \not\in \locExpr
}
{\actions \vdash (\vars, \ite{\locExpr}{s_1}{s_2}) \trans (\vars, s_2)}
\;{(\textsc{If-False})}
$
\medskip
%%%%%%%%%%%%%%%%%%%%
$
\inferrule
{
\vars = \MakeStore{\varsG}{\varsTL}{\varsL} \\
\varsL \in \locExpr
}
{\actions \vdash (\vars, \ite{\locExpr}{s_1}{s_2}) \trans (\vars, s_1)}
\;{(\textsc{If-True})}
$
\medskip
%%%%%%%%%%%%%%%%%%%%
$
\inferrule
{
\vars = \MakeStore{\varsG}{\varsTL}{\varsL} \\
\varsL \not\in \locExpr \\
}
{\actions \vdash (\vars, \while{e,\alpha}{\locExpr}{s}) \trans (\vars, \skipstmt)}
\;{(\textsc{While-False})}
$
\medskip
%%%%%%%%%%%%%%%%%%%%
$
\inferrule
{
\vars = \MakeStore{\varsG}{\varsTL}{\varsL} \\
\varsL \in \locExpr \\
\stmt' = \while{e,\alpha}{\locExpr}{\stmt}
}
{\actions \vdash (\vars, \stmt') \trans (\vars, \stmt;\stmt')}
\;{(\textsc{While-True})}
$
%%%%%%%%%%%%%%%%%%%%
}
\caption{Operational semantics for statement}
\label{fig:operational-semantics2}
\end{figure}

Figures~\ref{fig:operational-semantics1} and~\ref{fig:operational-semantics2} together present the operational semantics as a relation $\trans$ between a pair of programs.
These figures use the notation $\MakeStore{\varsG}{\varsTL}{\varsL}$ to denote the concatenation of the stores $\varsG$, $\varsTL$, and $\varsL$.
Rule \textsc{Step} in Figure~\ref{fig:operational-semantics2} allows the program $\ProgCtxt[\varsG][\varsTL][\varsL][\stmt]$ to move 
to $\ProgCtxt[\varsG'][\varsTL'][\varsL'][\stmt']$
if $(\MakeStore{\varsG}{\varsTL}{\varsL}, \stmt)$ can move to $(\MakeStore{\varsG'}{\varsTL'}{\varsL'}, \stmt)$ 
according to any rule in Figure~\ref{fig:operational-semantics2} except for rule \textsc{Assert-False}.
Similarly, rule~\textsc{Fail} allows $\ProgCtxt[\varsG][\varsTL][\varsL][\stmt]$ to fail if $(\MakeStore{\varsG}{\varsTL}{\varsL}, \stmt)$
fails according to rule \textsc{Assert-False}. 
The rules in  in Figure~\ref{fig:operational-semantics2} are straightforward.
The execution of primitive statements results in the store begin updated and the statement being rewritten to $\skipstmt$;
the rule \textsc{Seq} erases the $\skipstmt$ to move control to the next statement.

Rule \textsc{Async} describes the creation of a thread via $\async{P}$.  
There is an implicit yield at the beginning of the freshly-created thread.
The stable predicate and the linear permission associated with this yield are 
the precondition and the input linear permission of $P$.
The new thread is added to the sequence of threads in the program.
Since it is a yielding thread, the invariant that at most one thread is non-yielding is preserved.
Rule \textsc{End} describes the termination of a thread; the terminating thread is removed
from the sequence of threads in the program.
Rules \textsc{Call} and \textsc{Return} describe the call and return of a procedure via $\call{P}$.
There are implicit yields associated with the call and the return.
The stable predicates for the yields at call and return are the precondition and postcondition of $P$ respectively.
The linear permissions for the yields at call and return are the input and output linear permissions of $P$ respectively.


\section{Verification}
\label{sec:verification-appendix}

\begin{figure*}
\scriptsize{
\medskip
%%%%%%%%%%%%%%%%%%%%
$
\inferrule
{
}
{\Refines;\AvailableActions \vdash \skipstmt \leadsto \skipstmt}
\;(\textsc{Skip})
$
\medskip
%%%%%%%%%%%%%%%%%%%%
$
\inferrule
{
}
{\Refines;\AvailableActions \vdash \assert{\locExpr} \leadsto \assert{\locExpr}}
\;(\textsc{Assert})
$
\medskip
%%%%%%%%%%%%%%%%%%%%
$
\inferrule
{
}
{\Refines;\AvailableActions \vdash \yield{e,\lins} \leadsto \yield{e,\lins}}
\;(\textsc{Yield})
$
\medskip
%%%%%%%%%%%%%%%%%%%%
$
\inferrule
{
A \in \AvailableActions
}
{\Refines;\AvailableActions \vdash \call{A} \leadsto \call{A}}
\;(\textsc{Atomic})
$
\medskip
%%%%%%%%%%%%%%%%%%%%
$
\inferrule
{
\Refines;\AvailableActions \vdash s \leadsto s'
}
{
\Refines;\AvailableActions \vdash \ablock{e,\lins}{s} \leadsto s'
}
\;(\textsc{Ablock1})
$
\medskip
%%%%%%%%%%%%%%%%%%%%
$
\inferrule
{
\Refines;\AvailableActions \vdash s \leadsto s'
}
{
\Refines;\AvailableActions \vdash s \leadsto \ablock{e,\lins}{s'}
}
\;(\textsc{Ablock2})
$
\medskip
%%%%%%%%%%%%%%%%%%%%
$
\inferrule
{
P \in \dom(\Refines)
}
{
\Refines;\AvailableActions \vdash \call{P} \leadsto \call{\Refines(P)}
}
\;(\textsc{Proc1})
$
\medskip
%%%%%%%%%%%%%%%%%%%%
$
\inferrule
{
P \not\in \dom(\Refines)
}
{
\Refines;\AvailableActions \vdash \call{P} \leadsto \call{P}
}
\;(\textsc{Proc2})
$
\medskip
%%%%%%%%%%%%%%%%%%%%
$
\inferrule
{
P \not\in \dom(\Refines)
}
{
\Refines;\AvailableActions \vdash \async{P} \leadsto \async{P}
}
\;(\textsc{Async})
$
\medskip
%%%%%%%%%%%%%%%%%%%%
$
\inferrule
{
\Refines;\AvailableActions \vdash \StmtStack \leadsto \StmtStack'
}
{
\Refines;\AvailableActions \vdash (\varsL,\StmtStack) \leadsto (\varsL,\StmtStack')
}
\;(\textsc{Stack})
$
\medskip
%%%%%%%%%%%%%%%%%%%%
$
\inferrule
{
\Refines;\AvailableActions \vdash \StmtStack \leadsto \StmtStack' \\
\Refines;\AvailableActions \vdash \stmt \leadsto \stmt'
}
{
\Refines;\AvailableActions \vdash \StmtStack;\stmt \leadsto \StmtStack';\stmt'
}
\;(\textsc{Seq})
$
\medskip
%%%%%%%%%%%%%%%%%%%%
$
\inferrule
{
\Refines;\AvailableActions \vdash s_1 \leadsto s_1' \\
\Refines;\AvailableActions \vdash s_2 \leadsto s_2'
}
{
\Refines;\AvailableActions \vdash \ite{\locExpr}{s_1}{s_2} \leadsto \ite{\locExpr}{s_1'}{s_2'}
}
\;(\textsc{Ite})
$
\medskip
%%%%%%%%%%%%%%%%%%%%
$
\inferrule
{
\Refines;\AvailableActions \vdash s \leadsto s'
}
{
\Refines;\AvailableActions \vdash \while{e,\alpha}{\locExpr}{s} \leadsto \while{e,\alpha}{\locExpr}{s'}
}
\;(\textsc{While})
$
\medskip
%%%%%%%%%%%%%%%%%%%%
$
\inferrule
{
\Refines;\AvailableActions \vdash \stmt \leadsto \stmt'
}
{
\Refines;\AvailableActions \vdash (\phi,\mods,\psi,\stmt) \leadsto (\phi',\mods',\psi',\stmt')
}
\;(\textsc{Procedure})
$
\medskip
%%%%%%%%%%%%%%%%%%%%
$
\inferrule
{
\Refines;\AvailableActions \vdash \StmtStack \leadsto \StmtStack'
}
{
\Refines;\AvailableActions \vdash (\varsTL, (\varsL, \StmtStack)) \leadsto (\varsTL, (\varsL, \StmtStack'))
}
\;(\textsc{Thread})
$
\medskip\\
%%%%%%%%%%%%%%%%%%%%
$
\inferrule
{
\range(\Refines) \subseteq \AvailableActions \\
\forall (\rho,\alpha,m) \in \range(\actions \circ \Refines).\ (\accessVars(\rho) \cap \Local = \emptyset) \wedge (\alpha = ((\elim{\Local}.\ \alpha) \wedge \Same(\Local))) \\
\forall A \in \AvailableActions.\ \actions(A) = \actions'(A) \\
\forall 1 \le i \le n.\ \Refines;\AvailableActions \vdash T_i \leadsto T_i' \\
\forall P \not\in \dom(\Refines).\ \Refines;\AvailableActions \vdash \procs(P) \leadsto \procs'(P)
}
{
\Refines;\AvailableActions \vdash (\procs, \actions, \ProcLins, \varsG, T_1 \ldots T_n) \leadsto (\procs', \actions', \ProcLins', \varsG, T_1' \ldots T_n')
}
\;(\textsc{Program})
$
\medskip
%%%%%%%%%%%%%%%%%%%%
}
\caption{Program transformation}
\label{fig:program-transformation-full}
\end{figure*}

\begin{figure*}
\scriptsize{
\medskip
%%%%%%%%%%%%%%%%%%%%
$
\inferrule
{
}
{
\lins;\ABlockAny \vdash \skipstmt : \lins
}
\;(\textsc{Skip})
$
\medskip
%%%%%%%%%%%%%%%%%%%%
$
\inferrule
{
}
{
\lins;\ABlockOutside \vdash \assert{\locExpr} : \lins
}
\;(\textsc{Assert})
$
\medskip
%%%%%%%%%%%%%%%%%%%%
$
\inferrule
{
\lins_y \subseteq \lins
}
{
\lins;\ABlockOutside \vdash \yield{e,\lins_y} : \lins
}
\;(\textsc{Yield})
$
\medskip
%%%%%%%%%%%%%%%%%%%%
$
\inferrule
{
\ProcLins(A) = (\lins,\lins')
}
{
\lins;\ABlockInside \vdash \call{A} : \lins'
}
\;(\textsc{Atomic})
$
\medskip
%%%%%%%%%%%%%%%%%%%%
$
\inferrule
{
\ProcLins(P) = (\lins,\lins') \\
}
{
\lins;\ABlockOutside \vdash \call{P} : \lins'
}
\;(\textsc{Proc})
$
\medskip
%%%%%%%%%%%%%%%%%%%%
$
\inferrule
{
\lins_G \subseteq \Global \\
\lins \cup \lins_P \cup \lins_P' \subseteq \ThreadLocal \\
\ProcLins(P) = ((\lins_G,\lins_P),(\lins_G,\lins_P')) \\
}
{
\lins_G,\lins,\lins_P;\ABlockOutside \vdash \async{P} : \lins_G,\lins
}
\;(\textsc{Async})
$
\medskip
%%%%%%%%%%%%%%%%%%%%
$
\inferrule
{
\lins;\ABlockInside \vdash \stmt : \lins' \\
\lins_a \subseteq \lins
}
{
\lins;\ABlockOutside \vdash \ablock{e,\lins_a}{\stmt} : \lins'
}
\;(\textsc{Ablock})
$
\medskip
%%%%%%%%%%%%%%%%%%%%
$
\inferrule
{
\lins;\ABlockAny \vdash \StmtStack : \lins'
}
{
\lins;\ABlockAny \vdash (\varsL,\StmtStack) : \lins'
}
\;(\textsc{Stack})
$
\medskip
%%%%%%%%%%%%%%%%%%%%
$
\inferrule
{
\lins;\ABlockAny \vdash \StmtStack : \lins' \\
\lins';\ABlockAny \vdash \stmt : \lins''
}
{
\lins;\ABlockAny \vdash \StmtStack;\stmt : \lins''
}
\;(\textsc{Seq})
$
\medskip
%%%%%%%%%%%%%%%%%%%%
$
\inferrule
{
\lins;\ABlockAny \vdash \stmt_1 : \lins' \\
\lins;\ABlockAny \vdash \stmt_2 : \lins'
}
{
\lins;\ABlockAny \vdash \ite{\locExpr}{\stmt_1}{\stmt_2} : \lins'
}
\;(\textsc{Ite})
$
\medskip
%%%%%%%%%%%%%%%%%%%%
$
\inferrule
{
\lins;\ABlockAny \vdash \stmt : \lins
}
{
\lins;\ABlockAny \vdash \while{e,\alpha}{\locExpr}{\stmt} : \lins
}
\;(\textsc{While})
$
\medskip
%%%%%%%%%%%%%%%%%%%%
$
\inferrule
{
\ProcLins(P) = (\lins,\lins') \\
\procs(P) = (\phi, \mods, \psi, \stmt) \\
\lins;\ABlockOutside \vdash \stmt : \lins'
}
{
\vdash P
}
\;(\textsc{Procedure})
$
\medskip
%%%%%%%%%%%%%%%%%%%%
$
\inferrule
{
T = (\varsTL, (\varsL, \StmtStack)) \\
\lins;\ABlockOutside \vdash \StmtStack : \lins'
}
{
\lins \vdash T
}
\;(\textsc{Thread})
$
\medskip
%%%%%%%%%%%%%%%%%%%%
$
\inferrule
{
\ProcLins(A) = (\lins,\lins') \\
\lins \cap \Global = \lins' \cap \Global \\
\actions(A) = (\rho, \alpha, m) \\\\
\forall (\sigma,\sigma') \in \alpha.\ 
  \disjoint(\{\sigma(x) \mid x \in \lins\}) \Rightarrow
  \disjoint(\{\sigma'(x) \mid x \in \lins'\}) \\
\forall (\sigma,\sigma') \in \alpha.\ 
  \bigcup\{\sigma'(x) \mid x \in \lins'\} \subseteq \bigcup\{\sigma(x) \mid x \in \lins\}
}
{
\vdash A
}
\;(\textsc{Action})
$
\medskip
%%%%%%%%%%%%%%%%%%%%
$
\inferrule
{
\forall P \in \ProcName. \vdash P \\
\forall A \in \ActionName. \vdash A \\
\lins_G \subseteq \Global \\
\forall 1 \le i \le n. (\lins_i \subseteq \ThreadLocal) \\
\forall 1 \le i \le n. (\lins_G,\lins_i \vdash T_i) \\
\forall 1 \le i \le n. (T_i = (\varsTL_i, \ldots)) \\
\disjoint(\{\varsG(x) \mid x \in \lins_G\} \cup
  \{\varsTL_i(x) \mid 1 \le i \le n, x \in \lins_i\})
}
{
\vdash (\procs, \actions, \ProcLins, \varsG, T_1 \ldots T_n)
}
\;(\textsc{Program})
$
\medskip
%%%%%%%%%%%%%%%%%%%%
}
\caption{Linear variables and atomic blocks}
\label{fig:linearity-full}
\end{figure*}

\begin{figure}
\scriptsize{
\medskip
%%%%%%%%%%%%%%%%%%%%
$
\inferrule
{
}
{\procs;\actions;\RefinementAny;\{\} \js \FH{\phi}{\skipstmt}{\phi}}
\;(\textsc{Skip})
$
\medskip\\
%%%%%%%%%%%%%%%%%%%%
$
\inferrule
{
}
{\procs;\actions;\RefinementInside;\{\} \js \FH{\phi \wedge \locExpr}{\assert{\locExpr}}{\phi}}
\;(\textsc{Assert1})
$
\medskip
%%%%%%%%%%%%%%%%%%%%
$
\inferrule
{
}
{\procs;\actions;\RefinementOutside;\{\} \js \FH{\phi}{\assert{\locExpr}}{\phi}}
\;(\textsc{Assert2})
$
\medskip
%%%%%%%%%%%%%%%%%%%%
$
\inferrule
{
}
{\procs;\actions;\RefinementAny;\{\} \js \FH{e}{\yield{e,\lins}}{e}}
\;(\textsc{Yield})
$
\medskip
%%%%%%%%%%%%%%%%%%%%
$
\inferrule
{
\actions(A) = (\rho, \alpha, m) \\ 
\mods \subseteq \ThreadLocal \\
\alpha \Rightarrow \Havoc(\Global \cup \mods \cup \Local) \\
\phi \Rightarrow \rho \\ 
\Unsat{\phi \circ \alpha \circ \neg \psi} \\
}
{\procs;\actions;\RefinementAny;\mods \js \FH{\phi}{\call{A}}{\psi}}
\;(\textsc{Atomic})
$
\medskip
%%%%%%%%%%%%%%%%%%%%
$
\inferrule
{
\procs(P) = (\phi, \mods, \psi, \stmt) \\ P \not \in \dom(\Refines)
}
{\procs;\actions;\RefinementAny;\mods \js \FH{\phi}{\call{P}}{\psi}}
\;(\textsc{Proc1})
$
\medskip
%%%%%%%%%%%%%%%%%%%%
$
\inferrule
{
\procs(P) = (\phi, \mods, \psi, \stmt) \\ P \in \dom(\Refines) \\\\ \procs;\actions;\RefinementAny;\mods \js \FH{\phi}{\call{\Refines(P)}}{\psi}
}
{\procs;\actions;\RefinementAny;\mods \js \FH{\phi}{\call{P}}{\psi}}
\;(\textsc{Proc2})
$
\medskip
%%%%%%%%%%%%%%%%%%%%
$
\inferrule
{
\procs(P) = (\phi, \mods, \psi, \stmt)
}
{\procs;\actions;\RefinementAny;\{\} \js \FH{\rho \wedge \phi}{\async{P}}{\rho}}
\;(\textsc{Async})
$
\medskip
%%%%%%%%%%%%%%%%%%%%
$
\inferrule
{
\procs;\actions;\RefinementAny;\mods \js \FH{\phi_1 \wedge e}{s}{\phi_2}
}
{\procs;\actions;\RefinementAny;\mods \js \FH{\phi_1 \wedge e}{\ablock{e,\lins}{s}}{\phi_2}}
\;(\textsc{Ablock})
$
\medskip
%%%%%%%%%%%%%%%%%%%%
$
\inferrule
{
\procs;\actions;\RefinementAny;\mods_1 \js \FH{\phi_1}{\StmtStack}{\phi_2} \\ 
\procs;\actions;\RefinementAny;\mods_2 \js \FH{\phi_2}{s}{\phi_3}
}
{\procs;\actions;\RefinementAny;\mods_1 \cup \mods_2 \js \FH{\phi_1}{\StmtStack;s}{\phi_3}}
\;(\textsc{Seq})
$
\medskip
%%%%%%%%%%%%%%%%%%%%
$
\inferrule
{
\procs;\actions;\RefinementAny;\mods_1 \js \FH{e \wedge \phi_1}{s_1}{\phi_2} \\ 
\procs;\actions;\RefinementAny;\mods_2 \js \FH{\neg e \wedge \phi_1}{s_2}{\phi_2}
}
{\procs;\actions;\RefinementAny;\mods_1 \cup \mods_2 \js \FH{\phi_1}{\ite{\locExpr}{s_1}{s_2}}{\phi_2}}
\;(\textsc{Ite})
$
\medskip
%%%%%%%%%%%%%%%%%%%%
$
\inferrule
{
\procs;\actions;\RefinementAny;\mods \js \FH{e \wedge \locExpr}{s}{e}
}
{\procs;\actions;\RefinementAny;\mods \js \FH{e}{\while{e,\alpha}{\locExpr}{s}}{e \wedge \neg \locExpr}}
\;(\textsc{While})
$
\medskip
%%%%%%%%%%%%%%%%%%%%
$
\inferrule
{
\phi \Rightarrow \phi' \\ \procs;\actions;\RefinementAny;\mods \js \FH{\phi'}{\StmtStack}{\psi'} \\ \psi' \Rightarrow \psi
}
{\procs;\actions;\RefinementAny;\mods \js \FH{\phi}{\StmtStack}{\psi}}
\;(\textsc{Weaken})
$
\medskip
%%%%%%%%%%%%%%%%%%%%
$
\inferrule
{
\procs;\actions;\RefinementAny;\mods \js \FH{\phi}{\StmtStack}{\psi} \\ \accessVars(\rho) \cap \mods = \{\}
}
{\procs;\actions;\RefinementAny;\mods \js \FH{\rho \wedge \phi}{\StmtStack}{\rho \wedge \psi}}
\;(\textsc{Frame})
$
\medskip
%%%%%%%%%%%%%%%%%%%%
$
\inferrule
{
\procs(P) = (\phi, \mods, \psi, \stmt) \\
\RefinementAny = \RefinementInside \Longleftrightarrow P \in \dom(\Refines) \\\\
\procs;\actions;\RefinementAny;\mods' \js \FH{\phi}{\stmt}{\psi} \\
\mods' \subseteq \mods
}
{
\procs;\actions;\Refines \js P
}
\;(\textsc{Procedure})
$
\medskip
%%%%%%%%%%%%%%%%%%%%
$
\inferrule
{
\procs;\actions;\RefinementAny;\mods \js \FH{\phi'}{\StmtStack}{\psi} \\\\
(\accessVars(\phi) \cup \accessVars(\psi)) \cap \Local = \{\} \\\\
\forall \varsG,\varsTL,\varsL.\ \MakeStore{\varsG}{\varsTL}{\varsL} \in \phi \Rightarrow \MakeStore{\varsG}{\varsTL}{\varsL} \in \phi'
}
{
\procs;\actions;\RefinementAny;\mods \js \FH{\phi}{(\varsL',\StmtStack)}{\psi}
}
\;(\textsc{Stack})
$
\medskip
%%%%%%%%%%%%%%%%%%%%
$
\inferrule
{
T = (\varsTL, (\varsL, \StmtStack)) \\
\MakeStore{\varsG}{\varsTL}{\varsL} \in \phi \\
\procs;\actions;\RefinementOutside;\mods \js \FH{\phi}{\StmtStack}{\true}
}
{
\procs;\actions;\varsG \js T
}
\;(\textsc{Thread})
$
\medskip
%%%%%%%%%%%%%%%%%%%%
$
\inferrule
{
\forall P \in \ProcName.\ \procs;\actions;\Refines \js P \\\\
\forall 1 \le i \le n.\ \procs;\actions;\varsG \js T_i
}
{
\Refines \js (\procs, \actions, \ProcLins, \varsG, T_1 \ldots T_n)
}
\;(\textsc{Program})
$
\medskip
%%%%%%%%%%%%%%%%%%%%
}
\caption{Sequential correctness}
\label{fig:sequential-correctness-full}
\end{figure}

\begin{figure*}
\scriptsize{
\medskip
%%%%%%%%%%%%%%%%%%%%
$
\inferrule
{
}
{\Refines;\actions \jy \skipstmt : (x, x)}
\;(\textsc{Skip})
$
\medskip
%%%%%%%%%%%%%%%%%%%%
$
\inferrule
{
}
{\Refines;\actions \jy \assert{\locExpr} : (x, x)}
\;(\textsc{Assert})
$
\medskip
%%%%%%%%%%%%%%%%%%%%
$
\inferrule
{
}
{\Refines;\actions \jy \yield{e,\lins} : (x, \RM)}
\;(\textsc{Yield})
$
\medskip
%%%%%%%%%%%%%%%%%%%%
$
\inferrule
{
P \not \in \dom(\Refines)
}
{\Refines;\actions \jy \call{P} : (x, \RM)}
\;(\textsc{Call})
$
\medskip
%%%%%%%%%%%%%%%%%%%%
$
\inferrule
{
P \in \dom(\Refines) \\ \actions(\Refines(P)) = (\rho, \alpha, B)
}
{\Refines;\actions \jy \call{P} : (x, x)}
\;(\textsc{Both})
$
\medskip
%%%%%%%%%%%%%%%%%%%%
$
\inferrule
{
P \in \dom(\Refines) \\ \actions(\Refines(P)) = (\rho, \alpha, R)
}
{\Refines;\actions \jy \call{P} : (\RM, \RM)}
\;(\textsc{Right})
$
\medskip
%%%%%%%%%%%%%%%%%%%%
$
\inferrule
{
P \in \dom(\Refines) \\ \actions(\Refines(P)) = (\rho, \alpha, L)
}
{\Refines;\actions \jy \call{P} : (x, \LM)}
\;(\textsc{Left})
$
\medskip
%%%%%%%%%%%%%%%%%%%%
$
\inferrule
{
P \in \dom(\Refines) \\ \actions(\Refines(P)) = (\rho, \alpha, N)
}
{\Refines;\actions \jy \call{P} : (\RM, \LM)}
\;(\textsc{Non})
$
\medskip
%%%%%%%%%%%%%%%%%%%%
$
\inferrule
{
}
{\Refines;\actions \jy \async{P} : (x, \LM)}
\;(\textsc{Async})
$
\medskip
%%%%%%%%%%%%%%%%%%%%
$
\inferrule
{
x \in \{\RM,\CM\}
}
{\Refines;\actions \jy \ablock{e,\lins}{s} : (x, \CM)}
\;(\textsc{Ablock})
$
\medskip
%%%%%%%%%%%%%%%%%%%%
$
\inferrule
{
\Refines;\actions \jy \StmtStack : (x, y) \\ \Refines;\actions \jy s : (y, z)
}
{\Refines;\actions \jy \StmtStack;s : (x, z)}
\;(\textsc{Seq})
$
\medskip
%%%%%%%%%%%%%%%%%%%%
$
\inferrule
{
\Refines;\actions \jy s_1 : (x, y) \\ \Refines;\actions \jy s_2 : (x, y)
}
{\Refines;\actions \jy \ite{\locExpr}{s_1}{s_2} : (x, y)}
\;(\textsc{Ite})
$
\medskip
%%%%%%%%%%%%%%%%%%%%
$
\inferrule
{
\Refines;\actions \jy s : (x, x)
}
{\Refines;\actions \jy \while{e,\alpha}{\locExpr}{s} : (x, x)}
\;(\textsc{While})
$
\medskip
%%%%%%%%%%%%%%%%%%%%
$
\inferrule
{
\procs(P) = (\phi, \mods, \psi, \stmt) \\
\Refines;\actions \jy \stmt : (x, y)
}
{
\Refines;\actions \jy P
}
\;(\textsc{Procedure})
$
\medskip
%%%%%%%%%%%%%%%%%%%%
$
\inferrule
{
\Refines;\actions \jy \StmtStack : (x,y)
}
{
\Refines;\actions \jy (\varsL,\StmtStack) : (x,y)
}
\;(\textsc{Stack})
$
\medskip
%%%%%%%%%%%%%%%%%%%%
$
\inferrule
{
T = (\varsTL, (\varsL, \StmtStack)) \\
\Refines;\actions \jy (\varsL, \StmtStack) : (x, y)
}
{
\Refines;\actions \jy T
}
\;(\textsc{Thread})
$
\medskip
%%%%%%%%%%%%%%%%%%%%
$
\inferrule
{
\forall P \in \ProcName \setminus \dom(\Refines).\ \Refines;\actions \jy P \\\\
\forall 1 \le i \le n.\ \Refines;\actions \jy T_i
}
{
\Refines \jy (\procs, \actions, \ProcLins, \varsG, T_1 \ldots T_n)
}
\;(\textsc{Program})
$
\medskip
%%%%%%%%%%%%%%%%%%%%
}
\caption{Yield elimination}
\label{fig:yield-elimination}
\end{figure*}

The essence of the rules in Figure~\ref{fig:yield-elimination}
is to capture the effect of a computation as a pair $(x,y)$ where $x,y \in \{\RM,\CM,\LM\}$;
the meaning is that to simulate the computation the automaton moves from state $x$ to state $y$.
Rule \textsc{Program} performs checking for all threads and for all procedures not in $\dom(\Refines)$ since these are the 
only procedures that remain in the abstract program.
Rules \textsc{Thread}, \textsc{Stack}, and \textsc{Procedure} are straightforward.
Rule \textsc{While} checks that the body of the loop leaves the state of the automaton unchanged.
Rule \textsc{Ite} checks that the effect both branches is the same.
Rule \textsc{Seq} composes the effect of $\stmt_1$ with the effect of $\stmt_2$.
Rules \textsc{Both}, \textsc{Right}, \textsc{Left}, \textsc{Non}, and \textsc{Ablock} essentially 
examine the available edges in the automaton to validate the statement and calculate its effect.
Rule \textsc{Async} is interesting because it treats an asynchronous call as a left mover.
Rules \textsc{Yield} and \textsc{Call} treat their statements as a yield.
Rules \textsc{Skip} and \textsc{Assert} leave the state of the automaton unchanged.
