Although significant progress has been made in developing verifiers for sequential programs, practical verifiers for concurrent programs remain elusive.
Concurrent systems software, in particular, is complex and difficult to verify. 
Verifying existing software is difficult -- even when partial specifcations are available, since one must extract the design intent and the programming language used may not be conducive to verification. 
Instead, in many different contexts, researchers have pursued a refinement approach where a simpler high-level description is refined, through several phases, down to an optimized and more highly-concurrent implementation. 
However, currently there is no such tool-supported top-down method that can support the refinement of a very high, user-level specification down to the level of an implementation described in terms of individual memory accesses, which is ultimately what is needed for correctness of systems software. 
Industrial-scale concurrent software has been verified~\cite{hypervisor} but working off of a flat, implementation-level description and through a large amount of human effort. 
In this paper, we present such a proof approach and an intermediate language (\civl) and a verification tool that support it. 
We demonstrate our approach on a realistic concurrent garbage collector (GC) refined all the way down from atomic actions describing memory allocation and field accesses and the garbage collector summarized by ``skip'' to individual memory accesses performed on the hardware. 

In our proof method, a program is described at several different levels of abstraction. 
All of these descriptions as well as the refinement relationships connecting them reside in the same \civl file, the input to our tool. 
When going from a lower-level description to a higher-level one, some variables are hidden, and some procedures are proven atomic and replaced by their atomic action specifications. 
Alternatively, atomic actions in a high-level description are refined to more fine-grain, optimized implementations which may include more implementation-level variables. 
Modular specifications and proof annotations such as program and location invariants and procedure pre- and post-conditions, are specified separately at each level in terms of the variables and procedures visible at that level. 
As a result, for instance in the GC, the top-level specifications and annotations reflect the client's view in terms of an idealized infinite heap and field relationships between objects, 
 whereas lower levels represent different views of the implementation, referring to, e.g., atomic updates of objects' colors or lock operations.
In addition to making it possible to verify a specifications at the most abstract level possible, we find that such an approach is a natural way to design a program to be verified. 
Concurrent program refinement tools and techniques in the literature have typically been applied to small examples to relate representations that are relatively close in terms of level of abstraction, such as the abstract specification of a concurrent stack vs. a lock-based implementation. 
Provably correct systems software requires correct refinement from a very high-level description down to individual memory updates. To the best of our knowledge, there are no existing tools demonstrated to support a refinement proof of a realistic system that bridges a large abstraction gap as in our GC proof. 

The verification method, language, and tool design have all been carried out with an eye toward carrying out a modular hierarchical refinement proof in a way that feels natural to a user and is computationally efficient. Our method integrates as a sound proof system and verification tool the concurrent program verification techniques described below. 
The first verification technique available in \civl is a generalization of the Owicki-Gries method~\cite{OwickiG76} for non-preemptive concurrency, 
allowing arbitrary code to appear inside an atomic action. 
This simple but surprisingly powerful generalization allows us, among other things, to compactly encode rely-guarantee reasoning~\cite{Jones83}
and to reuse non-interference specifications from one program location to another.
The second verification technique in \civl is an implementation of linear type checking.
Linear types~\cite{Wadler90lineartypes} and separation logic~\cite{Reynolds02} enable substructural reasoning about data types
(e.g. using separating conjunction or linear maps~\cite{LahiriQW11}).
Perhaps surprisingly, we also use linear types to encode the notion of unique thread identifiers in order to reason about non-interference.
The third verification technique in \civl is atomicity refinement through Lipton-style reduction, mover actions, abstraction, and variable hiding.
Reasoning using coarser-grain actions in this manner both reduces the annotation burden and the computational cost of verification. 
All of these techniques have been used previously and they are each best suited for proof tasks of a particular nature. 
The sound combination of the techniques in a proof system enabled us to use them in a symbiotic manner for verifying refinement. 
For instance, stable location invariants verified using Owicki-Gries-style reasoning facilitated verification of atomicity refinement which, combined with variable hiding, allowed the use of simpler location invariants and rely-guarantee annotations for a higher-level version of the program. 
During the proof of the GC, we made use of each of these techniques and it would have been extremely difficult and computationally intensive if not impossible to carry out the entire proof using only one of the techniques. 

Design choices in our concurrent garbage collector, an extension of the collector by Dijkstra et. al. \cite{dijk78}  with handshakes \cite{doli93,doli94} and repeated scans, were made with verified refinement and high performance in mind. 
Our collector is not as complex as some popular ones that use a snapshot-based approach \cite{doli93,doli94,doma00,azat03} or that contain complex phases and write-barrier properties \cite{boeh91,prin00a,bara05} in order to allow a ``no-black-to-white'' invariant to hold continuously throughout the execution. 
Yet, our collector supports a garbage collection thread running concurrently with multiple mutator threads and its performance is competitive with well-known modern collectors. 
To the best of our knowledge this is a new collector that has not been previously proposed in the literature or in practice. 

The verification of the collector started from atomic action specifications for object allocation and field acccesses by mutator threads, and a specification of ``skip'' for the collector -- that it have no effects visible to the mutator threads. 
To the best of our knowledge, this is the first such automated verification of a practical, executable garbage collector implementation as contrasted with verification of garbage collection algorithms. 
In refinement proofs among lower phases, atomicity of lock implementations, lock-protected field accesses and data structure (mark stack) operations were verified. 
Later phases built larger atomic procedures from these primitives and verified garbage collector invariants involving garbage collector phases, colors used in the garbage-collection algorithm and field relationships among objects, and handshakes between mutator threads and the garbage collector. 
Using these invariants and variable hiding, we were able to verify the top-level specification of  ``skip'' for methods of the garbage collector and atomic action specifications for methods called by mutators. 
Atomicity refinement reasoning and variable hiding were crucial for the proof. 
Reasoning about lower-level lower-level issues, e.g., lock implementations, atomicity of lock-protected code blocks, and correctness of blocks containing racy reads critical for performance was cleanly separated from reasoning about complex invariants and annotations involving the garbage collection algorithm assuming the atomicity of actions in an algorithm-level description of the GC.

Key contributions of this paper are 
\begin{itemize}
\item The design and verified refinement of a realistic concurrent garbage collector from a very high-level description (``skip'') down to individual memory accesses
\item A modular refinement proof system, tool, and intermediate verification language to support this effort
\item A sound proof system that combines reduction, abstraction, Owicki-Gries- and rely-guarantee-style reasoning for powerful refinement verification
\end{itemize}

%%% Local Variables: 
%%% mode: latex
%%% TeX-master: "paper"
%%% End: 

