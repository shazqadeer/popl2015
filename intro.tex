Verifying concurrent software systems is difficult.
The verification of existing software starting from source code requires inferring the design intent and is even more difficult. 
Researchers have instead advocated a refinement approach in which a simple high-level description is refined, 
through several phases, down to an optimized, highly-concurrent implementation.
However, there is as yet no tool-supported method for refinement of realistic systems software down to an executable implementation.
In this paper, we present such a refinement method, and an intermediate language  and a verification tool (\civl) that support it.
We demonstrate our approach on a realistic concurrent garbage collection (GC) algorithm.
The high level specification for the GC consists of atomic action specifications for memory allocation and field accesses, 
and the garbage collector has a specification of ``skip.'' 
This description is refined down to a highly-concurrent implementation described in terms of individual memory accesses.

In our proof method, a program is described at several different levels of abstraction. 
All of these descriptions as well as the refinement relationships connecting them reside in the same \civl file, the input to our tool. 
Atomic actions in a high-level description are refined to more fine-grain, optimized lower-level implementations. 
An implementation of an atomic action may make use of additional variables which are hidden in the action's specification. 
Modular specifications and proof annotations such as program and location invariants and procedure pre- and post-conditions, 
are specified separately, independently at each level in terms of the variables visible at that level. 
For instance, in the GC, the top-level specifications and annotations reflect the client's view 
in terms of an idealized infinite heap and field relationships between objects, 
whereas lower levels represent different views of the implementation, referring to, 
e.g., atomic updates of objects' colors or lock operations.
In addition to making it possible to verify specifications at the most abstract level possible, 
we find that such an approach is a natural way to design a program to be verified.  


The verification method, language, and tool design have all been carried out with the goal of supporting  modular refinement 
verification in a natural and computationally efficient manner. 
\civl integrates in a sound proof system three concurrent program verification techniques that collectively enable refinement verification. 
The first of these is a generalization of the Owicki-Gries method~\cite{OwickiG76} for non-preemptive concurrency, 
allowing arbitrary code to appear inside an atomic action. 
This simple but surprisingly powerful generalization allows us to compactly encode rely-guarantee reasoning~\cite{Jones83}
and to reuse non-interference specifications from one program location to another.
The second verification technique in \civl is an implementation of linear type checking.
Linear types~\cite{Wadler90lineartypes} and separation logic~\cite{Reynolds02} enable substructural reasoning about data types
(e.g. using separating conjunction or linear maps~\cite{LahiriQW11}).
Linear types can also be used to encode the notion of unique thread identifiers useful for reasoning about non-interference. 
The third technique in \civl is atomicity refinement through reduction~\cite{Lipton75}, abstraction, and variable hiding. 
Atomicity refinement allows reasoning coarser-grain atomic actions at higher levels of abstraction, 
and reduces the annotation burden and the computational cost of verification. 
The sound, symbiotic combination of the techniques in a proof system results in a powerful refinement tool. 

This combination of techniques helps manage the complexity of our verification of a realistic GC algorithm.  
In particular, although our algorithm is based on an earlier algorithm by Dijkstra's~\cite{dijk78}, 
it extends the earlier algorithm with various modern optimizations and embellishments to improve generality and performance.  
These extensions include lower write barrier overhead, phase-based synchronization and handshaking, 
and coordination between the GC and mutator threads during root scanning; our use of linearity aids the proof of root scanning, 
while our rely-guarantee encoding aids management of colors inside the write barrier.  
Furthermore, our encoding of the algorithm in \civl spans a wide range of abstraction, 
from low-level memory operations all the way up to high-level specifications; 
we used 6 levels of refinement to help hide low-level details from the high-level portions of the verification.

To summarize, the key contributions of this paper are as follows:
\begin{itemize}
\item 
A modular and automated proof system to support refinement reasoning for concurrent programs.
\item 
An implementation of the proof system as a conservative extension to the Boogie programming language and verifier.
\item 
The design and verified implementation, comprising 2100 lines of code, of a realistic concurrent GC algorithm.
\end{itemize}

%To force commit

%%% Local Variables: 
%%% mode: latex
%%% TeX-master: "paper"
%%% End: 

