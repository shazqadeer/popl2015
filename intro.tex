Practical verifiers for concurrent programs remain elusive.  Concurrent systems software, in particular, is complex and difficult to verify. It is even more difficult to verify existing software starting from source code and partial specifications, since one must infer the design intent from the code.  Instead researchers have pursued a refinement approach in which a simple high-level description is refined, through several phases, down to an optimized, highly-concurrent implementation.  However, currently there is no such tool-supported top-down method that can support the refinement of realistic systems software down to an executable implementation.  In this paper, we present such a proof approach and an intermediate language (\civl) and a verification tool that support it.  We demonstrate our approach on a realistic concurrent garbage collector (GC). The high level specification consists of atomic action specifications for memory allocation and field accesses, and the garbage collector has a specification of``skip.'' Refinement is carried out down to an implementation individual memory accesses performed on the hardware.

In our proof method, a program is described at several different levels of abstraction. 
All of these descriptions as well as the refinement relationships connecting them reside in the same \civl file, the input to our tool. 
When going from a lower-level description to a higher-level one, some variables are hidden, and some procedures are proven atomic and replaced by their atomic action specifications. 
Alternatively, atomic actions in a high-level description are refined to more fine-grain, optimized implementations which may include more implementation-level variables. 
Modular specifications and proof annotations such as program and location invariants and procedure pre- and post-conditions, are specified separately at each level in terms of the variables and procedures visible at that level. 
As a result, for instance in the GC, the top-level specifications and annotations reflect the client's view in terms of an idealized infinite heap and field relationships between objects, 
 whereas lower levels represent different views of the implementation, referring to, e.g., atomic updates of objects' colors or lock operations.
In addition to making it possible to verify a specifications at the most abstract level possible, we find that such an approach is a natural way to design a program to be verified. 
% Concurrent program refinement tools and techniques in the literature have typically been applied to small examples to relate representations that are relatively close in terms of level of abstraction, such as the abstract specification of a concurrent stack vs. a lock-based implementation. 
% Provably correct systems software requires correct refinement from a very high-level description down to individual memory updates. 
To the best of our knowledge, there are no existing tools demonstrated to support a refinement proof of a realistic system that bridges a large abstraction gap as in our GC proof. 

The verification method, language, and tool design have all been carried out with an eye toward carrying out a modular hierarchical refinement proof in a way that feels natural to a user and is computationally efficient. 
In the service of the refinement verification goal, our tool integrates as a sound proof system three concurrent program verification techniques. 
The first \civl is a generalization of the Owicki-Gries method~\cite{OwickiG76} for non-preemptive concurrency, 
allowing arbitrary code to appear inside an atomic action. 
This simple but surprisingly powerful generalization allows us, among other things, to compactly encode rely-guarantee reasoning~\cite{Jones83}
and to reuse non-interference specifications from one program location to another.
The second verification technique in \civl is an implementation of linear type checking.
Linear types~\cite{Wadler90lineartypes} and separation logic~\cite{Reynolds02} enable substructural reasoning about data types
(e.g. using separating conjunction or linear maps~\cite{LahiriQW11}).
We also use linear types to encode of unique thread identifiers to reason about non-interference.
The third group of techniques in \civl is atomicity refinement through reduction~\cite{Lipton}, mover actions, abstraction, and variable hiding.
Reasoning using coarser-grain actions in this manner both reduces the annotation burden and the computational cost of verification. 
These techniques have been investigated by themselve previously and are each best suited for proof tasks of a particular nature. 
The sound combination of the techniques in a proof system results in a symbiosis resulting in a powerful refinement tool. 

We used a concurrent GC, intended to be used as part of a verified operating system, as a design driver for our proof system and tool. 
Design choices in our GC, an extension of the collector by Dijkstra et. al. \cite{dijk78}  with handshakes \cite{doli93,doli94} and repeated scans, were made with verified refinement and high performance in mind. 
Our collector is not as complex as some popular ones that use a snapshot-based approach \cite{doli93,doli94,doma00,azat03} or that contain complex phases and write-barrier properties \cite{boeh91,prin00a,bara05} in order to allow a ``no-black-to-white'' invariant to hold continuously throughout the execution. 
Yet, our collector supports a garbage collection thread running concurrently with multiple mutator threads and its performance is competitive with well-known modern collectors. 
To the best of our knowledge this is a new collector that has not been previously proposed in the literature or in practice. 

The verification of the collector spanned a large abtraction gap. Using our refinement approach, lower-level implementation issues in the GC were cleanly separated from higher-level invariants and annotations involving the garbage collection algorithm. In refinement proofs among lower phases, atomicity of lock implementations, lock-protected field accesses and data structure (mark stack) operations were verified. Higher-level phases built larger atomic procedures from these primitives and verified garbage collector invariants involving garbage collector phases, colors used in the garbage-collection algorithm and field relationships among objects. For verifying refinement, symbiotic use of the verification techniques in \civl was crucial. Stable location invariants verified using rely-guarantee-style reasoning facilitated verification of atomicity refinement which, combined with variable hiding, allowed the use of simpler location invariants and rely-guarantee annotations for a higher-level version of the program.  This is the first such automated verification of a practical, executable garbage collector implementation as contrasted with verification of garbage collection algorithms, and required integrated use of all of the features of our modular refinement verification tool.


Key contributions of this paper are 
\begin{itemize}
\item a modular and automated proof system to support refinement reasoning for concurrent program, and 
\item the design and verified refinement of a realistic concurrent garbage collector from a very high-level description (``skip'') down to individual memory accesses.
\end{itemize}

%%% Local Variables: 
%%% mode: latex
%%% TeX-master: "paper"
%%% End: 

