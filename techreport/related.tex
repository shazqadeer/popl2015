
\section{Related work}
\label{sec:related}

The key design goal for \civl was to facilitate interactive, verified top-down refinement
of a realistic system potentially via several intermediate
descriptions, down to a detailed implementation.  
This task poses some unique computational and human usability challenges. 
\civl verifies that each description correctly refines the description
at the level above, which requires a collection of correlated
whole-program refinement proofs. 
By contrast, existing work on atomicity refinement for concurrent software is
targeted at a single refinement proof, and not between two whole
programs, rather, between procedure implementations and
specifications. 
Enabling multiple ``stacked'' whole-program refinement proofs
within a single unifying theory and automating this process with tool support is task tackled for the first time by the
\civl theory and verification tool. 
While many of the verification techniques available in \civl appear in
the literature, \civl is the first to make sound, joint use of them to
support multiple layers of refinement. 
Below, we first contrast \civl with approaches for 
refinement in the literature. 
We then put \civl and other techniques in context by contrasting them
on \civl's design driver, our six-layer refinement proof of a
concurrent mark-sweep garbage collector~\cite{gc-techreport}.

Atomic action specifications have been explored by the
\calvin~\cite{FlanaganFQS05,FreundQ04} verifier previously. 
\civl makes a distinction between preemptive and cooperative
semantics, and carries out refinement verification on a procedure body
with cooperative semantics as enabled by movers types and reduction.
\calvin attempts to verify refinement directly on the preemptive
semantics, making only limited use of movers at the lowest-level
representation. 
\calvin, unlike \civl, does not support location invariants and linear
variables. 
It instead requires that a two-state rely predicate modeling all possible
interference on the shared state by other threads be supplied. 
Location invariants can encode rely-guarantee reasoning vice versa. 
However, in practice, carrying out non-interference reasoning only
using rely-guarantee reasoning can be significantly harder from a
human interactive proof standpoint. We discuss this point further on
the garbage collector below. 

\QED~\cite{ElmasQT09} is a simplifier for concurrent programs and is close in spirit to the 
refinement-oriented approach of \civl.
A key distinction between \civl and \QED is the fact that a proof step in \QED is a small rewrite in the concurrent program
that must be justified by potentially expensive reduction and invariant reasoning.
In \QED, abstract versions of a program are obtained by transforming the initial
version by applying a proof script. 
This makes it difficult to organize a proof in \QED by providing a
description of a program at several different levels of abstraction. 
Further, the number of small proof steps directly affect both programmer
and computer effort. 
By contrast, \civl supports large proof steps, in each of which the bodies of several procedures
are automatically replaced by atomic actions, thereby lowering the cost of both interaction and automation.
The non-interference reasoning in \QED is even more limited than \calvin.
\QED supports only global invariants and does not support rely-guarantee reasoning or linear variables.

Liang et al.~\cite{LiangRGSim}) present a method for verifying that procedure
bodies refine atomic specifications. 
The key verification approach is
rely-guarantee reasoning and the refinement (simulation) relation between
a procedure and its specification is constrained so it is preserved under
parallel composition. 
This work also lists a concurrent garbage collector as one of the case
studies. 
Their proof corresponds to a few layers of our proof, in particular,
does not refine the implementation to the level of hardware
instructions. 
Since this work uses different languages to describe the high-level
and low-level programs, it is not immediately possible to carry out a
multi-level stepwise refinement proof. 
This work does not provide any tool support for verifying refinement. 

Turon and Wand~\cite{TuronM11} use ownership disciplines and
separation logic to verify refinement of atomic specifications by 
concurrent data structure implementations. Rely-guarantee reasoning is
supported to provide compositionality and non-interference
arguments. This work targets a single refinement step between atomic
specifications for methods and their implementations. 
No tool support for this verification method is provided. 

Verifying linearizability of concurrent data structures (see, e.g.,
\cite{tacasLin,aliLin}) can be viewed as an instance of one-level of
refinement in our setting. \civl can be used for mechanical
verification of linearizability, as we did for the Treiber stack, for
instance. Tools and techniques specific to verifying linearizability
cannot, however, be easily generalized for stepwise refinement proofs
through multiple levels. 

Refinement proofs
between implementations and specifications of protocols have been
investigated using the TLA+~\cite{Lamport2004} specification
language. 
Compositional proofs between specifications and
implementations consisting of modules~\cite{AbadiAssumeGuarantee} have
been investigated in this context. 
Modular refinement proofs for hardware systems have been investigated extensively
(e.g.,~\cite{Henzinger1999,Eiriksson2000}) using the SMV~\cite{McMillan00} and Mocha~\cite{AlurHMQRT98} 
model checking tools.
For concurrent, shared-memory
software, there is a significant cognitive gap between the imperative
multithreaded program text as written in, e.g., \civl, and the TLA+,
SMV or Mocha descriptions for the program. The latter encode the
semantics of an imperative concurrent program as a
state-transition relation. 
Proofs in TLA+, SMV and Mocha, whether modular or not, are
distinguished from proofs in \civl and other techniques that operate
off of the imperative program text in this key regard. 
Reasoning on a
semantic model of a program rather than the program text itself, for
shared-memory concurrent programs, adds a significant amount of
cognitive load to the user. 
It also makes it difficult to structure a
proof along the syntactic structure of the imperative program. 

We now contrast techniques discussed above and \civl on our design
driver, the stepwise refinement of a garbage collector~\cite{gc-techreport} which spans six levels of abstraction.
Each of these six refinement proofs is made feasible by a different
blend of the techniques in \civl. 
While other refinement techniques have also used garbage collectors as
case studies, the refinement tasks tackled there correspond only to one or two of the levels in 
our refinement proof. 
Accordingly, the techniques only target the refinement verification
challenges apparent at those levels of the proof. 
In the top few levels of the garbage collector proof, location
invariants were the key tool we used, while reduction simplified our
job some by enabling us to use coarser atomic actions and fewer
location invariants.  
In the lowest few levels of the garbage collector proof, where
correctness of synchronization primitives were proven, we made
relatively little use of location invariants, and made heavier use of
reduction.
For instance, when verifying the correctness of a
lock-protected concurrently-accessed data structure, ownership
arguments, separation logic, or \QED-style atomicity may suffice. 
But,
at a higher level of abstraction, atomicity alone or ownership or
separation logic arguments alone run into difficulty. 
Rely-guarantee,
or even better, location invariants and Owicki-Gries style reasoning
are best suited to carry out the correctness arguments here. 
All of these
techniques are all aided significantly when reduction is available as
a reasoning technique to enable reasoning about coarser atomic blocks
and fewer interleaving locations. 
\civl combines, in one tool and consistent theory, the ability to
deploy and combine these techniques for a refinement proof spanning
many different descriptions of a program at many different levels of
abstraction. 

Reasoning about non-interference using only rely-guarantee reasoning requires the use of potentially a large number of auxiliary variables
and a rely predicate for each thread that is valid at each interleaving point in
the preemptive semantics. 
Such a rely predicate is difficult to construct and maintain from a
human standpoint.
In phases ... our garbage collector proof, each location invariant in
the body of the mark procedure and the location invariants in the
mutators were different. 
Encoding this using rely-guarantee reasoning would have necessitated
the use of auxiliary variables that correspond to the program counter
of each thread active during the mark phase. 
Being able to simply attach the location invariant to the
corresponding ``yield'' instead was instrumental in the interactive
refinement proof at this level. 
Variable hiding
is an important capability when relating multiple versions of a
program at different levels of abstraction. 
In our gargabe collector
proof, the top-level description hides almost all implementation
variables to provide an idealized interface to application threads. 


{\bf Other approaches.}

Overview of what ``other approaches'' means: There is a single
program, or a concurrent object and its method specification. Methods
help combat complexity and structuring of proof on this flat
description. These approaches are *complementary* to what we do.

VCC~\cite{VCC} is a tool for verifying concurrent C programs.  
Chalice~\cite{LM09} is a language and modular verification tool for concurrent programs. 
VCC does not support refinement and Chalice does so only for sequential programs;
unlike these tools, \civl is designed to verify refinement for concurrent programs.  
VCC and Chalice base their invariant reasoning on objects, object ownership, and type invariants. 
Invariant reasoning in \civl is more primitive and based on predicates in yield statements. 
Although the approach in VCC and Chalice is more convenient when applicable, \civl's approach is more flexible. 
VCC and Chalice can reason sequentially about objects exclusively owned by a thread;
\civl accomplishes the same using linear variables.
Neither VCC nor Chalice support movers and reduction reasoning.

Concurrent separation logic~\cite{OHearn07} reasons about concurrency without 
explicitly checking for non-interference between threads. 
Recently, tools based on this logic that blend in explicit non-interference reasoning have been developed~\cite{SAGL,RGSep}. 
\civl's combination of interference checking and linear variables is an extreme example of this trend,
supplying very primitive abstractions and letting programmers mix and
match these abstractions freely.
%agnostic, general, encode thread identifiers, permissions
%and uses these to support refinement
%non-interference on higher or lower-level

Is there any other POPL-style work that needs to be cited?


%% A key challenge in specifying and verifying multi-threaded software, differently from sequential software, is that procedures do not provide a clean factoring of a program into modules. 
%% For sequential software, in the popular software verification approach based on the method of Floyd~\cite{Floyd67} and Hoare~\cite{Hoare69}, 
%% the verification of a large software component is performed modularly by verifying each procedure in it separately.
%% This modularity is enabled by the use of interface specifications (preconditions and postconditions) constraining the behavior of each procedure.
%% It is well-known that the Floyd-Hoare method is difficult to generalize to reasoning about concurrent shared-memory programs,
%% primarily because the execution of a procedure may be interfered with by concurrently executing threads.
%% Several attempts have been made to extend Floyd-Hoare reasoning to deal with concurrent interference, 
%% including classical approaches such as location invariants~\cite{Ashcroft75,OwickiG76} and rely-guarantee~\cite{Jones83},
%% and other more recent approaches~\cite{OHearn07,RGSep}. 
%% In this paper, our goal is orthogonal and complementary to these approaches to combating the complexity of verifying safety properties.
%% We provide a methodology in which the {\em refinement} problem can be stated in a modular way guided by the syntactic structure of the program as in the Floyd-Hoare method. 
%% We check refinement by reasoning about the code of a single procedure at a time, in fact, one interference-free step of the implementation at a time. 

% In this section, we discuss related work on verification of concurrent systems.
% We first discuss approaches based on refinement and later approaches more directly inspired 
% by Floyd-Hoare reasoning.

% A generic, widely-applicable approach to specifying and verifying refinement for shared-memory concurrent software is lacking. 
%For many systems, a natural way to write a full functional specification is to provide a description of the entire system at an abstract level. 
%Checking refinement in this context is a means for full functional verification. 
%Alternatively, a series of increasingly more abstract models of a system can be used to reduce the computational cost of verifying safety properties, since safety properties of higher-level models are preserved by lower-level models. 
%This use of refinement is complementary to other techniques that combat the difficulty of verification, such as Floyd-Hoare, rely-guarantee, separation logic for modularity. 

%%{\bf Refinement-oriented verification.}  

%*** Hammer home ***
%*** Two full programs ***
%The first: Layered refinement proofs that stack on top of each other.
%Reverse order. 

%Much shorter
%Rely-guarantee vs location invariants. Allows us to encode
%rely-guarantee, but location invariants is convenient. 
%Make this point in the GC proof. ONLY. 

%The ability to reason on cooperative semantics using location-specific
%annotations lowers the complexity of annotations required for the refinement proof.

%Non-interference
%Summarize in a few sentences
