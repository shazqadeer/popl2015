\section{Related work}
\label{sec:related}

\civl draws inspiration from Calvin~\cite{FlanaganFQS05} for atomic action specifications
and from QED~\cite{ElmasQT09} for the use of reduction for program simplification.
However \civl has important advantages over these tools.
First, \civl supports the specification and verification of invariants for each program location 
and explicit non-interference verification including the use of linear variables. 
Second, \civl is designed to support large proof steps orchestrating assertion reasoning, 
refinement checking, and yield elimination.
These novel aspects of \civl are key to the construction of proofs spanning a large abstraction gap
between specification and implementation.

VCC~\cite{VCC} is a tool for verifying concurrent C programs using the Z3 SMT solver.  
Chalice~\cite{LM09} is a language and modular verification tool for concurrent programs. 
\civl is designed to verify refinement for concurrent programs.  
VCC does not support refinement and Chalice does so only for sequential programs. 
Neither VCC nor Chalice support movers and reduction reasoning.
VCC and Chalice base their invariant reasoning on objects, object ownership, and type invariants. 
Invariant reasoning in \civl is more primitive and based on assertions in yield statements. 
Although the approach in VCC and Chalice is more convenient when applicable, \civl's approach is more flexible. 
VCC and Chalice can reason sequentially about objects exclusively owned by a thread;
\civl accomplishes the same using linear variables.

Concurrent separation logic~\cite{OHearn07} reasons about concurrency without 
explicitly checking for non-interference between threads. 
Recently, tools based on this logic that blend in explicit non-interference reasoning have been developed~\cite{SAGL,RGSep}. 
\civl's combination of interference checking and linear variables is an extreme example of this trend,
supplying very primitive abstractions and letting programmers mix and match these abstractions freely.

Most work on mechanical verification of garbage collectors has focused on sequential collectors~\cite{mccr07,hawb09},
with some support for incremental collection~\cite{mccr07},
but without the fine-grained interleaving that makes concurrent collection so challenging.
Gonthier~\cite{gont96} verified the invariants of a concurrent mark-sweep algorithm,
although not down to the level of individual memory operations
and without using refinement to prove simple atomic specifications.

More recently,
Jagannathan et al.~\cite{Jagannathan14} used a refinement-style Coq proof to verify the atomicity of various GC-related operations
using weak consistency (TSO) rather than sequential consistency.
However, they do not fully verify the collector,
but rather assume invariants needed by the atomicity proofs.
Nevertheless, combining their low-level TSO model with our complete,
higher-level proof could produce a practical verified collector for modern hardware.

\section{Conclusions}

By combining a generalized implementation of interference checking, a powerful linear type checking system, 
and a refinement verification system that supports reduction and abstraction, 
\civl supports many forms of reasoning about concurrency without committing programmers to a particular methodology.
Two-state rely-guarantee invariants arise naturally from procedures that encapsulate one-state yield statements,
and disjoint thread identifiers arise naturally from the linear type system. 
The combination of these techniques with reduction and abstraction supports refinement verification, 
allowing users to formally relate representations of program described at different levels of abstraction and atomicity. 

%%% Local Variables: 
%%% mode: latex
%%% TeX-master: "paper"
%%% End: 
