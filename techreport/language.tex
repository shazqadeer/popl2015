\begin{figure}
\setlength{\tabcolsep}{3pt}
\begin{tabular}{rclcl}
$g$ & $\in$ & $\Global \subseteq \VarName$ \\
$\tl$ & $\in$ & $\ThreadLocal \subseteq \VarName$ \\
$l$ & $\in$ & $\Local \subseteq \VarName$ \\
$x,y$ & $\in$ & $\Var = \Global \cup \ThreadLocal \cup \Local$ \\
$v$ &  $\in$ & $\Value$ \\
$\sigma$ & $\in$ & $\Store = \Var \rightarrow \Value$ \\
$\varsG$ & $\in$ & $\StoreGlobal = \Global \rightarrow \Value$ \\
$\varsTL$ & $\in$ & $\StoreThreadLocal = \ThreadLocal  \rightarrow \Value$ \\
$\varsL$ & $\in$ & $\StoreLocal = \Local \rightarrow \Value$ \\
$e, \phi, \psi, \rho, \locExpr$ & $\in$ & $\StateExpr = 2^{\Store}$ \\
$\alpha, \beta$ & $\in$ & $\TransExpr = 2^{(\Store,\Store)}$ \\
$P$ & $\in$ & $\ProcName$ \\
$A$ & $\in$ & $\ActionName$ \\
$m$ & $\in$ & $\Mover = \{B,R,L,N\}$\\
$\actions$ & $\in$ & $\ActionName \rightarrow (\StateExpr, \TransExpr, \Mover)$ \\
$\procs$ & $\in$ & $\ProcName \rightarrow (\StateExpr, 2^\ThreadLocal, \StateExpr, \Stmt)$ \\
$\lins$ & $\in$ & $\LinearVar = 2^{\Global \cup \ThreadLocal}$ \\
$\ProcLins$ & $\in$ & $(\ActionName \cup \ProcName) \rightarrow (\LinearVar, \LinearVar)$ \\
$\ABlockAny$ & $\in$ & $\mathit{InsideABlock} ::= \ABlockInside \mid \ABlockOutside$ \\
$\Refines$ & $\in$ & $\ProcName \pf \ActionName$ \\
$\AvailableActions$ & $\in$ & $2^\ActionName$ \\
$\Perm$ & $\in$ & $\Var \rightarrow \Value$
\end{tabular}
\caption{Definitions}
\label{fig:definitions}
\end{figure}

\begin{figure}
\setlength{\tabcolsep}{3pt}
\begin{tabular}{rclcl}
$\stmt \in \Stmt$ &::= & $\skipstmt \mid \yield{e,\lins} \mid \call{A} \mid \call{P} \mid$ \\
                  & & $\ablock{e,\lins}{\stmt} \mid \stmt;\;\stmt \mid \async{P} \mid$\\
                  & & $\ite{\locExpr}{\stmt}{\stmt} \mid \while{e,\alpha}{\locExpr}{\stmt}$ \vspace{6pt} \\
\end{tabular}
\begin{tabular}{rclcl}
$\StmtStack \in \mathit{StmtStack}$ &::= & $\stmt \mid (\varsL,\StmtStack) \mid \StmtStack;\stmt$ \\
$T \in \mathit{Thread}$ &::= &$(\varsTL, (\varsL, \StmtStack))$ \\
$\Prog \in \Program$ &::= & $\ProgCtxt[\varsG][\varsTL][\varsL][\stmt]$ \vspace{6pt} \\
$\StmtCtxt \in \mathit{StmtCtxt}$ &::= &$[]_{Stmt} \mid \StmtCtxt;\stmt$ \\
$\StmtStackCtxt \in \mathit{StmtStackCtxt}$ &::= & $([]_{\Local}, \StmtCtxt) \mid (L,\StmtStackCtxt) \mid \StmtStackCtxt;\stmt$ \\
$\ThreadCtxt \in \mathit{ThreadCtxt}$ &::= &$([]_{\ThreadLocal}, ([]_{\Local}, \StmtCtxt)) \mid$ \\
 & &$([]_{\ThreadLocal}, (\varsL, \StmtStackCtxt))$ \\
$\ProgCtxt \in \mathit{ProgCtxt}$ &::= & $(\procs, \actions, \ProcLins, []_{\Global}, \TS \cdot \ThreadCtxt \cdot \TS)$ \vspace{6pt} \\
$\YieldingThread \in \mathit{YieldingThread}$ &::= &$\ThreadCtxt[\varsTL][\varsL][\yield{e,\lins}]$ \\
$\XProgram$ &::= & $(\procs, \actions, \ProcLins, \varsG, \YieldingThreads \cdot T \cdot \YieldingThreads)$
\end{tabular}
\caption{Syntax}
\label{fig:syntax}
\end{figure}

\begin{figure}
\scriptsize{
\medskip
%%%%%%%%%%%%%%%%%%%%
$
\inferrule
{
\ProgCtxt[\varsG][\varsTL][\varsL][s] = (\_, \actions, \_, \_, \_) \\
\actions \vdash (\MakeStore{\varsG}{\varsTL}{\varsL}, s) \trans (\MakeStore{\varsG'}{\varsTL'}{\varsL'}, s')
}
{\ProgCtxt[\varsG][\varsTL][\varsL][s] \trans \ProgCtxt[\varsG'][\varsTL'][\varsL'][s']}
\;(\textsc{Step})
$
\medskip
%%%%%%%%%%%%%%%%%%%%
$
\inferrule
{
\ProgCtxt[\varsG][\varsTL][\varsL][s] = (\_, \actions, \_, \_, \_) \\
\actions \vdash (\MakeStore{\varsG}{\varsTL}{\varsL}, s) \fails
}
{\ProgCtxt[\varsG][\varsTL][\varsL][s]) \fails}
\;(\textsc{Fail})
$
\medskip
%%%%%%%%%%%%%%%%%%%%
$
\inferrule
{
\procs(P) = (\phi, \mods, \psi,\stmt) \\
\TS = \Vector{T_1} \cdot \ThreadCtxt[\varsTL][\varsL][\async{P}] \cdot \Vector{T_2} \\
\TS' = \Vector{T_1} \cdot \ThreadCtxt[\varsTL][\varsL][\skipstmt] \cdot \Vector{T_2}
}
{
(\procs, \actions, \ProcLins, \varsG, \TS)
\trans
(\procs, \actions, \ProcLins, \varsG, \TS' \cdot (\varsTL, (\varsL, \stmt)))
}
\;(\textsc{Async})
$
\medskip
%%%%%%%%%%%%%%%%%%%%
$
\inferrule
{
T = (\varsTL, (\varsL,\skipstmt)) 
}
{(\procs, \actions, \ProcLins, \varsG, \Vector{T} \cdot T \cdot \Vector{T'}) \trans (\procs, \actions, \ProcLins, \varsG, \Vector{T} \cdot \Vector{T'})}
\;(\textsc{End})
$
\medskip
%%%%%%%%%%%%%%%%%%%%
$
\inferrule
{
\procs(P) = (\phi, \mods, \psi,\stmt)
}
{\ProgCtxt[\varsG][\varsTL][\varsL][\call{P}] \trans \ProgCtxt[\varsG][\varsTL][\varsL][\stmt]}
\;{(\textsc{Call})}
$
\medskip
%%%%%%%%%%%%%%%%%%%%
$
\inferrule
{
\\
}
{\ProgCtxt[\varsG][\varsTL][\varsL][\Frame{\varsL'}{\skipstmt}] \trans \ProgCtxt[\varsG][\varsTL][\varsL][\skipstmt]}
\;{(\textsc{Return})}
$
%%%%%%%%%%%%%%%%%%%%
}
\caption{Operational semantics for program}
\label{fig:operational-semantics1}
\end{figure}

\begin{figure}
\scriptsize{
\medskip
%%%%%%%%%%%%%%%%%%%%
$
\inferrule
{
\actions(A) = (\rho, \alpha, m) \\
\vars \not \in \rho
}
{
\actions \vdash (\vars, \call{A}) \fails
}
\;{(\textsc{Atomic-False})}
$
\medskip
%%%%%%%%%%%%%%%%%%%%
$
\inferrule
{
\actions(A) = (\rho, \alpha, m) \\
\vars \in \rho \\
(\vars, \vars') \in \alpha
}
{
\actions \vdash (\vars, \call{A}) \trans (\vars',\skipstmt)
}
\;{(\textsc{Atomic-True})}
$
\medskip
%%%%%%%%%%%%%%%%%%%%
$
\inferrule
{
\\
}
{\actions \vdash (\vars, \yield{e,\lins}) \trans (\vars, \skipstmt)}
\;{(\textsc{Yield})}
$
\medskip
%%%%%%%%%%%%%%%%%%%%
$
\inferrule
{
}
{\actions \vdash (\vars, \ablock{e,\lins}{\stmt}) \trans (\vars, \stmt)}
\;{(\textsc{AtomicBlock})}
$
\medskip
%%%%%%%%%%%%%%%%%%%%
$
\inferrule
{
\\
}
{\actions \vdash (\vars, \skipstmt;\stmt) \trans (\vars, \stmt)}
\;{\;\;\;\;\;\;\;\;\;\;\;\;(\textsc{Seq})}
$
\medskip
%%%%%%%%%%%%%%%%%%%%
$
\inferrule
{
\vars \not\in \locExpr
}
{\actions \vdash (\vars, \ite{\locExpr}{s_1}{s_2}) \trans (\vars, s_2)}
\;{(\textsc{If-False})}
$
\medskip
%%%%%%%%%%%%%%%%%%%%
$
\inferrule
{
\vars \in \locExpr
}
{\actions \vdash (\vars, \ite{\locExpr}{s_1}{s_2}) \trans (\vars, s_1)}
\;{(\textsc{If-True})}
$
\medskip
%%%%%%%%%%%%%%%%%%%%
$
\inferrule
{
\vars \not\in \locExpr \\
}
{\actions \vdash (\vars, \while{e,\alpha}{\locExpr}{s}) \trans (\vars, \skipstmt)}
\;{(\textsc{While-False})}
$
\medskip
%%%%%%%%%%%%%%%%%%%%
$
\inferrule
{
\vars \in \locExpr \\
\stmt' = \while{e,\alpha}{\locExpr}{\stmt}
}
{\actions \vdash (\vars, \stmt') \trans (\vars, \stmt;\stmt')}
\;{(\textsc{While-True})}
$
%%%%%%%%%%%%%%%%%%%%
}
\caption{Operational semantics for statements}
\label{fig:operational-semantics2}
\end{figure}

In this paper, we will formalize our verification method on a core concurrent programming language \civl.
In this section, we present the syntax and operational semantics of our language.
Figure~\ref{fig:definitions} contains basic definitions.
Figure~\ref{fig:syntax} uses these definitions to define the syntax of a (dynamically executing) \civl program.

Informally, a \civl program is comprised of threads each of which has a stack, global variables shared across threads, 
thread-local variables that are shared across procedures, and local variables inside a procedure.
The sets $\Global$, $\ThreadLocal$, and $\Local$ are the names of global, thread-local, and procedure-local variables 
respectively.
These sets are mutually disjoint and $\Var$ denotes their union.
There is an uninterpreted set $\Value$ of values that may be stored in these variables.
$\Store$, $\StoreGlobal$, $\StoreThreadLocal$, and $\StoreLocal$ are all sets of maps into $\Value$
from $\Var$, $\Global$, $\ThreadLocal$, and $\Local$ respectively.
We denote elements of $\Store$, $\StoreGlobal$, $\StoreThreadLocal$, and $\StoreLocal$
by $\sigma$, $\varsG$, $\varsTL$, and $\varsL$ respectively.
Formally, a \civl program is a tuple $(\procs, \actions, \ProcLins, G, \TS)$ with the following components.

\noindent
{\bf Procedures.}
$\procs$ maps a procedure name $P \in \ProcName$ to a tuple $(\phi, \mods, \psi, \stmt)$, 
where $\phi$ is the precondition of $P$, $\mods$ is the set of thread-local variables potentially modified by $P$, 
$\psi$ is the postcondition of $P$, and $\stmt$ is the body of $P$.
The predicates $\phi$ and $\psi$ cannot refer to procedure-local variables.
Procedures do not have parameters, but parameters may be modeled using thread-local variables.

\noindent
{\bf Actions.}
$\actions$ maps an action name $A \in \ActionName$ to a tuple $(\rho,\alpha,m)$.
Actions are used inside procedure bodies to access variables.
The predicate $\rho$ is the gate and the predicate $\alpha$ is the transition relation of $A$;
$\rho$ is a predicate over $\Store$ and $\alpha$ is a predicate over $\Store \times \Store$.
Finally, $m$ is one of four values in $\{B,R,L,N\}$;
it denotes the commutativity type of the action, $B$ for both mover, $R$ for right mover, $L$ for left mover, 
and $N$ for non mover~\cite{FlanaganFLQ08}. 

\noindent
{\bf Linear interfaces.}
$\ProcLins$ maps each procedure and action name in $\ProcName \cup \ActionName$ to a linear interface 
$(\lins,\lins')$, where each of $\lins$ and $\lins'$ is a set comprising global and thread-local variables.
The procedure (or action) requires that the set $\lins$ be available at the beginning of its invocation 
and makes available the set $\lins'$ at the end of its invocation; 
we refer to $\lins$ and $\lins'$ as the input and output linear permissions respectively.

\noindent
{\bf Global store.}
$\varsG$ is the global store of the program.

\noindent
{\bf Threads.}
$\TS$ is a sequence of threads.
Each thread $T$ in $\TS$ is a pair $(\varsTL, (\varsL, \StmtStack))$, where $\varsTL$ is the thread-local 
store and $(\varsL, \StmtStack)$ the continuation of $T$.  
The continuation $(\varsL, \StmtStack)$ comprises a procedure-local store $\varsL$ and a statement stack $\StmtStack$.
Informally, $\varsL$ is the local store at the top of the pending stack of computation $\StmtStack$ of the thread.
A statement stack $\StmtStack$ is recursively defined to be either a statement $\stmt$ 
or a continuation $(\varsL,\StmtStack)$ or the sequential composition $\stmt;\StmtStack$ of a statement $\stmt$ with 
a statement stack $\StmtStack$.

\subsection{Operational semantics}
\label{sec:operational-semantics}

Figures~\ref{fig:operational-semantics1} and~\ref{fig:operational-semantics2} together present the operational semantics as a relation 
$\trans$ between a pair of programs.
The rules for the relation extend the syntax of statements with an embedded context~\cite{WrightF94}.
A statement context $\StmtCtxt$ is either a context $[]_{\Stmt}$ or the sequential composition of a 
statement context and a statement.
Thus, each statement context has a unique context $[]_{\Stmt}$ inside it;
this context encodes the current position of the program execution.
The result of substituting a statement $\stmt$ inside the statement context $\StmtCtxt$ is another statement and is denoted by $\StmtCtxt[\stmt]$.
Similarly, we define a statement stack context $\StmtStackCtxt$ with two contexts embedded inside it, $[]_{\Local}$ for substituting a procedure-local store
and $[]_{\Stmt}$ for substituting a statement.
The result of substituting a procedure-local store $\varsL$ and a statement $\stmt$ inside 
$\StmtStackCtxt$ is a statement stack and is denoted by $\StmtStackCtxt[\varsL][\stmt]$;
here, $\varsL$ is the procedure-local store for the procedure containing the currently executed statement $\stmt$.
A thread context $\ThreadCtxt$ has three contexts embedded inside it, $[]_{\ThreadLocal}$ for substituting a thread-local store, 
$[]_{\Local}$ for substituting a procedure-local store, and $[]_{\Stmt}$ for substituting a statement.
The result of substituting a thread-local store $\varsTL$, a procedure-local store $\varsL$, and a statement $\stmt$ inside $\ThreadCtxt$ 
is a thread and is denoted by $\ThreadCtxt[\varsTL][\varsL][\stmt]$.
Finally, the result of substituting a global store $\varsG$, a thread-local store $\varsTL$, a procedure-local store $\varsL$, and a statement $\stmt$ inside $\ProgCtxt$ is a program and is denoted by $\ProgCtxt[\varsG][\varsTL][\varsL][\stmt]$.

The operational semantics use the notation $\MakeStore{\varsG}{\varsTL}{\varsL}$ to denote the concatenation of the 
stores $\varsG$, $\varsTL$, and $\varsL$.
Rule \textsc{Step} in Figure~\ref{fig:operational-semantics1} allows the program $\ProgCtxt[\varsG][\varsTL][\varsL][\stmt]$ to move 
to $\ProgCtxt[\varsG'][\varsTL'][\varsL'][\stmt']$
if $(\MakeStore{\varsG}{\varsTL}{\varsL}, \stmt)$ can move to $(\MakeStore{\varsG'}{\varsTL'}{\varsL'}, \stmt)$ 
according to any rule in Figure~\ref{fig:operational-semantics2} except for rule \textsc{Atomic-False}.
Similarly, rule~\textsc{Fail} allows $\ProgCtxt[\varsG][\varsTL][\varsL][\stmt]$ to fail if $(\MakeStore{\varsG}{\varsTL}{\varsL}, \stmt)$
fails according to rule \textsc{Atomic-False}. 
Rule \textsc{Async} describes the creation of a thread via $\async{P}$.  
There is an implicit yield at the beginning of the freshly-created thread.
The stable predicate and the linear permission associated with this yield are 
the precondition and the input linear permission of $P$.
The new thread is added to the sequence of threads in the program.
Since it is a yielding thread, the invariant that at most one thread is non-yielding is preserved.
Rule \textsc{End} describes the termination of a thread; the terminating thread is removed
from the sequence of threads in the program.
Rules \textsc{Call} and \textsc{Return} describe the call and return of a procedure via $\call{P}$.
There are implicit yields associated with the call and the return.
The stable predicates for the yields at call and return are the precondition and postcondition of $P$ respectively.
The linear permissions for the yields at call and return are the input and output linear permissions of $P$ respectively.

It is important to note that all except at most one thread in a program are yielding threads blocked at a yield statement;
such a thread is captured by the symbol $\YieldingThread$ in Figure~\ref{fig:syntax}.
The one (possibly) non-yielding thread is the only one allowed to execute until it reaches a yield statement and becomes a yielding thread.
At that point, all threads are yielding and any one can be picked for execution.

The rules in Figure~\ref{fig:operational-semantics2} provide the operational semantics for the statements in \civl.
The statement $\skipstmt$ has no side effect and is used as a marker in the formal operational semantics.
The statement $\yield{e,\lins}$ yields control to the scheduler which may pick any existing thread to execute.
The predicate $e$, called a stable or yield predicate, is expected to be established by the executing thread and preserved across
execution of other threads subsequent to the yield.
Similarly, the linear permissions $\lins$ are expected to be available as long as the control of the thread
is at the yield.
The statement $\call{A}$ invokes the action $A$.
Suppose $\actions(A) = (\rho, \alpha, m)$.
If $\rho$ does not hold, the statement $\call{A}$ fails, 
otherwise the store is updated according to the transition relation $\alpha$.
The statement $\call{P}$ invokes the procedure $P$;
if $\procs(P) = (\phi, \mods, \psi, \stmt)$, then the precondition $\phi$ is expected to hold at the invocation,
the statement $\stmt$ begins execution, and the postcondition $\psi$ is expected to hold when $\stmt$ finishes.
An important aspect of the semantics of a procedure $P$ is that there is an implicit $\yield{\phi}$ before $\stmt$
and an implicit $\yield{\psi}$ after $\stmt$; therefore $\phi$ and $\psi$ are expected to be preserved by other threads in the environment. 
The statement $\async{P}$ creates a new thread that begins by invoking the procedure $P$.
The statement $\ablock{e,\lins}{\stmt}$ is a block of statements that may not have a yield inside it and consequently executes atomically;
the predicate $e$ is expected to hold and the permissions $\lins$ expected to be available at the beginning.
This statement is useful for compactly encoding non-interference checks (Section~\ref{sec:refinement}).
The statement $\stmt_1;\;\stmt_2$ executes $\stmt_1$ followed by $\stmt_2$.
The statement $\ite{\locExpr}{\stmt_1}{\stmt_2}$ executes $\stmt_1$ if $\locExpr$ is true and otherwise executes $\stmt_2$.
The statement $\while{e,\alpha}{\locExpr}{\stmt}$ executes $\stmt$ repeatedly until $\locExpr$ becomes false.
The expression $e$ is a loop invariant and the expression $\alpha$ is a summary for failure-free executions of the loop that do not go 
through a yield (Section~\ref{sec:refinement}).
The condtional expression $\locExpr$ in $\mathit{if}$ and $\mathit{while}$ statements cannot refer to global variables.

A finite $\trans$-execution from $\Prog$ is a sequence of programs $\Prog_0,\Prog_1,\ldots,\Prog_n$ such that $\Prog_0 = \Prog$ 
and $\Prog_i \trans \Prog_{i+1}$ for all $0 \leq i < n$.
An infinite $\trans$-execution from $\Prog$ is defined similarly.
The program $\Prog$ is {\em unsafe\/} if there is a finite $\trans$-execution from $\Prog$ ending in a state $\Prog'$ 
such that $\Prog'$ fails.
$\Prog$ is {\em safe}, denoted $\Safe(\Prog)$, iff $\Prog$ is not unsafe.

Let $\xtrans$ be the relation $\trans$ restricted to $\XProgram$.
A finite or infinite $\xtrans$-execution from a program $\Prog \in \XProgram$ is defined similarly to a $\trans$-execution.
Define the predicate $\Yielding$ as follows:
\[
\begin{array}{lcl}
\Yielding(\Prog) & = & \mathit{let}\ \Prog = (\procs, \actions, \ProcLins, G, \TS)\ \mathit{in}\\
                 &   & ~~~~~~\forall T \in \TS.\ T \in \mathit{YieldingThread}.
\end{array}
\]
Consider a program $\Prog \in \XProgram$.
$\Prog$ is {\em responsive\/}, denoted $\Responsive(\Prog)$, 
iff every infinite $\xtrans$-execution from $\Prog$ satisfies $\Yielding$ infinitely often.

\subsection{Definitions}
\label{sec:definitions}
We conclude this section by presenting a few definitions of concepts that are used in Section~\ref{sec:verification} to formalize
our verification method.
For any transition relation $\alpha$, let $\elim{\Local}.\ \alpha$ be the transition relation
\[\{(\MakeStore{\varsG}{\varsTL}{\varsL}, \MakeStore{\varsG'}{\varsTL'}{\varsL'}) \mid \exists \varsL_1,\varsL_2.\ (\MakeStore{\varsG}{\varsTL}{\varsL_1}, \MakeStore{\varsG'}{\varsTL'}{\varsL_2}) \in \alpha \}.\]
For any $X \subseteq \Var$, let $\Havoc(X)$, be the transition relation 
$\{(\sigma,\sigma') \mid \forall x \in \Var \setminus X.\ \sigma(x) = \sigma'(x)\}$.
For any partial map $\pi$ from $\VarName$ to $\Value$ and $X \subseteq \dom(\pi)$, 
let $\Collect(\pi, X)$ be the multiset $\bigcup \{\Perm(\pi(x)) \mid x \in X\}$.
For any multiset $\Lambda$ over $\Value$, let $\IsSet(\Lambda)$ be true iff $\Lambda$ is a set.

Given $\rho_1, \rho_2 \in \StateExpr$, we write $\rho_1 \vee \rho_2$ for the union of $\rho_1$ and $\rho_2$, 
$\rho_1 \wedge \rho_2$ for the intersection of $\rho_1$ and $\rho_2$, and $\neg \rho_1$ for $\StateExpr \setminus \rho_1$.
Similarly, given $\alpha_1, \alpha_2 \in \StateExpr$, we write $\alpha_1 \vee \alpha_2$ for the union of $\alpha_1$ and $\alpha_2$, 
$\alpha_1 \wedge \alpha_2$ for the intersection of $\alpha_1$ and $\alpha_2$, and $\neg \alpha_1$ for $\TransExpr \setminus \alpha_1$.
We denote the relational composition of $\alpha_1$ and $\alpha_2$ by $\alpha_1 \circ \alpha_2$.
Given $\rho \in \StateExpr$ and $\alpha \in \TransExpr$, we denote by $\ga{\locExpr}{\alpha}$ the
set $\{(\sigma, \sigma') \mid \sigma \in \locExpr \Rightarrow (\sigma,\sigma') \in \alpha\}$.
We let $\rho \circ \alpha$ denote the relation obtained by restricting the domain of $\alpha$ to $\rho$.
Similarly, we let $\alpha \circ \rho$ denote the relation obtained by restricting the co-domain of $\alpha$ to $\rho$.

%%% Local Variables: 
%%% mode: latex
%%% TeX-master: "paper"
%%% End: 


