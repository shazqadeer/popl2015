\section{Modules}
\label{sec:modules}

Section~\ref{sec:examples} used many small examples to illustrate various verification techniques supported by \civl.
Nevertheless, it's also important to show that \civl's features can scale up to larger programs in a modular way:
we should be able to check a large program by breaking it into smaller pieces and checking the pieces independently.
Many languages built on Boogie, including Dafny~\cite{Dafny10} and BoogieX86~\cite{IronClad14}, encode modules in Boogie,
so it's important that concurrent extensions to Boogie continue to support modular programming.
This section describes a simple module system built on \civl that allows separate verification of modules,
allowing programmers to make changes to the private implementation of one module
without disturbing the verification of other modules.
Although the focus is on separate verification,
this module system can easily be extended with other traditional module features,
such as abstraction of types and hiding of definitions.

A key challenge for modular verification in \civl is the $\CommutativitySafe$ and $\InterferenceSafe$ judgments.
Section~\ref{sec:verification} defines these as whole-program judgments,
quadratically checking all pairs of actions or all pairs of yields and atomic blocks from an entire program.
To achieve separate verification for \civl,
we must be able to check these judgments on a per-module basis.
To achieve this,
we observe that $\CommutativitySafe$ and $\InterferenceSafe$ are trivially true for operations that act on disjoint sets of global variables.
If an atomic block modifies only variables $g_1$ and $g_2$, it will not interfere with a yield that refers only to variables $g_3$ and $g_4$.
More generally, let each module $M$ own a set of global variables, such that each global variable is owned by exactly one module,
and decree that only $M$'s procedures and actions can access $M$'s global variables.
Formally, define ownership for global variables, procedures and actions as:

\hspace{10mm}\begin{tabular}{rclcl}
$M$ & $\in$ & $\Module$ \\
$\own_{\Global}$ & $\in$ & $\Global \rightarrow \Module$ \\
$\own_{\ProcName}$ & $\in$ & $\ProcName \rightarrow \Module$ \\
$\own_{\ActionName}$ & $\in$ & $\ActionName \rightarrow \Module$ \\
\end{tabular}

\noindent
If $\own_{\ProcName}(P) = M$, then $P$'s preconditions, postconditions, and statements only refer to global $g$ where $\own_{\Global}(g) = M$.
Similarly, if $\own_{\ProcName}(A) = M$, then $A$'s gate and transition relation can depend only on $g$ where $\own_{\Global}(g) = M$.
With this restriction, $\CommutativitySafe$ and $\InterferenceSafe$ can be checked for each module $M$ in isolation;
no other module $M'$ could interfere with any of the global variables owned by $M$.

As a more concrete example, suppose modules \verb`Lock1` and \verb`Lock2` independently implement locks,
expressed as atomic actions on variables \verb`lock1` and \verb`lock2`:

\begin{verbatim}
module Lock1
  var lock1:Tid;
  procedure Acq1(linear tid:Tid)
    right [assume lock1 == none;
           lock1 := tid;]
  {...}
  procedure Rel1()
    left [lock1 := none;]
  {...}
  ...

module Lock2
  var lock2:Tid;
  procedure Acq2(linear tid:Tid)
    right [assume lock2 == none;
           lock2 := tid;]
  {...}
  procedure Rel2()
    left [lock2 := none;]
  {...}
  ...
\end{verbatim}

By defining $\own_{\Global}(\mathtt{lock1}) = \mathtt{Lock1}$ and $\own_{\Global}(\mathtt{lock2}) = \mathtt{Lock2}$,
the \verb`Lock1` and \verb`Lock2` modules can be verified independently.

Nevertheless, a fixed ownership assignment is too inflexible to allow effective sharing of global state between modules.
Therefore, it's also important to notice that $\own_{\Global}$, $\own_{\ProcName}$, and $\own_{\ActionName}$ can change across refinement layers.
For example, suppose that in the first (lowest) layer,
we verify \verb`Lock1` and \verb`Lock2`,
replacing the \verb`Acq1`, \verb`Acq2`, \verb`Rel1`, and \verb`Rel2` procedures with \verb`Acq1`, \verb`Acq2`, \verb`Rel1`, and \verb`Rel2` actions,
and hiding the implementations of the bodies of these procedures.
Then suppose that in the second layer,
we want to use \verb`Lock1` and \verb`Lock2` in a module \verb`Counter`:

\begin{verbatim}
module Counter
  var x1:int, x2:int;
  ...
  procedure Inc1(linear tid:Tid)
    returns(n1:int)
    both [assert lock1 == tid;
          n1 := x1; x1 := x1 + 1;]

  procedure Inc2(linear tid:Tid)
    returns(n2:int)
    both [assert lock2 == tid;
          n2 := x2; x2 := x2 + 1;]

  procedure IncCnt(linear tid:Tid)
    atomic [...]
  { call Acq1(tid);
    call _ := Inc1(tid);
    call Rel1(); }

  procedure DecCnt(linear tid:Tid)
    atomic [...]
  { call Acq2(tid);
    call _ := Inc2(tid);
    call Rel2(); }

  procedure ReadCnt(linear tid:Tid)
    returns(n:int)
    atomic [...]
  { call Acq1(tid);
    call Acq2(tid);
    var n1 := Inc1(tid);
    var n2 := Inc2(tid);
    n := n1 - n2;
    call Rel2();
    call Rel1(); }
\end{verbatim}

\verb`Counter` uses \verb`Lock1` and \verb`Lock2` to implement an atomic counter supporting increment and decrement,
using only primitive increment operations \verb`Inc1` and \verb`Inc2` to access its variables \verb`x1` and \verb`x2`.
Since \verb`Inc1` and \verb`Inc2` refer to \verb`lock1` and \verb`lock2`,
the owner of \verb`Inc1` and \verb`Inc2` must also own \verb`lock1`, and \verb`lock2`.
Fortunately, ownership is free to migrate between layers,
so in the second layer we can define
$\own_{\Global}(\mathtt{lock1}) = \mathtt{Counter}$ and $\own_{\ActionName}(\mathtt{Acq1}) = \mathtt{Counter}$ and so on.
(Note that $\own_{\ProcName}(\mathtt{Acq1})$ doesn't matter in the second layer,
since the procedure \verb`Acq1` was hidden in the first layer.)

After verifying \verb`Counter` in the second layer,
we can hide the \verb`lock1` and \verb`lock2` variables entirely,
along with the \verb`Inc1`, \verb`Inc2`, \verb`Acq1`, \verb`Acq2`, \verb`Rel1`, and \verb`Rel2` actions
and the \verb`IncCnt`, \verb`DecCnt`, and \verb`ReadCnt` procedures.
This leaves just the \verb`x1` and \verb`x2` variables
and the \verb`IncCnt`, \verb`DecCnt`, and \verb`ReadCnt` atomic actions.

We can then repeat this process, transferring ownership of
\verb`x1`, \verb`x2`, \verb`IncCnt`, \verb`DecCnt`, and \verb`ReadCnt`
to another module in a third layer of refinement/abstraction:

\begin{verbatim}
module Client
  ..call IncCnt..call DecCnt..call ReadCnt..
\end{verbatim}

\subsection{Functors}

For the sake of explicitness, the example above used two replicas of a lock module
(\verb`Lock1` and \verb`Lock2`),
and verified each replica separately.
In practice, we only want to want to write and verify such modules once.
Borrowing from Standard ML~\cite{StandardML},
we can introduce a notion of {\it functors} that generate modules:

\begin{verbatim}
functor LockFunctor()
  var lock:Tid;
  procedure Acq(linear tid:Tid) right [...]
  procedure Rel() left [...]
  ...

module Lock1 := LockFunctor();
module Lock2 := LockFunctor();
\end{verbatim}

Following Standard ML's approach, the \verb`Lock1` and \verb`Lock2` modules each get their own copies of the \verb`lock` variable (named \verb`Lock1.lock` and \verb`Lock2.lock`) and their own copies of \verb`Acq` and \verb`Rel`.
Nevertheless, we only have to verify \verb`LockFunctor` once, rather than verifying each instantiation of \verb`LockFunctor` separately
(similar to Standard ML, which only has to type-check a functor once, rather than type-checking each functor instantiation separately).

In some situations, we might want to create an unbounded number of locks at run-time;
static instantiation of a bounded number of modules doesn't suffice for this.
Instead, without adding any new features to \civl, we can simply change the \verb`lock` variable to represent an array of locks (as an array of \verb`Tid`s),
and implement acquire/release operations on elements of the array.
Note that the functor approach and array approach are complementary and can be combined.
The functor approach has the advantage of creating multiple independent \verb`lock` variables,
whose ownership can be independently transferred to other modules.
This allows, say, module $M_1$ to own one \verb`lock` variable replica,
while $M_2$ owns a different \verb`lock` variable replica,
so that $M_1$'s \verb`lock` variable cannot interfere with $M_2$'s \verb`lock` variable.
The array approach, on the other hand,
is useful for supporting dynamically allocated locks within a single module.

\subsection{Module invariants}

Modules often need to maintain invariants about their variables.
For example, the following module maintains an invariant \verb`x >= 0`
that is true after initialization:

\begin{verbatim}
module X
  var x:int;
  procedure Init()
    ensures x >= 0;
  { x := 0; }

  procedure Inc()
    requires x >= 0;
    ensures  x >= 0;
  { x := x + 1; }

  procedure Yield()
    ensures  old(x) >= 0 ==> x >= 0;
  { yield old(x) >= 0 ==> x >= 0; }
\end{verbatim}

\noindent
Other modules may need to maintain \verb`X`'s invariant to call \verb`X`'s procedures.
However, if \verb`X` owns \verb`x`, mentioning \verb`x` from another module is prohibited:

\begin{verbatim}
module Y
  procedure P()
  {
    call Init();
    ...
    yield x >= 0; // not allowed
    ...
    call Inc();
  }
\end{verbatim}

\noindent
Instead, following the approach from Figure~\ref{fig:ex3},
the other module can call a procedure in module \verb`X` that yields,
since that procedure can refer to \verb`x`:

\begin{verbatim}
module Y
  procedure P()
  {
    call Init();
    ...
    call Yield(); // allowed
    ...
    call Inc();
  }
\end{verbatim}

\noindent
Intuitively, module \verb`X` is responsible for declaring its invariants in \verb`Yield`,
and these invariants are checked when verifying \verb`X`.
Other modules can then maintain these invariants indirectly through \verb`Yield`.

To maintain invariants from multiple modules (say, \verb`X1` and \verb`X2`),
we can allow parallel calls to multiple yield procedures simultaneously:

\begin{verbatim}
module X1 ...
module X2 ...

module Y
  procedure Yield12()
    ensures old(x1) >= 0 ==> x1 >= 0;
    ensures old(x2) >= 0 ==> x2 >= 0;
  {
    // parallel call:
    call Yield1() | Yield2();
  }
\end{verbatim}

\noindent
(Before a parallel call, the preconditions of all the calls must hold.
After the parallel call, the postconditions of the all calls hold.)

