\documentclass[twocolumn]{article}
\usepackage[latin1]{inputenc}
\usepackage{amsmath}
\usepackage{xspace}
\usepackage{amsfonts}
\usepackage{amssymb}
\usepackage{latexsym}
\usepackage{graphicx}
\usepackage[latin1]{inputenc}
\usepackage{amsmath,amsfonts,amssymb}
\usepackage[noend]{algpseudocode}
\usepackage{url}
\usepackage{amsmath,amsthm,amssymb}
\usepackage{amsmath,amssymb}
\usepackage{color}
\usepackage{fancyvrb}
\usepackage{mathpartir}

\usepackage{caption}
\usepackage{subcaption}

\usepackage{tikz}
\usetikzlibrary{arrows,automata}
\usetikzlibrary{shapes.symbols}
\usetikzlibrary{shapes}

\newtheorem{theorem}{Theorem}
\newtheorem{lemma}{Lemma}

\newcommand{\calvin}{{\sc calvin}\xspace}
\newcommand{\QED}{{\sc qed}\xspace}
\newcommand{\civl}{{\sc civl}\xspace}
\newcommand{\boogie}{{\sc boogie}\xspace}
\newcommand{\zthree}{{\sc Z3}\xspace}
\newcommand{\casm}{{\sc casm}\xspace}
\newcommand{\why}{{\sc why}\xspace}

\renewcommand{\floatpagefraction}{0.75}

\makeatletter
\let\@copyrightspace\relax
\makeatother

\begin{document}

\title{Automated and modular refinement reasoning for concurrent programs}
\author{Chris Hawblitzel \\ Microsoft \and Erez Petrank \\ Technion \and Shaz Qadeer \\ Microsoft \and Serdar Tasiran \\ Ko\c{c} University}
\date{}

\maketitle


\newcommand{\Type}{\mathit{Type}}
\newcommand{\VarName}{\mathit{VarName}}
\newcommand{\Var}{\mathit{Var}}
\newcommand{\Value}{\mathit{Value}}
\newcommand{\Expr}{\mathit{Expr}}
\newcommand{\StateExpr}{\mathit{StateExpr}}
\newcommand{\TransExpr}{\mathit{TransExpr}}
\newcommand{\LocalStateExpr}{\mathit{LocalStateExpr}}
\newcommand{\Spec}{\mathit{Spec}}
\newcommand{\AtomicSpecName}{\mathit{AtomSpecName}}
\newcommand{\AtomicAction}{\mathit{AtomAction}}
\newcommand{\Entry}{\mathit{Entry}}
\newcommand{\Program}{\mathit{Program}}
\newcommand{\Prog}{\mathit{Prog}}
\newcommand{\ProgSpec}{\mathit{ProgSpec}}
\newcommand{\ProcName}{\mathit{ProcName}}
\newcommand{\ActionName}{\mathit{ActionName}}
\newcommand{\Proc}{\mathit{Proc}}
\newcommand{\proc}{\mathit{Pr}}
\newcommand{\Vinfo}{\mathit{Vinfo}}
\newcommand{\Realz}{\mathit{Real}}
\newcommand{\realz}{\mathit{rl}}
\newcommand{\Vstore}{\mathit{Vstore}}
\newcommand{\Vinfos}{\overrightarrow{\mathit{Vinfo}}}
\newcommand{\Blocks}{\mathit{Blocks}}
\newcommand{\Block}{\mathit{Block}}
\newcommand{\BlockHeader}{\mathit{BlockHdr}}
\newcommand{\blockheader}{\mathit{bh}}
\newcommand{\Thread}{\mathit{Thread}}
\newcommand{\Label}{\mathit{Label}}
\newcommand{\Body}{\mathit{Body}}
\newcommand{\ProcSpec}{\mathit{ProcSp}}
\newcommand{\ProcRefSpec}{\mathit{RefSp}}
\newcommand{\Stmt}{\mathit{Stmt}}
\newcommand{\stm}{\mathit{stm}}
\newcommand{\coll}{\mathit{coll}}
\newcommand{\stmt}{\mathit{s}}
\newcommand{\ProcBody}{\mathit{ProcBody}}
\newcommand{\locExpr}{\mathit{le}}

\newcommand{\yieldorborder}[1]{\mathit{yield\_or\_border}\ #1}
\newcommand{\ite}[3]{{\mathit{if}\ {#1}\ }{\mathit{then}\ {#2}\ }{\mathit{else}\ {#3}\ }}
\newcommand{\while}[3]{\mathit{while}\ \{{#1}\}\  {{#2}\ } \mathit{do}\ {{#3}\ } }
\newcommand{\twhile}[4]{\mathit{while}\ \{{#1}\}\ \{{#2}\}\ {{#3}\ } \mathit{do}\ {{#4}\ } }
\newcommand{\yield}[1]{\mathit{yield}\ #1}
\newcommand{\join}[1]{\mathit{join}\ #1}
\newcommand{\async}[1]{\mathit{async}\ #1}
\newcommand{\call}[1]{\mathit{call}\ #1}
\newcommand{\noyield}[1]{\mathit{noyield}\ {#1}\ }
\newcommand{\ablock}[2]{\mathit{ablock}\ {\{#1\}}\ {#2}\ }
\newcommand{\parcall}[2]{\mathit{call}\ {#1} || {#2}\ }
\newcommand{\border}[1]{\mathit{border}\ #1}
\newcommand{\assert}[1]{\mathit{assert}\ #1}
\newcommand{\assume}[1]{\mathit{assume}\ #1}
\newcommand{\distinguish}[1]{\mathit{distinguish}\ #1}
\newcommand{\distinguishother}[1]{\overline{\mathit{distinguish}}\ #1}
\newcommand{\havoc}[1]{\mathit{havoc}\ #1}
\newcommand{\goto}[1]{\mathit{goto}\ #1}
\newcommand{\fork}[1]{\mathit{par}\ #1}
\newcommand{\impl}[3]{ \{ {#1} \}  {#2} \preceq {#3} }
\newcommand{\skipstmt}{\mathit{skip}}
\newcommand{\transfer}[2]{#2 \leftarrow #1}
\newcommand{\block}{b}
\newcommand{\hiddenVars}{\vec{h}}
\newcommand{\bodies}{\mathit{bs}}
\newcommand{\blocks}{\mathit{bls}}
\newcommand{\specs}{\mathit{ps}}
\newcommand{\procSpecs}{\mathit{prSp}}
\newcommand{\procRefSpecs}{\mathit{refSp}}
\newcommand{\ControlFlow}{\mathit{Control}}
\newcommand{\cf}{\mathit{cf}}
\newcommand{\state}{\sigma}
\newcommand{\unchangedExcept}[1]{\mathit{UnchangedExcept}_{ {#1} }}


\newcommand{\varsG}{G}
\newcommand{\varsTL}{\mathit{TL}}
\newcommand{\varsL}{L}
\newcommand{\true}{\mathit{true}}
\newcommand{\false}{\mathit{false}}
\newcommand{\varsGTLL}{\varsG\cup\varsTL\cup\varsL}
\newcommand{\vars}{\sigma}
\newcommand{\PS}{\vec{\mathit{P}}}
\newcommand{\TS}{\vec{\mathit{T}}}
\newcommand{\LS}{\vec{\mathit{L}}}
\newcommand{\yielding}[1]{\mathit{yielding}(#1)}
\newcommand{\trans}{\longrightarrow}
\newcommand{\transmany}{\trans^{*}}
\newcommand{\xs}{\vec{\mathit{x}}}
\newcommand{\ws}{\vec{\mathit{w}}}
\newcommand{\ys}{\vec{\mathit{y}}}
\newcommand{\vs}{\vec{\mathit{v}}}
\newcommand{\ts}{\vec{\mathit{t}}}
\newcommand{\stmts}{\vec{\mathit{s}}}
\newcommand{\Collect}{\mathit{Collect}}
\newcommand{\Inv}{\mathit{Inv}}
\newcommand{\anAction}{{\eta}}
\newcommand{\emptyAction}{\epsilon}

\newcommand{\SeqProcStmt}{\mathit{PS}}
\newcommand{\SeqVer}{{\mathit{SeqVer}\ }}
\newcommand{\Visit}{{\mathit{Visit}}}
\newcommand{\VarTypes}{\mathit{VarTypes}}
\newcommand{\BlockTypes}{\mathit{BlockTypes}}
\newcommand{\YieldLabels}{\mathit{YieldLabels}}
\newcommand{\LinearVars}{\mathit{LinearVars}}
\newcommand{\JoinLabels}{\mathit{JoinLabels}}
\newcommand{\vartypes}{\Gamma}
\newcommand{\blocktypes}{\Delta}
\newcommand{\yieldlabels}{\Psi}
\newcommand{\lins}{\lambda}
\newcommand{\linss}{\vec{\lambda}}
\newcommand{\dom}{\mathit{dom}}
\newcommand{\Globals}{\mathit{Globals}}
\newcommand{\Locals}{\mathit{Locals}}
\newcommand{\ThreadLocals}{\mathit{ThreadLocals}}

\newcommand{\ProcessProgram}{\mathit{ProcessProgram}}
\newcommand{\ProcessModule}{\mathit{ProcessModule}}

\newcommand{\InterfereStatic}{\mathit{InterfereStatic}}
\newcommand{\InterfereDynamic}{\mathit{InterfereDynamic}}
\newcommand{\Interfere}{\mathit{Interfere}}
\newcommand{\Yields}{\mathit{Yields}}
\newcommand{\Ablocks}{\mathit{Ablocks}}
\newcommand{\ProcessBlocks}{\mathit{ProcessBlocks}\xspace}
\newcommand{\ProcessBlock}{\mathit{ProcessBlock}\xspace}
\newcommand{\ProcessStmt}{\mathit{ProcessStmt}\xspace}
\newcommand{\ProcessHeaderBlock}{\mathit{ProcessHeaderBlock}\xspace}
\newcommand{\requires}{\mathit{requires}}
\newcommand{\availLocals}{\vec{\mathit{a}}}
\newcommand{\ea}{\mathit{e}_a}
\newcommand{\ey}{\mathit{e}_y}
\newcommand{\ew}{\mathit{e}_w}
\newcommand{\xsw}{\vec{\mathit{x}}_w}
\newcommand{\xsr}{\vec{\mathit{x}}_r}
\newcommand{\dummyLocal}{l_d}
\newcommand{\Add}{\mathit{Add}}

\newcommand{\modulename}{\mathit{M}}
\newcommand{\modulenames}{\vec{\mathit{M}}}
\newcommand{\ModuleName}{\mathit{ModuleName}}
\newcommand{\ModuleHeader}{\mathit{ModuleHdr}}
\newcommand{\ModuleHeaders}{\mathit{ModuleHdrs}}
\newcommand{\modulelabels}{\mathit{mls}}
\newcommand{\modulepreconditions}{\mathit{mps}}
\newcommand{\moduleheaders}{\mathit{mhs}}
\newcommand{\moduleheader}{\mathit{mh}}

\newcommand{\Good}{\mathit{Good}}
\newcommand{\GoodModule}{\mathit{GoodModule}}
\newcommand{\GoodModules}{\mathit{GoodModules}}
\newcommand{\NeverFail}{\mathit{NeverFail}}

\newcommand{\YieldingThread}{\mathit{YT}}

\newcommand{\YieldingThreads}{\overrightarrow{\YieldingThread}}

\newcommand{\StmtStack}{\mathit{SS}}

\newcommand{\StmtCtxt}{\mathit{SC}}
\newcommand{\StmtStackCtxt}{\mathit{SSC}}
\newcommand{\ThreadCtxt}{\mathit{TC}}
\newcommand{\ProgCtxt}{\mathit{PC}}

\newcommand{\fails}{\ \mathrm{fails}}


\newcommand{\tl}{\mathit{tl}}
\newcommand{\tls}{\mathit{tls}}
\newcommand{\Frame}[2]{(#1,#2)}

\newcommand{\pre}{\mathit{pre}}
\newcommand{\post}{\mathit{post}}
\newcommand{\modvars}{\mathit{mod}}
\newcommand{\mov}{\mathit{mov}}
\newcommand{\act}{\mathit{act}}
\newcommand{\Continuation}{C}
\newcommand{\YieldingContinuation}{\mathit{YC}}

\newcommand{\jp}{\vdash_p}
\newcommand{\jr}{\vdash_r}

\newcommand{\FH}[3]{\{#1\} #2 \{#3\}}
\newcommand{\re}{\mathit{re}}
\newcommand{\RM}{\mathit{RM}}
\newcommand{\CM}{\mathit{CM}}
\newcommand{\LM}{\mathit{LM}}

\newcommand{\pf}{\rightharpoonup}

\newcommand{\actions}{\mathit{as}}
\newcommand{\refines}{\mathit{rs}}

\newcommand{\Mover}{\mathit{Mover}}

\newcommand{\accessVars}{\mathit{Vars}}


\begin{abstract}
We present \civl, a language and verifier for concurrent programs based on automated and modular refinement reasoning.  
\civl supports reasoning about a concurrent program at many levels of abstraction. 
Atomic actions in a high-level description are refined to fine-grain and optimized lower-level implementations. 
Modular specifications and proof annotations, such as location invariants and procedure pre- and post-conditions, 
are specified separately, independently at each level in terms of the variables visible at that level. 
We have implemented \civl as an extension to the \boogie language and verifier.
We have used \civl to refine a realistic concurrent garbage collection algorithm
from a simple high-level specification down to a highly-concurrent implementation described in terms of individual memory accesses.
\end{abstract}

\section{Introduction}
\label{sec:introduction}

What is refinement?  What distinguishes refinement from other approaches to verification? 
The big functional correctness proofs are all refinement proofs, e.g., Sel4 verification 
assuming no concurrency inside the kernel. or the works in hardware verification.  sel4 verification in the first paragraph and hardware verification in the second paragraph.

Hardware, although concurrent, is different from concurrent software.  
Control structures are different.  Hence refinement verification techniques in hardware don't work.
TLA+ drops down to logic but proofs are tedious, do not scale because of lack of software structuring.

Hallmarks of our refinement approach:
- Annotations respect the most important software abstraction of procedures; atomic actions are refined by procedures.
We develop techniques to manipulate multiple representations of a single program just as classical program verifiers manipulate programs and annotations such as types.


- Refinement supported by hiding to eliminate details due to data and reduction to eliminate details due to control.  

- Explicit non-interference reasoning based on assertions and a linear type system for encoding permissions disjoint memory, permissions, etc. logically.


This combination of techniques helps manage the complexity of our verification of a realistic GC algorithm.  
In particular, although our algorithm is based on an earlier algorithm by Dijkstra et al~\cite{dijk78}, 
it extends the earlier algorithm with various modern optimizations and embellishments to improve generality and performance.  
These extensions include lower write barrier overhead, phase-based synchronization and handshaking, 
and coordination between the GC and mutator threads during root scanning; our use of linearity aids the proof of root scanning, 
while our rely-guarantee encoding aids management of colors inside the write barrier.  
Furthermore, our encoding of the algorithm in \civl spans a wide range of abstraction, 
from low-level memory operations all the way up to high-level specifications; 
we used six levels of refinement to help hide low-level details from the high-level portions of the verification.
We believe that \civl's combination of features makes practical, for the first time, verification across such a wide range of abstraction.


\section{Overview}
\label{sec:overview}

\begin{figure} 
\begin{verbatim}
var Color: int; // WHITE=1, GRAY=2, BLACK=3
procedure WB(linear tid:Tid)
atomic [if (Color == WHITE) Color := GRAY];
{
  var cNoLock:int;
  cNoLock := GetColorNoLock(tid);
  yield Color >= cNoLock;
  if (cNoLock == WHITE) 
    call WBSlow(tid);
}
procedure WBSlow(linear tid:Tid)
atomic [if (Color == WHITE) Color := GRAY];
{
  var cLock:int;
  call AcquireLock(tid);
  cLock := GetColorLocked(tid);
  if (cLock == WHITE) 
    call SetColorLocked(tid, GRAY);
  call ReleaseLock(tid);
}
procedure GetColorNoLock(linear tid:Tid) 
  returns (cl:int) atomic [...];
procedure AcquireLock(linear tid:Tid) 
  right [...];
procedure ReleaseLock(linear tid:Tid) 
  left [...];
procedure GetColorLocked(linear tid:Tid) 
  returns (cl:int) both [...];
procedure SetColorLocked(linear tid:Tid, 
  cl: int) atomic [...];
\end{verbatim}
\caption{Write barrier}
\label{fig:reft}
\end{figure}

\begin{figure}
\begin{center}
\includegraphics[scale=0.35]{WBSlow.pdf}
\end{center}
\caption{Abstraction of \exC{WBSlow}}
\label{fig:midwb}
\end{figure}

\begin{figure}
\begin{center}
\includegraphics[scale=0.35]{YieldTypeCheckingAutomaton.pdf}
\end{center}
\caption{Yield sufficiency automaton ($YSA$)}
\label{fig:ysa}
\end{figure}

We present an overview of our approach to refinement on the program in Figure~\ref{fig:reft},
a simplified version of the write barrier in a concurrent garbage collector (GC).
In a concurrent GC, a color (either \exC{WHITE}, \exC{GRAY}, or \exC{BLACK})
is associated with each object on the heap.  
Before writing to an address \exC{addr}, a mutator thread checks 
\exC{addr} has color \exC{WHITE}
and sets it to \exC{GRAY}, indicating that the object at \exC{addr}
and objects reachable from it should not be garbage collected. 

Procedure \exC{WB} implements the write barrier.
To simplify exposition, 
we consider a single object whose color is stored in the shared variable \exC{Color}.
Acquiring and releasing a loc for each address encountered by a
mutator slows applications down. To avoid this, 
\exC{WB} reads \exC{Color} without holding a lock.
If the color is {\tt WHITE}, it calls the more expensive procedure \exC{WBSlow} 
that re-examines and possibly updates \exC{Color} while holding the lock.
\civl simplifies reasoning about \exC{WB} and \exC{WBSlow} by allowing us to 
compactly express their specification as the following atomic action:
\begin{verbatim}
[if (Color == WHITE) Color := GRAY]
\end{verbatim}
This specification indicates that regardless of the different implementations of 
\exC{WB} and \exC{WBSlow} and regardless of how the environment interferes
with their execution, to their respective callers it appears as if they atomically execute the above code.

{\bf Per-procedure simulation, non-interference via invariants.}
The correctness of \exC{WB} is not obvious, and its verification
requires a combination of techniques as discussed next. 
Consider the following potential scenario. 
\exC{WB}, not holding a lock, reads {\tt Color} and
sets \exC{cNoLock} to \exC{GRAY} and then yields. Another thread sets {\tt Color} to
{\tt WHITE}. \exC{WB} resumes, but does nothing and exits,
because the procedure-local variable \exC{cNoLock} is \exC{GRAY}. In this scenario, the atomic action
specification of \exC{WB} would not be satisfied. However, in the GC this
scenario is not possible. 
The yield predicate (location invariant) expresses the fact that
other threads in the environment of the thread running \exC{WB} can
only modify {\tt Color} to a higher (darker) value -- \civl verifies the correctness
of this location invariant.
Using this location invariant, \civl then verifies atomicity
refinement for {\tt WB} by verifying the existence of a particular simulation-relation:
for every control path through \exC{WB}, exactly one {\tt yield}-to-{\tt yield} execution
fragment is simulated by the atomic action specification and other fragments do not modify
global state. 
This proof requires both correct modeling of environment interference in the yield predicate
and the atomic action specification for the called procedure \exC{WBSlow}.
The \civl verifier automatically computes a logical verification condition capturing
these proof obligations from the body and specification of \exC{WB}.
Just as the verification of \exC{WB} builds on the specification of \exC{WBSlow},
the verification of \exC{WBSlow} builds on other refinement proofs (not shown) 
of the procedures called in \exC{WBSlow};
these called procedures are shown at the bottom of the figure. 
While this example shows only one procedure at this level, in programs
with many procedures with atomic specifications at each level,
\civl combines the per-procedure refinement proofs soundly into a
whole-program refinement proof. 

{\bf Reduction, preemptive semantics vs collaborative semantics.}
The verification of \exC{WBSlow} highlights reduction-related features
in \civl. 
As explained above, the refinement checking is performed on cooperative semantics in which a 
{\tt yield}-to-{\tt yield} execution fragment of code is executed atomically.
However, in a real execution, control can switch between threads at any point in the code. 
A naive modeling of a real execution would put a yield statement before every instruction in the code.
The absence of a yield statement before every instruction is justified by reasoning about mover types~\cite{FlanaganFLQ08}. 
The procedures called in \exC{WBSlow} have the mover types claimed in their
declarations and verified by \civl. 
For example, the mover type of \exC{Acquirelock} is {\tt right} which indicates 
that it commutes later in time against concurrently executing environment actions.
% \exC{ReleaseLock} has mover type {\tt left} and it commutes earlier in time;
% \exC{SetColorLocked} has mover type {\tt atomic} and it does not commute;
% \exC{GetColorLocked} has mover type {\tt both} and it commutes both earlier and later in time.
These mover types are checked by constructing verification conditions from each pair of atomic actions.

The use of movers is entirely optional in \civl, but very beneficial
in our experience. The programmer can choose to avoid mover and
commutativity reasoning, and simply
annotating atomic action specifications with the mover type {\tt
  atomic}.
However, without mover reasoning,  a yield statement and an
accompanying predicate
must be inserted before every invocation of an atomic action.
It has been demonstrated in work in the literature that mover
reasoning simplifies many proofs~\cite{ElmasQT09}. 
In our experience with \civl, we have observed that using more yield
predicates rather than mover reasoning can make proofs difficult in two ways.
First, the annotation burden goes up because sophisticated ghost variables may need to be introduced in the 
program semantics.\footnote{Well-scoped location invariants that cannot refer to the state of other threads are known to be incomplete, 
both in theory and in practice.}
Second, the computational cost of the pairwise mover reasoning is replaced by the cost of pairwise non-interference checks between yield predicates 
and concurrently executing atomic actions. 
\civl does not force the use of mover
reasoning but provides automation for this important verification
feature and its use in conjunction with other techniques illustrated in
this section. 

\civl verifies mover-based reduction, i.e., the correctness of the placement of {\tt yield}s using a novel  approach.
A {\em Yield
  Sufficiency Automaton\/} ($\YSA$ in Figure~\ref{fig:ysa}) encodes all sequences of atomic actions and yields for which safety of cooperative semantics is sufficient 
for safety of preemptive semantics. 
Roughly speaking, each ``transaction'' starts with a sequence of right movers (or both movers) and ends with a sequence of left movers (or both movers).
In the middle, it can have at most one non mover.
\civl then interprets the control-flow graphs of each procedure as an automaton with mover types as edge labels. 
This abstraction for \exC{WBSlow} is shown in Figure~\ref{fig:midwb}
\civl verifies that this automaton is simulated by the yield-sufficiency automaton using an existing algorithm for computing simulation relation~\cite{HenzingerHK95}.

{\bf Variable hiding.}
The atomic action specification of \exC{WBSlow} makes no reference to the lock variable, although its implementation involves a lock. 
When verifying refinement for \exC{WBSlow}, the lock variable has been hidden. 
\civl allows the programmer to both introduce and hide variables in
each refinement step, thereby providing the capability to perform data refinement.
The ability to introduce and hide variables and write yield predicates specific to each refinement step 
facilitates proofs spanning large abstraction gaps between the specification and implementation.

%%%%%%%%%% Shaz, we make this point in the intro and it breaks the
%%%%%%%%%% flow in this example.
%
% There are two important benefits of atomic action specifications.
% First, an atomic action is often the most compact and precise way to express the specification a procedure.
% Second, an atomic action specification successfully hides from the caller of a procedure $P$
% the potentially numerous yields inside the body of $P$.
% This capability provides modularity akin to that provided by the frame rule for sequential programs.
% Without this capability, any postcondition $\phi$ needed by the caller of $P$ must be made explicit in the precondition 
% and yield predicates of $P$ and all procedures recursively called inside it.
% With this capability, the caller achieves the same effect by writing $\yield{\phi}$ before and after the call, 
% provided that $\phi$ is preserved by the atomic action specification of $P$.


\begin{figure}
\setlength{\tabcolsep}{3pt}
{\bf Program Syntax} \\
\begin{tabular}{rclcl}
$g$ & $\in$ & $\Globals \subseteq \VarName$ \\
$\tl$ & $\in$ & $\ThreadLocals \subseteq \VarName$ \\
$l$ & $\in$ & $\Locals \subseteq \VarName$ \\
$x,y$ & $\in$ & $\Vars = \Globals \cup \ThreadLocals \cup \Locals$ \\
$v$ &  $\in$ & $\Value$ \\
$\sigma$ & $\in$ & $\Store = \Vars \rightarrow \Value$ \\
$\varsG$ & $\in$ & $StoreGlobals = \Globals \rightarrow \Value$ \\
$\varsTL$ & $\in$ & $StoreThreadLocals = \ThreadLocals  \rightarrow \Value$ \\
$\varsL$ & $\in$ & $StoreLocals = \Locals \rightarrow \Value$ \\
$e, \phi, \psi, \rho$ & $\in$ & $\StateExpr = 2^{\Store}$ \\
$\alpha, \beta$ & $\in$ & $\TransExpr = 2^{(\Store,\Store)}$ \\
$\locExpr$ & $\in$ & $\LocalStateExpr = 2^{\StoreLocals}$ \\
$P$ & $\in$ & $\ProcName$ \\
$A$ & $\in$ & $\ActionName$ \\
$m$ & $\in$ & $\Mover = \{B,R,L,N\}$\\
$\actions$ & $\in$ & $\ActionName \rightarrow (\StateExpr, \TransExpr, \Mover)$ \\
$\procs$ & $\in$ & $\ProcName \rightarrow (\StateExpr, 2^\ThreadLocals, \StateExpr, \Stmt)$ \\
$\refines$ & $\in$ & $\ProcName \pf \ActionName$ \\
$\lins$ & $\in$ & $\LinearVars = 2^{\Globals \cup \ThreadLocals}$ \\
$\ProcLins$ & $\in$ & $(\ActionName \cup \ProcName) \rightarrow (\LinearVars, \LinearVars)$ \\
$\ABlockAny$ & $\in$ & $\mathit{InsideABlock} ::= \ABlockInside \mid \ABlockOutside$ \\
$\RefinementAny$ & $\in$ & $\mathit{InsideRefinement} ::= \RefinementInside \mid \RefinementOutside$ \\
\end{tabular}
~\\
~\\
\begin{tabular}{rclcl}
$\stmt \in \Stmt$ &::= & $\skipstmt \mid \assert{\locExpr} \mid \yield{e,\lins} \mid$ \\
                  & & $\call{A} \mid \call{P} \mid \async{P} \mid $\\
                  & & $\ablock{e,\lins}{s} \mid s;\;s \mid$\\
                  & & $\ite{\locExpr}{s}{s} \mid$ \\
                  & & $\while{e,\alpha}{\locExpr}{s}$ \\ 
$\StmtStack \in \mathit{StmtStack}$ &::= & $\stmt \mid (\varsL,\StmtStack) \mid \StmtStack;\stmt$ \\
$T \in \mathit{Thread}$ &::= &$(\varsTL, (\varsL, \StmtStack))$ \\
$\Prog \in \Program$ &::= & $\ProgCtxt[\varsG][\varsTL][\varsL][\stmt]$ \\
~\\
$\StmtCtxt \in \mathit{StmtCtxt}$ &::= &$[]_{Stmt} \mid \StmtCtxt;\stmt$ \\
$\StmtStackCtxt \in \mathit{StmtStackCtxt}$ &::= & $([]_{\Locals}, \StmtCtxt) \mid (L,\StmtStackCtxt) \mid \StmtStackCtxt;\stmt$ \\
$\ThreadCtxt \in \mathit{ThreadCtxt}$ &::= &$([]_{\ThreadLocals}, ([]_{\Locals}, \StmtCtxt)) \mid$ \\
 & &$([]_{\ThreadLocals}, (\varsL, \StmtStackCtxt))$ \\
$\YieldingThread \in \mathit{YieldingThread}$ &::= &$\ThreadCtxt[\varsTL][\varsL][\yield{e,\lins}]$ \\
$\ProgCtxt \in \mathit{ProgCtxt}$ &::= & $(\procs, \actions, \ProcLins, []_{\Globals}, \YieldingThreads \cdot \ThreadCtxt \cdot \YieldingThreads)$ \\
\end{tabular}
~\\
~\\
\begin{tabular}{rcl}

\end{tabular}
\setlength{\tabcolsep}{6pt}
\caption{Syntax}
\label{fig:syntax}
\end{figure}

\begin{figure}
\scriptsize{
\medskip
%%%%%%%%%%%%%%%%%%%%
$
\inferrule
{
\ProgCtxt[\varsG][\varsTL][\varsL][s] = (\_, \actions, \_, \_, \_) \\
\actions \vdash (\MakeStore{\varsG}{\varsTL}{\varsL}, s) \trans (\MakeStore{\varsG'}{\varsTL'}{\varsL'}, s')
}
{\ProgCtxt[\varsG][\varsTL][\varsL][s] \trans \ProgCtxt[\varsG'][\varsTL'][\varsL'][s']}
\;(\textsc{Program-Step})
$
\medskip
%%%%%%%%%%%%%%%%%%%%
$
\inferrule
{
\ProgCtxt[\varsG][\varsTL][\varsL][s] = (\_, \actions, \_, \_, \_) \\
\actions \vdash (\MakeStore{\varsG}{\varsTL}{\varsL}, s) \fails
}
{\ProgCtxt[\varsG][\varsTL][\varsL][s]) \fails}
\;(\textsc{Program-Fail})
$
\medskip
%%%%%%%%%%%%%%%%%%%%
$
\inferrule
{
\procs(P) = (\phi, \mods, \psi,\stmt) \\
\ProcLins(P) = (\lins, \lins') \\
T' = (\varsTL, (\varsL, \yield{\phi,\lins};\stmt))
}
{
(\procs, \actions, \ProcLins, \varsG, \ThreadCtxt[\varsTL][\varsL][\async{P}])
\trans
(\procs, \actions, \ProcLins, \varsG, \ThreadCtxt[\varsTL][\varsL][\skipstmt] \cdot T')
}
\;(\textsc{Async})
$
\medskip
%%%%%%%%%%%%%%%%%%%%
$
\inferrule
{
\\
}
{(\procs, \actions, \ProcLins, \varsG, \YieldingThreads \cdot (\varsTL, (\varsL,\skipstmt)) \cdot \YieldingThreads') \trans (\procs, \actions, \ProcLins, \varsG, \YieldingThreads \cdot \YieldingThreads')}
\;(\textsc{Thread-End})
$
\medskip
%%%%%%%%%%%%%%%%%%%%
$
\inferrule
{
\procs(P) = (\phi, \mods, \psi,\stmt) \\
\ProcLins(P) = (\lins, \lins') \\
\StmtStack = \yield{\phi,\lins};\stmt;\yield{\psi,\lins'}
}
{\ProgCtxt[\varsG][\varsTL][\varsL][\call{P}] \trans \ProgCtxt[\varsG][\varsTL][\varsL][\Frame{L}{\StmtStack}]}
\;{(\textsc{Call})}
$
\medskip
%%%%%%%%%%%%%%%%%%%%
$
\inferrule
{
\\
}
{\ProgCtxt[\varsG][\varsTL][\varsL][\Frame{\varsL'}{\skipstmt}] \trans \ProgCtxt[\varsG][\varsTL][\varsL][\skipstmt]}
\;{(\textsc{Return})}
$
%%%%%%%%%%%%%%%%%%%%
}
\caption{Operational semantics for program}
\label{fig:operational-semantics1}
\end{figure}


\begin{figure}
\scriptsize{
\medskip
\medskip
%%%%%%%%%%%%%%%%%%%%
$
\inferrule
{
\vars \vdash \locExpr \rightarrow \true
}
{\actions \vdash (\vars, \assert{\locExpr}) \trans (\vars, \skipstmt)}
\;{(\textsc{Assert-True})}
$
\medskip
%%%%%%%%%%%%%%%%%%%%
$
\inferrule
{
\vars \vdash \locExpr \rightarrow \false
}
{\actions \vdash (\vars, \assert{\locExpr}) \fails}
\;{(\textsc{Assert-False})}
$
\medskip
~\\
%%%%%%%%%%%%%%%%%%%%
$
\inferrule
{
\actions(A) = (\rho, \alpha, m) \\
(\vars, \vars') \vdash \alpha \\
}
{
\actions \vdash (\vars, \call{A}) \trans (\vars',\skipstmt)
}
\;{(\textsc{Atomic})}
$
\medskip
%%%%%%%%%%%%%%%%%%%%
$
\inferrule
{
\\
}
{\actions \vdash (\vars, \yield{e,\lins}) \trans (\vars, \skipstmt)}
\;{(\textsc{Yield})}
$
\medskip
%%%%%%%%%%%%%%%%%%%%
$
\inferrule
{
}
{\actions \vdash (\vars, \ablock{e,\lins}{\stmt}) \trans (\vars, \stmt)}
\;{(\textsc{AtomicBlock})}
$
\medskip
%%%%%%%%%%%%%%%%%%%%
$
\inferrule
{
\\
}
{\actions \vdash (\vars, \skipstmt;\stmt) \trans (\vars, \stmt)}
\;{\;\;\;\;\;\;\;\;\;\;\;\;(\textsc{Seq})}
$
\medskip
%%%%%%%%%%%%%%%%%%%%
$
\inferrule
{
\vars \vdash \locExpr \rightarrow \mathit{true}
}
{\actions \vdash (\vars, \ite{\locExpr}{s_1}{s_2}) \trans (\vars, s_1)}
\;{(\textsc{If-True})}
$
\medskip
%%%%%%%%%%%%%%%%%%%%
$
\inferrule
{
\vars \vdash \locExpr \rightarrow \mathit{false}
}
{\actions \vdash (\vars, \ite{\locExpr}{s_1}{s_2}) \trans (\vars, s_2)}
\;{(\textsc{If-False})}
$
\medskip
%%%%%%%%%%%%%%%%%%%%
$
\inferrule
{
\vars \vdash \locExpr \rightarrow \mathit{false}
}
{\actions \vdash (\vars, \while{e,\alpha}{\locExpr}{s}) \trans (\vars, \skipstmt)}
\;{(\textsc{While-False})}
$
\medskip
%%%%%%%%%%%%%%%%%%%%
$
\inferrule
{
\vars \vdash \locExpr \rightarrow \mathit{true}
}
{\actions \vdash (\vars, \while{e,\alpha}{\locExpr}{s}) \trans (\vars, s;\while{e,\alpha}{\locExpr}{s})}
\;{(\textsc{While-True})}
$
%%%%%%%%%%%%%%%%%%%%
}
\caption{Operational semantics for statement}
\label{fig:operational-semantics2}
\end{figure}

%%% Local Variables: 
%%% mode: latex
%%% TeX-master: "paper"
%%% End: 



\subsection{Type checking}
\begin{figure}
\scriptsize{
\medskip
%%%%%%%%%%%%%%%%%%%%
$
\inferrule
{
}
{
\lins;\RefinementAny;\ABlockAny \vdash \skipstmt : \lins
}
\;(\textsc{Skip})
$
\medskip
%%%%%%%%%%%%%%%%%%%%
$
\inferrule
{
}
{
\lins;\RefinementAny;\ABlockOutside \vdash \assert{\locExpr} : \lins
}
\;(\textsc{Assert})
$
\medskip
%%%%%%%%%%%%%%%%%%%%
$
\inferrule
{
\lins_y \subseteq \lins
}
{
\lins;\RefinementAny;\ABlockOutside \vdash \yield{e,\lins_y} : \lins
}
\;(\textsc{Yield})
$
\medskip
%%%%%%%%%%%%%%%%%%%%
$
\inferrule
{
\ProcLins(A) = (\lins,\lins')
}
{
\lins;\RefinementAny;\ABlockInside \vdash \call{A} : \lins'
}
\;(\textsc{Atomic})
$
\medskip
%%%%%%%%%%%%%%%%%%%%
$
\inferrule
{
\ProcLins(P) = (\lins,\lins') \\
\RefinementAny = \RefinementInside \Longrightarrow P \in \dom(\refines)
}
{
\lins;\RefinementAny;\ABlockOutside \vdash \call{P} : \lins'
}
\;(\textsc{Proc})
$
\medskip
%%%%%%%%%%%%%%%%%%%%
$
\inferrule
{
\lins_G \subseteq \Globals \\
\lins \cup \lins_P \cup \lins_P' \subseteq \ThreadLocals \\
\ProcLins(P) = ((\lins_G,\lins_P),(\lins_G,\lins_P')) \\
\RefinementAny = \RefinementInside \Longrightarrow P \in \dom(\refines)
}
{
\lins_G,\lins,\lins_P;\RefinementAny;\ABlockOutside \vdash \async{P} : \lins_G,\lins
}
\;(\textsc{Async})
$
\medskip
%%%%%%%%%%%%%%%%%%%%
$
\inferrule
{
\lins;\RefinementAny;\ABlockInside \vdash \stmt : \lins' \\
\lins_a \subseteq \lins
}
{
\lins;\RefinementAny;\ABlockOutside \vdash \ablock{e,\lins_a}{\stmt} : \lins'
}
\;(\textsc{Ablock})
$
\medskip
%%%%%%%%%%%%%%%%%%%%
$
\inferrule
{
\lins;\RefinementAny;\ABlockAny \vdash \StmtStack : \lins'
}
{
\lins;\RefinementAny;\ABlockAny \vdash (\varsL,\StmtStack) : \lins'
}
\;(\textsc{StackFrame})
$
\medskip
%%%%%%%%%%%%%%%%%%%%
$
\inferrule
{
\lins;\RefinementAny;\ABlockAny \vdash \StmtStack : \lins' \\
\lins';\RefinementAny;\ABlockAny \vdash \stmt : \lins''
}
{
\lins;\RefinementAny;\ABlockAny \vdash \StmtStack;\stmt : \lins''
}
\;(\textsc{Seq})
$
\medskip
%%%%%%%%%%%%%%%%%%%%
$
\inferrule
{
\lins;\RefinementAny;\ABlockAny \vdash s_1 : \lins' \\
\lins;\RefinementAny;\ABlockAny \vdash s_2 : \lins'
}
{
\lins;\RefinementAny;\ABlockAny \vdash \ite{\locExpr}{s_1}{s_2} : \lins'
}
\;(\textsc{Ite})
$
\medskip
%%%%%%%%%%%%%%%%%%%%
$
\inferrule
{
\lins;\RefinementAny;\ABlockAny \vdash s : \lins
}
{
\lins;\RefinementAny;\ABlockAny \vdash \while{e}{\locExpr}{s} : \lins
}
\;(\textsc{While})
$
\medskip
%%%%%%%%%%%%%%%%%%%%
$
\inferrule
{
\ProcLins(P) = (\lins,\lins') \\
\RefinementAny = \RefinementInside \Longleftrightarrow P \in \dom(\refines) \\
\lins;\RefinementAny;\ABlockOutside \vdash \bodies(P) : \lins'
}
{
\vdash P
}
\;(\textsc{Procedure})
$
\medskip
%%%%%%%%%%%%%%%%%%%%
$
\inferrule
{
\ProcLins(A) = (\lins,\lins') \\
\lins \cap \Globals = \lins' \cap \Globals \\
\actions(A) = (\rho, \alpha, m) \\
m \in \{B, L\} \Longrightarrow \forall \sigma \in \rho. \exists \sigma'. (\sigma, \sigma') \in \alpha \\
A \in \mathit{range}(\refines) \Longrightarrow \accessVars(\rho) \subseteq \Locals \\
A \in \mathit{range}(\refines) \Longrightarrow \alpha = ((\exists \Locals,\Locals'.\alpha) \wedge \mathit{\Same(\Locals,\Locals')}) \\
\forall (\sigma,\sigma') \in \alpha.
  \biguplus\{lv(\sigma'(x)) \mid x \in \lins'\} \subseteq
  \biguplus\{lv(\sigma(x)) \mid x \in \lins\}
}
{
\vdash A
}
\;(\textsc{Action})
$
\medskip
%%%%%%%%%%%%%%%%%%%%
$
\inferrule
{
T = (\varsTL, (\varsL, \StmtStack)) \\
\lins;\RefinementOutside;\ABlockOutside \vdash \StmtStack : \lins'
}
{
\lins \vdash T
}
\;(\textsc{Thread})
$
\medskip
%%%%%%%%%%%%%%%%%%%%
$
\inferrule
{
\forall P \in \ProcName. \vdash P \\
\forall A \in \ActionName. \vdash A \\
\lins = \lins_G,\lins_1,\ldots,\lins_n  \\
\lins_G \subseteq \Globals \\
\forall 1 \le i \le n. (\lins_G,\lins_i \vdash T_i) \\
\forall 1 \le i \le n. (\lins_i \subseteq \ThreadLocals) \\
\forall 1 \le i \le n. (T_i = (\varsTL_i, \ldots)) \\
\biguplus\{lv(\varsG(x)) \mid x \in \lins_G\} \uplus
  \biguplus\{lv(\varsTL_i(x)) \mid 1 \le i \le n, x \in \lins_i\} \subseteq \linsmax
}
{
\vdash (\bodies, \actions, \specs, \refines, \ProcLins, \lins, \varsG, T_1 \ldots T_n)
}
\;(\textsc{Program})
$
\medskip
%%%%%%%%%%%%%%%%%%%%
}
\caption{Type checking rules}
\label{fig:type-checking}
\end{figure}

\subsection{Safety}
\begin{figure}
\scriptsize{
\medskip
%%%%%%%%%%%%%%%%%%%%
$
\inferrule
{
}
{\{\} \jp \FH{\phi}{\skipstmt}{\phi}}
\;(\textsc{Skip})
$
\medskip
%%%%%%%%%%%%%%%%%%%%
$
\inferrule
{
P \not \in \dom(\refines)
}
{\{\} \jp \FH{\phi}{\assert{\locExpr}}{\phi}}
\;(\textsc{Assert1})
$
\medskip
%%%%%%%%%%%%%%%%%%%%
$
\inferrule
{
}
{\{\} \jp \FH{\phi \wedge \locExpr}{\assert{\locExpr}}{\phi}}
\;(\textsc{Assert2})
$
\medskip
%%%%%%%%%%%%%%%%%%%%
$
\inferrule
{
}
{\{\} \jp \FH{e}{\yield{e,\lins}}{e}}
\;(\textsc{Yield})
$
\medskip
~\\
%%%%%%%%%%%%%%%%%%%%
$
\inferrule
{
\actions(A) = (\rho, \alpha, m) \\ 
\mods \subseteq \ThreadLocals \\
\alpha \Rightarrow \Havoc(\Globals \cup \mods \cup \Locals) \\
\phi_1 \Rightarrow \rho \\ 
\mathit{old}(\phi_1) \wedge \alpha \Rightarrow \phi_2 \\
}
{\mods \jp \FH{\phi_1}{\call{A}}{\phi_2}}
\;(\textsc{Atomic})
$
\medskip
%%%%%%%%%%%%%%%%%%%%
$
\inferrule
{
\specs(P) = (\phi, \mods, \psi) \\ P \not \in \dom(\refines)
}
{\mods \jp \FH{\phi}{\call{P}}{\psi}}
\;(\textsc{Proc1})
$
\medskip
%%%%%%%%%%%%%%%%%%%%
$
\inferrule
{
\specs(P) = (\phi, \mods, \psi) \\ P \in \dom(\refines) \\ \FH{\phi}{\call{\refines(P)}}{\psi}
}
{\mods \jp \FH{\phi}{\call{P}}{\psi}}
\;(\textsc{Proc2})
$
\medskip
%%%%%%%%%%%%%%%%%%%%
$
\inferrule
{
\specs(P) = (\phi, \mods, \psi)
}
{\{\} \jp \FH{\rho \wedge \phi}{\async{P}}{\rho}}
\;(\textsc{Async})
$
\medskip
~\\
%%%%%%%%%%%%%%%%%%%%
$
\inferrule
{
\mods \jp \FH{\phi_1 \wedge e}{s}{\phi_2}
}
{\mods \jp \FH{\phi_1 \wedge e}{\ablock{e,\lins}{s}}{\phi_2}}
\;(\textsc{Ablock})
$
\medskip
%%%%%%%%%%%%%%%%%%%%
$
\inferrule
{
\mods \jp \FH{\phi'}{\StmtStack}{\psi} \\ 
(\accessVars(\phi) \cup \accessVars(\psi)) \cap \Locals = \{\} \\
\forall G, \varsTL.\ \phi(G,\varsTL) \Rightarrow \phi'(G, \varsTL,\varsL)
}
{
\mods \jp \FH{\phi}{(\varsL,\StmtStack)}{\psi}
}
\;(\textsc{StackFrame})
$
\medskip
%%%%%%%%%%%%%%%%%%%%
$
\inferrule
{
\mods_1 \jp \FH{\phi_1}{\StmtStack}{\phi_2} \\ \mods_2 \jp \FH{\phi_2}{s}{\phi_3}
}
{\mods_1 \cup \mods_2 \jp \FH{\phi_1}{\StmtStack;s}{\phi_3}}
\;(\textsc{Seq})
$
\medskip
%%%%%%%%%%%%%%%%%%%%
$
\inferrule
{
\mods_1 \jp \FH{e \wedge \phi_1}{s_1}{\phi_2} \\ \mods_2 \jp \FH{\neg e \wedge \phi_1}{s_2}{\phi_2}
}
{\mods_1 \cup \mods_2 \jp \FH{\phi_1}{\ite{\locExpr}{s_1}{s_2}}{\phi_2}}
\;(\textsc{Ite})
$
\medskip
%%%%%%%%%%%%%%%%%%%%
$
\inferrule
{
\mods \jp \FH{e \wedge \locExpr}{s}{e}
}
{\mods \jp \FH{e}{\while{e}{\locExpr}{s}}{e \wedge \neg \locExpr}}
\;(\textsc{While})
$
\medskip
%%%%%%%%%%%%%%%%%%%%
$
\inferrule
{
\phi \Rightarrow \phi' \\ \FH{\phi'}{s}{\psi'} \\ \psi' \Rightarrow \psi
}
{\FH{\phi}{s}{\psi}}
\;(\textsc{Weaken})
$
\medskip
%%%%%%%%%%%%%%%%%%%%
$
\inferrule
{
\mods \phi \FH{\phi}{s}{\psi} \\ \accessVars(\rho) \cap \mods = \{\}
}
{\mods \FH{\rho \wedge \phi}{s}{\rho \wedge \psi}}
\;(\textsc{Frame})
$
\medskip
%%%%%%%%%%%%%%%%%%%%
$
\inferrule
{
\specs(P) = (\phi, \mods, \psi) \\
\mods' \jp \FH{\phi}{\bodies(P)}{\psi} \\
\mods' \subseteq \mods
}
{
\jp P
}
\;(\textsc{Procedure})
$
\medskip
%%%%%%%%%%%%%%%%%%%%
$
\inferrule
{
T = (\varsTL, (\varsL, \StmtStack)) \\
G\!\cdot\!\varsTL\!\cdot\!\varsL \jp \phi \rightarrow \true \\\\
\mods \jp \FH{\phi}{\StmtStack}{\true}
}
{
G \jp T
}
\;(\textsc{Thread})
$
\medskip
%%%%%%%%%%%%%%%%%%%%
$
\inferrule
{
\forall P \in \ProcName.\ \jp P \\
\forall 1 \le i \le n.\ G \jp T_i
}
{
\jp (\bodies, \actions, \specs, \refines, \varsG, T_1 \ldots T_n)
}
\;(\textsc{Program})
$
\medskip
%%%%%%%%%%%%%%%%%%%%
}
\caption{Sequential rules for partial correctness}
\label{fig:sequential-correctness}
\end{figure}

{\bf Non-interference.}
Let $\Yields$ be the set of yield predicates, preconditions, and postconditions in the program.
Let $\Ablocks$ be the set of atomic blocks in the program except those inside the bodies of procedures
in $\dom(\refines)$.
Let $\FV \subseteq \VarName \setminus \Var$ be a set of fresh variables and $\Lambda$ be a one-one 
substitution function from $\ThreadLocals \cup \Locals$ to $\FV$.
Let $\Lambda(\phi)$ represent the result of applying $\Lambda$ to the expression $\phi$.
For each predicate ($\phi,\lins_y) \in \Yields$
and for each atomic block $\ablock{e,\lins_a}{s} \in \Ablocks$, we prove the following judgment:
\[
\FH{\Lambda(\phi) \wedge e \wedge
  \{lv(\Lambda(x)) \mid x \in \lins_y\} \uplus
  \{lv(x) \mid x \in \lins_a\} \subseteq \linsmax
}{s}{\Lambda(\phi)}
\]
For each predicate $\phi \in \Yields$ and for each $P \in \dom(\refines)$ such that
$\actions(\refines(P)) = (\rho, \alpha, m)$, we prove the following to be unsatisfiable:
\[
\Lambda(\phi) \circ \rho \circ \alpha \circ \neg\Lambda(\phi)
\]

\subsection{Refinement}
\begin{figure}
\scriptsize{
\medskip
%%%%%%%%%%%%%%%%%%%%
$
\inferrule
{
}
{P \jr \skipstmt : \epsilon}
\;(\textsc{Skip})
$
\medskip
%%%%%%%%%%%%%%%%%%%%
$
\inferrule
{
}
{P \jr \assert{\locExpr} : \epsilon}
\;(\textsc{Assert})
$
\medskip
%%%%%%%%%%%%%%%%%%%%
$
\inferrule
{
}
{P \jr \yield{e,\lins} : \epsilon}
\;(\textsc{Yield})
$
\medskip
%%%%%%%%%%%%%%%%%%%%
$
\inferrule
{
\actions(\refines(P')) = (\rho',\alpha',m) \\\\ \rho' \circ \alpha' \Rightarrow \Havoc(\{\})
}
{P \jr \call{P'} : \epsilon}
\;(\textsc{Call-Loop})
$
\medskip
%%%%%%%%%%%%%%%%%%%%
$
\inferrule
{
\actions(\refines(P)) = (\rho,\alpha,m) \\ \actions(\refines(P')) = (\rho',\alpha',m) \\\\ \rho' \circ \alpha' \Rightarrow \alpha
}
{P \jr \call{P'} : N}
\;(\textsc{Call-Action})
$
\medskip
%%%%%%%%%%%%%%%%%%%%
$
\inferrule
{
\actions(\refines(P')) = (\rho',\alpha',m) \\ \rho' \circ \alpha' \Rightarrow \Havoc(\{\})
}
{P \jr \async{P'} : \epsilon}
\;(\textsc{Async})
$
\medskip
%%%%%%%%%%%%%%%%%%%%
$
\inferrule
{
s \preceq \mathit{old}(e) \Rightarrow \Havoc(L)
}
{P \jr \ablock{e,\lins}{s} : \epsilon}
\;(\textsc{Ablock-Loop})
$
\medskip
%%%%%%%%%%%%%%%%%%%%
$
\inferrule
{
\actions(\refines(P)) = (\rho,\alpha,m) \\ s \preceq \mathit{old}(e) \Rightarrow \beta
}
{P \jr \ablock{e,\lins}{s} : N}
\;(\textsc{Ablock-Action})
$
\medskip
%%%%%%%%%%%%%%%%%%%%
$
\inferrule
{
P \jr s_1 : \re_1 \\ P \jr s_2 : \re_2
}
{P \jr s_1;s_2 : \re_1\cdot\re_2}
\;(\textsc{Seq})
$
\medskip
%%%%%%%%%%%%%%%%%%%%
$
\inferrule
{
P \jr s_1 : \re_1 \\ P \jr s_2 : \re_2
}
{P \jr \ite{\locExpr}{s_1}{s_2} : \re_1+\re_2}
\;(\textsc{Ite})
$
\medskip
%%%%%%%%%%%%%%%%%%%%
$
\inferrule
{
P \jr s : \re
}
{P \jr \while{e}{\locExpr}{s} : \re^*}
\;(\textsc{While})
$
\medskip
%%%%%%%%%%%%%%%%%%%%
$
\inferrule
{
\forall P \in \dom(\refines).\ \bodies(P) \jr \{N\}
}
{
\jr (\bodies, \actions, \specs, \refines, \varsG, T_1 \ldots T_n)
}
\;(\textsc{Program})
$
\medskip
%%%%%%%%%%%%%%%%%%%%
}
\caption{Refinement rules}
\label{fig:refinement}
\end{figure}

\begin{figure}
\scriptsize{
\medskip
%%%%%%%%%%%%%%%%%%%%
$
\inferrule
{
}
{\skipstmt \preceq \stutter}
\;(\textsc{Skip})
$
\medskip
%%%%%%%%%%%%%%%%%%%%
$
\inferrule
{
}
{\assert{\locExpr} \preceq \stutter}
\;(\textsc{Assert})
$
\medskip
%%%%%%%%%%%%%%%%%%%%
$
\inferrule
{
}
{\yield{e,\lins} \preceq \false}
\;(\textsc{Yield})
$
\medskip
%%%%%%%%%%%%%%%%%%%%
$
\inferrule
{
\actions(A) = (\rho, \alpha, m) 
}
{\call{A} \preceq \alpha}
\;(\textsc{Atomic})
$
\medskip
%%%%%%%%%%%%%%%%%%%%
$
\inferrule
{
}
{\call{P} \preceq \false}
\;(\textsc{Call})
$
\medskip
%%%%%%%%%%%%%%%%%%%%
$
\inferrule
{
}
{\async{P} \preceq \stutter}
\;(\textsc{Async})
$
\medskip
%%%%%%%%%%%%%%%%%%%%
$
\inferrule
{
s \preceq \alpha
}
{\ablock{e,\lins}{s} \preceq \alpha}
\;(\textsc{Ablock})
$
\medskip
%%%%%%%%%%%%%%%%%%%%
$
\inferrule
{
s_1 \preceq \alpha_1 \\ s_2 \preceq \alpha_2
}
{s_1;s_2 \preceq \alpha_1 \circ \alpha_2}
\;(\textsc{Seq})
$
\medskip
%%%%%%%%%%%%%%%%%%%%
$
\inferrule
{
s_1 \preceq \alpha_1 \\ s_2 \preceq \alpha_2
}
{\ite{\locExpr}{s_1}{s_2} \preceq (\locExpr \circ \alpha_1) \vee (\neg \locExpr \circ \alpha_2)}
\;(\textsc{Ite})
$
\medskip
%%%%%%%%%%%%%%%%%%%%
$
\inferrule
{
s \preceq \beta \\ \neg \locExpr \circ \stutter \Rightarrow \alpha \\ \beta \circ \alpha \Rightarrow \alpha 
}
{\while{e,\alpha}{\locExpr}{s} \preceq e \circ \alpha \circ \neg \locExpr}
\;(\textsc{While})
$
\medskip
%%%%%%%%%%%%%%%%%%%%
$
\inferrule
{
s \preceq \alpha \\ \alpha \Rightarrow \alpha'
}
{s \preceq \alpha'}
\;(\textsc{Weaken})
$
\medskip
}
\caption{Abstracting statements by actions}
\label{fig:statement-to-action}
\end{figure}

\subsection{Yield sufficiency}

{\bf Commutativity.}
Let $\FV_1,\FV_2 \subseteq \VarName \setminus \Var$ be two sets of disjoint fresh variables.
Let $\Lambda_1$ and $\Lambda_2$ be one-one 
substitution functions from $\ThreadLocals \cup \Locals$ to $\FV_1$ and $\FV_2$ respectively.
For all $A_1,A_2 \in \ActionName$ such that $\actions(A_1) = (\rho_1,\alpha_1,m_1)$ and $\actions(A_2) = (\rho_2,\alpha_2,m_2)$,
if $m_1 \in \{B,R\}$ or $m_2 \in \{B,L\}$ then prove the following valid:
\[
\begin{array}{l}
(\Lambda_1(\rho_1) \wedge \Lambda_2(\rho_2)) \circ (\Lambda_1(\alpha_1) \wedge \Same(\FV_2)) \circ (\Lambda_1(\alpha_2) \wedge \Same(\FV_1)) \\ 
\Rightarrow (\Lambda_1(\alpha_2) \wedge \Same(\FV_1)) \circ (\Lambda_1(\alpha_1) \wedge \Same(\FV_2))
\end{array}
\]
For all $A_1,A_2 \in \ActionName$ such that $\actions(A_1) = (\rho_1,\alpha_1,m_1)$ and $\actions(A_2) = (\rho_2,\alpha_2,m_2)$,
if $m_1 \in \{B,R\}$ then prove the following unsatisfiable:
\[
\Lambda_1(\rho_1) \circ (\Lambda_2(\rho_2) \circ \Lambda_2(\alpha_2)) \circ \neg \Lambda_1(\rho_1)
\]
For all $A_1,A_2 \in \ActionName$ such that $\actions(A_1) = (\rho_1,\alpha_1,m_1)$ and $\actions(A_2) = (\rho_2,\alpha_2,m_2)$,
if $m_1 \in \{B,L\}$ then prove the following unsatisfiable:
\[
\neg \Lambda_1(\rho_1) \circ (\Lambda_2(\rho_2) \circ \Lambda_2(\alpha_2)) \circ \Lambda_1(\rho_1)
\]

\begin{figure}
\includegraphics[scale=0.35]{YieldTypeCheckingAutomaton.pdf}
\caption{Specification for yield sufficiency}
\label{fig:YieldTypeCheckingAutomaton}
\end{figure}

\begin{figure}
\scriptsize{
\medskip
%%%%%%%%%%%%%%%%%%%%
$
\inferrule
{
}
{\jy \skipstmt : (x, x)}
\;(\textsc{Skip})
$
\medskip
%%%%%%%%%%%%%%%%%%%%
$
\inferrule
{
}
{\jy \assert{\locExpr} : (x, x)}
\;(\textsc{Assert})
$
\medskip
%%%%%%%%%%%%%%%%%%%%
$
\inferrule
{
}
{\jy \yield{e,\lins} : (x, \RM)}
\;(\textsc{Yield})
$
\medskip
~\\
%%%%%%%%%%%%%%%%%%%%
$
\inferrule
{
P \in \dom(\refines) \\ \actions(\refines(P)) = (\rho, \alpha, B)
}
{\jy \call{P} : (x, x)}
\;(\textsc{CallBothMover})
$
\medskip
%%%%%%%%%%%%%%%%%%%%
$
\inferrule
{
P \in \dom(\refines) \\ \actions(\refines(P)) = (\rho, \alpha, R)
}
{\jy \call{P} : (\RM, \RM)}
\;(\textsc{CallRightMover})
$
\medskip
%%%%%%%%%%%%%%%%%%%%
$
\inferrule
{
P \in \dom(\refines) \\ \actions(\refines(P)) = (\rho, \alpha, L)
}
{\jy \call{P} : (x, \LM)}
\;(\textsc{CallLeftMover})
$
\medskip
%%%%%%%%%%%%%%%%%%%%
$
\inferrule
{
P \in \dom(\refines) \\ \actions(\refines(P)) = (\rho, \alpha, N)
}
{\jy \call{P} : (\RM, \LM)}
\;(\textsc{CallNonMover})
$
\medskip
%%%%%%%%%%%%%%%%%%%%
$
\inferrule
{
P \not \in \dom(\refines)
}
{\jy \call{P} : (x, \RM)}
\;(\textsc{CallYield})
$
\medskip
%%%%%%%%%%%%%%%%%%%%
$
\inferrule
{
}
{\jy \async{P} : (x, \LM)}
\;(\textsc{Async})
$
\medskip
%%%%%%%%%%%%%%%%%%%%
$
\inferrule
{
x \in \{\RM,\CM\}
}
{\jy \ablock{e,\lins}{s} : (x, \CM)}
\;(\textsc{Ablock})
$
\medskip
%%%%%%%%%%%%%%%%%%%%
$
\inferrule
{
\jy \StmtStack : (x,y)
}
{
\jy (\varsL,\StmtStack) : (x,y)
}
\;(\textsc{StackFrame})
$
\medskip
%%%%%%%%%%%%%%%%%%%%
$
\inferrule
{
\jy \StmtStack : (x, y) \\ \jy s : (y, z)
}
{\jy \StmtStack;s : (x, z)}
\;(\textsc{Seq})
$
\medskip
%%%%%%%%%%%%%%%%%%%%
$
\inferrule
{
\jy s_1 : (x, y) \\ \jy s_2 : (x, y)
}
{\jy \ite{\locExpr}{s_1}{s_2} : (x, y)}
\;(\textsc{Ite})
$
\medskip
%%%%%%%%%%%%%%%%%%%%
$
\inferrule
{
\jy s : (x, x)
}
{\jy \while{e}{\locExpr}{s} : (x, x)}
\;(\textsc{While})
$
\medskip
%%%%%%%%%%%%%%%%%%%%
$
\inferrule
{
\jy \bodies(P) : (x, y)
}
{
\jy P
}
\;(\textsc{Procedure})
$
\medskip
%%%%%%%%%%%%%%%%%%%%
$
\inferrule
{
T = (\varsTL, (\varsL, \StmtStack)) \\
\jy \StmtStack : (x, y)
}
{
\jy T
}
\;(\textsc{Thread})
$
\medskip
%%%%%%%%%%%%%%%%%%%%
$
\inferrule
{
\forall P \in \ProcName.\ \jy P \\
\forall 1 \le i \le n.\ \jy T_i
}
{
\jy (\bodies, \actions, \specs, \refines, \varsG, T_1 \ldots T_n)
}
\;(\textsc{Program})
$
\medskip
%%%%%%%%%%%%%%%%%%%%
}
\caption{Yield sufficiency rules}
\label{fig:yield-sufficiency}
\end{figure}

\subsection{Soundness}

\begin{theorem}
Let $\Prog$ and $\Prog'$ be two programs such that the following conditions are satisfied:
\begin{enumerate}
\item
$\vdash \Prog$ and $\vdash \Prog'$ and $\Prog$ is abstracted by $\Prog'$.
\item 
All finite executions of $\Prog'$ are safe.
\item
All infinite executions of $\Prog'$ are responsive.
\item
The program $\Prog$ is commutativity-safe.
\item
The program $\Prog$ is interference-free.
\item
$\jp Prog$, $\jr Prog$, and $\jy Prog$.
\end{enumerate}
Then all finite executions of $\Prog'$ are safe.
\end{theorem}

\subsection{Responsiveness}

\[
\inferrule
{
\FH{e \wedge \locExpr}{s}{e} \\ e \wedge \locExpr \Rightarrow f \geq 0 \\ s \preceq (old(e \wedge \locExpr) \Rightarrow \mathit{old}(f) > f)
}
{\FH{e}{\while{e,\alpha,f}{\locExpr}{s}}{e \wedge \neg \locExpr}}
\;(\textsc{While})
\]


\section{Implementation}
\label{sec:implementation}

We have implemented the method described in this section as a conservation extension 
of the Boogie~\cite{BarnettCDJL05} language and verifier.
Our implementation provides new language primitives for linear variables, asynchronous and parallel procedure calls, 
yields, and atomic actions as procedure specifications.
At its core, Boogie is an unstructured language comprising code blocks and goto statements.
Our implementation handles the complexity of unstructured control flow.
To simplify the exposition, our formalization uses Floyd-Hoare triples to present sequential correctness and 
annotated atomic code blocks to present refinement and non-interference checks.
However, our implementation is considerably more automated.  
All the annotations, except those at yields, loops, and procedure boundaries, are automatically generated 
using the technique of verification conditions~\cite{BL05}.
Annotated atomic code blocks are also inferred automatically.
Non-interference checks are collected as inlined procedures
invoked at appropriate places within the code of a procedure for increased precision.

The judgment $\Refines \jy \Prog$ from Section~\ref{sec:yield-sufficiency} is fully automated.
We adapted an algorithm by Henzinger et al.\cite{HenzingerHK95} for computing the similarity relation of 
labeled graphs to check that the control flow graph of a procedure is simulated by
the $\YSA$ automaton.
The complexity of the algorithm is $O(n*m)$, where $n$ and $m$ are the number of control-flow graph nodes and edges.
In practice, this part of the verification is fast.

A large proof usually comprises multiple layers of refinement chained together.
Our implementation allows the specification of multiple views of a program in a single file by using the mechanism of {\em layers}.
The programmer may attach a positive layer number to each annotation and procedure; 
version $i$ of the program is constructed from annotations labeled $i$ and procedures labeled at least $i$.
We have implemented a type checker to make sure that layer numbers are used appropriately, e.g., 
it is illegal for a procedure with layer $i$ to call a procedure with layer $j$ greater than $i$.

Our verifier also supports hiding of global variables.
Hiding global variables is particularly useful because simple summaries can be written for complicated procedures
once enough variables relevant to their implementation have been hidden.

\section{Overview}
\label{sec:overview}
\begin{figure}
\begin{verbatim}
var x:int;
\end{verbatim}
\begin{verbatim}
procedure p()
  requires x >= 5;
  ensures  x >= 8;
{
  yield x >= 5;  x := x + 1;
  yield x >= 6;  x := x + 1;
  yield x >= 7;  x := x + 1;
}
\end{verbatim}
\begin{verbatim}
procedure q() modifies x; { x := x + 3; }
\end{verbatim}
\caption{Program~1}
\label{fig:ex1}
\end{figure}

We present an overview of the \civl language through a sequence of examples.
Figure~\ref{fig:ex1} shows Program~1 containing a procedure {\tt p}
executing concurrently with another procedure {\tt q}. 
An execution of a \civl program is non-preemptive; a thread explicitly yields control to the
scheduler via the {\tt yield} statement following which execution continues on a 
nondeterministically chosen thread.
The {\tt yield} statement has a local assertion $\varphi$ attached to it.
The yielding thread must establish $\varphi$ when it yields and the execution of other threads 
must preserve $\varphi$; these two requirements are usually known as {\em sequential correctness}
and {\em non-interference}, respectively.
To check these requirements, the \civl verifier creates verification conditions, whose number is at most
quadratic in the number of yield statements in the program.
For example, in Program~1 each yield predicate in {\tt p} must be checked against the action 
{\tt x := x + 3} in {\tt q}.

\civl requires that a procedure that may potentially execute a yield statement during its execution 
must be annotated as {\tt yielding}.
This annotation is checked in a manner similar to the checking of modifies clauses; if a procedure is labeled 
as {\tt yielding} so must all of its callers.
A procedure marked as {\tt yielding} is exempt from providing a modifies clause; 
the presence of {\tt yielding} allows the caller to conclude that any global variable could have changed
potentially as a result of modification by a concurrently-executing thread.
A procedure not labeled as {\tt yielding} is called atomic; such a procedure must supply a modifies clause as usual.

\begin{figure}
\begin{verbatim}
var x:int;
\end{verbatim}
\begin{verbatim}
procedure yield_x(n:int)
  requires x >= n;
  ensures  x >= n;
{
  yield x >= n;
}
\end{verbatim}
\begin{verbatim}
procedure p()
  requires x >= 5;
  ensures  x >= 8;
{
  call yield_x(5);  x := x + 1;
  call yield_x(6);  x := x + 1;
  call yield_x(7);  x := x + 1;
}
\end{verbatim}
\caption{Program~2}
\label{fig:ex2}
\end{figure}

{\bf From quadratic to linear verification conditions.}
Figure~\ref{fig:ex2} shows Program~2, a variation of Program~1 in which the procedure {\tt yield\_x} 
contains a single yield statement and {\tt p} calls {\tt yield\_x} instead of yielding directly.
If the calls to {\tt yield\_x} are inlined in Program~2, then we will get Program~1.
Both Program~1 and~2 are verifiable in \civl but the cost of verifying Program~2 is less because it has fewer yield statements.
In fact, if it is possible to capture all interference in a concurrent program in a single yield predicate, 
then the trick in Program~2 can be used to verify the program with a linear number of verification conditions.

\begin{figure}
\begin{verbatim}
procedure yield_x(n: int)
  requires x >= n;
  ensures  x >= n;
{
  yield x >= n;
}
\end{verbatim}
\begin{verbatim}
procedure p()
  requires x >= 5;
  ensures  x >= 8;
{
  call yield_x(x);  x := x + 1;
  call yield_x(x);  x := x + 1;
  call yield_x(x);  x := x + 1;
}
\end{verbatim}
\caption{Program~3}
\label{fig:ex3}
\end{figure}

{\bf Encoding rely-guarantee specifications.}
Figure~\ref{fig:ex3} shows Program~3, yet another variation of Programs~1 and~2 which shows how to encode a rely-guarantee-style~\cite{Jones83} (two-state invariant)
proof using \civl's one-state yield statements. 
The standard rely-guarantee specification to prove the assertions in {\tt p} is that the environment of {\tt p} 
may only increase {\tt x}.
We can encode this in \civl by first exploiting the trick in Program~2 to factor out the yield statement in a separate procedure
and then passing the current value of {\tt x} as a parameter to {\tt yield\_x}.
In fact, our implementation of \civl requires even less work; the value of {\tt x} upon entering {\tt yield\_x} is available 
in the postcondition using the syntax {\tt old(x)}, allowing us to write {\tt yield\_x} without any parameter as follows:
\begin{verbatim}
procedure yield_x()
  ensures  x >= old(x);
{
  yield x >= old(x);
}
\end{verbatim}


\begin{figure}
\begin{verbatim}
type Tid;
var linear alloc:[Tid]bool;
const nil: Tid;
procedure Allocate() returns (linear tid: Tid);
  modifies alloc;
  ensures tid != nil;
\end{verbatim}
\begin{verbatim}
var a:[Tid]int;
\end{verbatim}
\begin{verbatim}
procedure main()
{
  var linear tid: Tid;
  while (true) {
    call tid := Allocate();
    async call P(tid);
    yield true;
  }
}
\end{verbatim}
\begin{verbatim}
procedure P(linear tid: Tid)
  requires tid != nil;
  ensures a[tid] == old(a)[tid] + 1;
{
  var t: int;
  t := a[tid];
  yield t == a[tid];
  a[tid] := t + 1;
}
\end{verbatim}
\caption{Program 4}
\label{fig:ex5}
\end{figure}

{\bf Linear variables.}
Program~5 in Figure~\ref{fig:ex5} introduces linear variables, a feature of \civl 
that is useful for encoding disjointness among values contained in 
different variables.  
This example uses this feature for encoding the concept of an identifier 
that is unique to each thread.
Program~5 contains a shared global array {\tt a} indexed by an uninterpreted type {\tt Tid} 
representing the set of thread identifiers.
A collection of threads are executing procedure {\tt P} concurrently.
The identifier of the thread executing {\tt P} is passed in as the parameter {\tt tid}.
A thread with identifier {\tt tid} owns {\tt a[tid]} and can increment it without danger of interference.
The yield assertion {\tt t == a[tid]} in {\tt P} indicates this expectation, yet it is not possible to prove it 
unless the reasoning engine knows that the value of {\tt tid} in one thread is distinct 
its value in a different thread.

Instead of building a notion of thread identifiers into \civl, we provide a more 
primitive and general notion of linear variables.
The \civl type system ensures that values contained in linear variables cannot be duplicated.
Consequently, the parameter {\tt tid} of distinct concurrent calls to {\tt P} are known to be distinct;
the \civl verifier exploits this invariant while checking for non-interference.

Program~5 also shows the mechanisms of allocation of thread identifiers,
based on the use of global variable {\tt alloc}, the constant {\tt nil}, and the procedure 
{\tt Allocate}.  
Section~\ref{sec:formal} describes values and linear variables like {\tt nil} and {\tt alloc} in more detail.

\begin{figure}
\begin{verbatim}
var Color:[int]int; // 1: WHT(), 2: GRY(), 3: BLK()
var Set:[int]bool;

procedure SetColGrayIfWht({:cnst "tid"} tid:int,
                                       addr:int)
ensures {:atomic} [if (Color[addr] == WHT(){)
                     Color[addr] := GRY();
                     Set[addr] := true;
                   }]
{
  call cNoLock:= GetColorNoLock();
  call YieldColorOnlyGetsDarker();
  if (cNoLock == WHT()) {
       call L_SetColorToGrayIfWht(tid,addr);
  }
}

procedure YieldColorOnlyGetsDarker()
  ensures Color >= old(Color);

procedure L_SetColGrayIfWht({:cnst "tid"} tid:int,
                                         addr:int)
ensures {:atomic} [if (Color[addr] == WHT(){)
                     Color[addr] := GRY();
                     Set[addr] := true;
                   }]
{
  call AcquireLock(tid);
  call cLock := GetColorLocked(tid);
  if (cLock == WHT()) {
    call SetColorLocked(tid, GRY());
    call InsertSetLocked(tid, addr);
  } 
  call ReleaseLock(tid);
}

procedure AcquireLock({:cnst "tid"} tid: X);
  ensures {:right} [ ...]
procedure ReleaseLock({:cnst "tid"} tid: X);
  ensures {:left} [...]
procedure InsertSetLocked(
              {:cnst "tid"} tid:X, addr:int); 
  ensures {:both} [...]
procedure SetColorLocked(
                {:cnst "tid"} tid:X, nC:int); 
  ensures {:both} [...]
procedure GetColorLocked(
       {:cnst "tid"} tid:X) returns (cl:int);
  ensures {:both} [...];

\end{verbatim}
\caption{Program 5}
\label{fig:reft}
\end{figure}

{\bf Refinement.} 

\civl supports the
verification of a refinement relationship between different versions of a
program. In higher-level versions, calls to some procedures in the
lower levels are
replaced by their atomic action specifications. 
Program 5 in Figure~\ref{fig:reft} introduces refinement in \civl. 
Atomic action specifications are shown in square brackets. 

In earlier phases of verification (not shown) \civl has verified that
the implementations of the five procedures called in {\tt
  L\_SetColGrayIfWht} satisfy their atomic action specifications, shown
at the bottom of the figure. In Program 5, in the first refinement
pass, \civl verifies that {\tt L\_SetColGrayIfWhite} is atomic and
satisfies its specification. In the second, higher-level 
pass, using the atomic action specification for {\tt
  L\_SetColGrayIfWhite}, Owicki-Gries-style reasoning using the yield
predicate in {\tt
  YieldColorsGetDarker()} and refinement reasoning, \civl verifies
that {\tt SetColGrayIfWht()} satisfies its atomic action
specification. 

This example is a simplified version of a pattern in the garbage
collector we verified. Here, several threads concurrenty check 
whether the global variable {\tt Color[addr]} has the value {\tt WHITE()}
and set it to {\tt GRAY()} and insert {\tt addr} into a 
set, represented by a map from {\tt addr} to {\tt bool} in this
example. 

 
The procedure  
\exC{L\_SetColGrayIfWhite} holds a global lock while reading
reading and possibly modifying {\tt Color[addr]} and inserting {\tt
  addr} into the set. 
As a performance optimization \exC{SetColGrayIfWhite}, before calling
\exC{L\_SetColGrayIfWhite}, reads \exC{Color[addr]} without holding a lock
and checks if it has the value {\tt WHITE()}. The correctness of this
optimization is not obvious -- if {\tt Color[addr]} is {\tt GRAY()} when
read without holding a lock, but can be set to {\tt WHITE()} by other 
threads before {\tt SetColGrayIfWht()} returns, then the atomic action
specification of the procedure may not be satisfied. The yield
predicate in {\tt YieldColorsOnlyGetDarker()} expresses the fact that
other threads in the environment of the thread running {\tt
  SetColGrayIfWht} can only modify {\tt Color[addr]} to a higher
value. \civl verifies that this is the case, 

Other than phrasing reduction as a type-checking problem, this is
ordinary static reduction reasoning. The verification of atomicity for
\exC{SetColGrayIfWht} highlights the capability of \civl to combine
Owicki-Gries-style reasoning and reduction to verify refinement.

Observe that in the procedure \exC{L\_SetColGrayIfWht}, there are no {\tt
  yield}s between statements. Since threds explicitly yield
control, this means that the entire body of \exC{L\_SetColGrayIfWht}
is executed atomically. In a real execution, control can switch
between threads at any point in the code. The absence of {\tt yield}s,
and treating the entire body of \exC{L\_SetColGrayIfWht} as atomic 
is justified by reasoning about mover types and reduction. The
four procedures called in \exC{L\_SetColGrayIfWht} have the mover
types claimed in their declarations and verified by \civl. Given
the mover types of all statements, including accesses to local
variables, calls to procedures with and without atomic action
specifications, a ``yield type checker'' in the \civl verifier 
(explained in Section~\ref{}) checks whether the absence of yields is
justified using reduction.  \civl then verifies that the body of
\exC{L\_SetColGrayIfWht} satisfies its atomic specification. 


%%% Local Variables: 
%%% mode: latex
%%% TeX-master: "paper"
%%% End: 

\section{Modules}
\label{sec:modules}

This section describes a simple module system built on \civl that allows separate verification of modules,
allowing programmers to check a large program by breaking it into smaller pieces and checking the pieces independently.
A key challenge for modular verification in \civl is the $\CommutativitySafe$ and $\InterferenceSafe$ judgments.
Naively, these are whole-program judgments,
quadratically checking all pairs of actions or all pairs of yields and atomic blocks from an entire program.
To check these judgments on a per-module basis rather for a whole-program,
we observe that $\CommutativitySafe$ and $\InterferenceSafe$ are trivially true for operations that act on disjoint sets of global variables.
If an atomic block modifies only variables $g_1$ and $g_2$, it will not interfere with a yield that refers only to variables $g_3$ and $g_4$.
More generally, let each module $M$ own a set of global variables, such that each global variable is owned by exactly one module,
and decree that only $M$'s procedures and actions can access $M$'s global variables.
Formally, define ownership for global variables, procedures and actions as:

\hspace{30mm}\begin{tabular}{rclcl}
$M$ & $\in$ & $\Module$ \\
$\own_{\Global}$ & $\in$ & $\Global \rightarrow \Module$ \\
$\own_{\ProcName}$ & $\in$ & $\ProcName \rightarrow \Module$ \\
$\own_{\ActionName}$ & $\in$ & $\ActionName \rightarrow \Module$ \\
\end{tabular}

\noindent
If $\own_{\ProcName}(P) = M$, then $P$'s preconditions, postconditions, and statements only refer to global $g$ where $\own_{\Global}(g) = M$.
Similarly, if $\own_{\ProcName}(A) = M$, then $A$'s gate and transition relation can depend only on $g$ where $\own_{\Global}(g) = M$.
With this restriction, $\CommutativitySafe$ and $\InterferenceSafe$ can be checked for each module $M$ in isolation;
no other module $M'$ can interfere with the global variables owned by $M$.
Note that $\own_{\Global}$, $\own_{\ProcName}$, and $\own_{\ActionName}$ can change across refinement layers.
For example, a library module implementing locks may define a variable to represent the abstract state of a lock;
after the lock module is verified at a low layer,
another module can take ownership of the lock variable in a higher layer
(see \cite{gc-techreport} for a detailed example of ownership transfer across three layers, from a lock module to a datatype module to a client module).

\section{Case study: a verified concurrent GC algorithm}
\label{sec:experience}

The \civl verifier has been under development for around two years.  
Over that period, we have developed a collection of 32 benchmarks, 
ranging in size from 17 to 539 LOC, to illustrate various features of
\civl and for regression testing as we evolved the verifier.
In addition to microbenchmarks, this collection also includes
standard benchmarks from the literature such as a multiset implementation~\cite{ElmasTQ05}, 
the ticket algorithm~\cite{FarzanKP14}, 
Treiber stack~\cite{Herlihy2008}, work-stealing queue~\cite{Blumofe1999},
device cache~\cite{ElmasQT09}, and lock-protected increment~\cite{FlanaganQ03}. 
The \civl verifier is fast; the entire benchmark set verifies in 20 seconds on a standard 4-core Windows PC (2.8GHz, 8GB)
with no benchmark requiring more than a few seconds.

In addition to these 32 small benchmarks,
we also verified a larger algorithm:
a concurrent mark-sweep garbage collector (GC).
The rest of this section discusses the GC and its verification in detail.

\subsection{Garbage collector}


\begin{figure*}
\includegraphics[scale=1.0]{VerifiedGC.pdf}
\caption{Verified garbage collector phases 5, 6 (pseudocode excerpts in solid boxes; atomic action specs in dashed boxes)}
\label{fig:VerifiedGC}
\end{figure*}

We demonstrate the verification methodology and tool on a realistic modern concurrent garbage collector algorithm.
Our algorithm builds on the concurrent collector of Dijkstra et al.~\cite{dijk78}.
Dijkstra's collector is attractive for verification because it maintains a simple tri-color invariant
on the heap objects (in contrast to snapshot-oriented collectors~\cite{doli93,doli94,doma00,azat03}
whose tri-color invariants are more subtle).
By itself, though, Dijkstra's collector is not a modern or performant collector.
First, it becomes incorrect in the presence of more than one program thread (mutator).
Second, it requires that the write-barrier be run not only on updates of heap pointers,
but also on modifications of root pointers, i.e., on modifications of the runtime stacks and the registers;
modern high-performance collectors avoid this overhead.

Therefore, our algorithm (shown inside the solid boxes in Figure~\ref{fig:VerifiedGC}) extends and modifies Dijkstra's collector
to make it work with parallel programs and to not require a write-barrier on root modifications.
Like Dijkstra's collector, our algorithm first {\em marks} all objects reachable from roots (registers and stacks),
shown in Figure~\ref{fig:VerifiedGC}'s Mark procedure, and then {\em sweeps} away all unreached objects,
shown in Figure~\ref{fig:VerifiedGC}'s Sweep procedure.
As in Dijkstra's collector, our algorithm employs a tri-color abstraction to describe the trace of the reachable objects.
Objects are said to be {\em white} if the collector has not seen them yet during the trace.
Objects that the collector encounters become gray and remain gray until the collector scans their children.
Once all the children of an object are noted (meaning that none of them are white), the object becomes black.
The collector works by choosing a node from the set of gray objects (GraySetChoose, called from MarkAllGrays in Figure~\ref{fig:VerifiedGC}),
{\em shading} all its white children to gray (GraySetInsertChildIfWhite), and then removing the object from the gray set by making the object black (GraySetRemove).
The shading operation grays a node if it is white, and does nothing otherwise.
The trace terminates when all roots point to black objects (according to ScanRoots)
and there are no more gray objects (according to IsGraySetEmpty, called by ScanRoots).
Termination is guaranteed because objects can only get darker.
Correctness is guaranteed using an invariant that a black object never points to a white object during the trace
(black objects can only point to gray objects or black objects).
At the end of the trace, objects pointed by the roots must be black, and since no gray objects remain,
black objects only point to black objects,
so the entire set of objects reachable from the roots must be black.

Concurrent mutator operations on objects (ReadField and WriteField in Figure~\ref{fig:VerifiedGC})
could potentially break the no-black-to-white invariant,
because a mutator's WriteField operation could potentially redirect a pointer of a black object to point to a white object.
Therefore, coordination between the program and the concurrent collector is required:
before each raw pointer update (WriteFieldRaw), the WriteField procedure executes a {\em write-barrier} (WriteBarrier).
Before pointer field $x.f$ is set to reference an object $y$,
WriteBarrier shades $y$, ensuring that even if $x$ is black, a pointer from $x$ to $y$
will not violate the no-white-to-black invariant.

The write barrier should shade objects only while the collector is in its mark phase,
not when the collector is sweeping or is idle, and the collector may only switch between phases (mark, sweep, or idle)
when no mutator is in the middle of a WriteField or Alloc operation.
To achieve this (and thereby support correct and efficient support for multiple mutator threads),
we extend Dijkstra's collector with explicit tracking of phases, via a handshaking mechanism~\cite{doli93,doli94}.
A shared variable, collectorPhase, contains the current collector phase.
The collector initiates a handshake by incrementing collectorPhase (in Handshake, called by GarbageCollect).
Each mutator thread keeps cached copy of collectorPhase, and periodically checks to see if
the cached copy mismatches the current collectorPhase, and if so, updates the cached copy
with the most recent value
(in our algorithm, a mutator's call to the allocator, Alloc, checks this in UpdateMutatorPhase,
but the exact location of the check is not critical to correctness).
The GarbageCollector waits until all cached copies equal collectorPhase (WaitForMutators in GarbageCollect),
and then executes a phase (Mark, Sweep, or, for the idle phase, nothing).
Note that each mutator thread can read its own cached phase without acquiring a lock (ReadMutatorPhase in WriteBarrier),
leading to efficient WriteBarrier performance.

Dijkstra's collector requires a write barrier on modifications to roots as well as modifications to objects. 
We eliminate this overhead by employing repeated tracing 
phases until all objects referenced by roots are black. 
A tracing phase starts by stopping all mutators and marking their roots. 
The process of stopping the mutators is similar to a handshake and is done using the CollectorRootScanBarrierStart, CollectorRootScanBarrierWait, and CollectorRootScanBarrierEnd procedures on the collector side and TestRootScanBarrier procedure on the mutator side. 
At the end of the root scan (before the mutators reawaken), all roots point to gray or black objects.
If no gray objects remain (IsGraySetEmpty), then all roots point to black objects, and marking is complete. 
Otherwise, we trace from gray objects until completion and start a new
tracing phase (by stopping the mutators and checking the roots again). 
In a worst-case theoretical scenario we may need to run many root scans and discover more and more white root descendants to trace each time. 
But in practice we usually finish after a small number of scans,
so we obtain correctness and termination in all scenarios and we obtain good performance in real-world scenarios. 

\subsection{Collector Verification in Boogie}
\label{sec:gc-verify}

We have implemented and verified our algorithm in Boogie,
including initialization (Initialize), the GC (GarbageCollect), the allocator (Alloc),
and the mutator operations (ReadField, WriteField, and Eq),
and all the lower-level operations required to implement them (some of these appear
in Figure~\ref{fig:VerifiedGC}; others are omitted from the figure to save space).
To make the verification as realistic as possible,
our Boogie code implements everything in terms of individual CPU operations,
such as load, store, atomic increment/decrement, and CAS (compare-and-swap);
in contrast to some previous work~\cite{gont96},
we do not assume any built-in higher-level operations.
To ease verification, we make some simplifications:
we use a naive allocator (sequential search for free space),
we assume a sequentially consistent memory model,
and we assume that all objects have the same number of fields.
(Except for the assumption of sequential consistency, none of these substantially alter the nature of the proof.)

Overall, our implementation consists of about 2100 lines of Boogie code.
The GC verification takes 60 seconds on the same PC used for microbenchmarks.
The bulk of this time, 54 seconds, is taken by the verification of the refinement checks from Section~\ref{sec:refinement}.
The linear type checking, the yield safety checks, and the commutativity checks take the rest of the time and are insignificant in comparison.

Our verification takes advantage of all techniques in \civl: refinement, assertions, reduction, and linearity.
Refinement gives us extremely simple high-level action specifications for Initialize, ReadField, WriteField, Eq, and Alloc,
shown in their entirety in Figure~\ref{fig:VerifiedGC}'s dashed boxes.
(Initialize and GarbageCollect have empty actions; the GarbageCollector itself is just an internal implementation detail inside Initialize,
which serves only to set up the global GC invariant needed by the other high-level actions.)
Crucially, ReadField, WriteField, Eq, and Alloc appear atomic to mutators, even though internally,
WriteField and Alloc involve many interleaved operations on shared GC data structures.
Figure~\ref{fig:VerifiedGC} shows only shows phases 5 and 6, the two most abstract phases of refinement;
phases 1-4 fill in the implementation details,
such as implementing the set of gray objects as an explicit stack (an array of elements with a pointer to the stack top, in phase 4),
handshaking (phase 3), locks (phase 2), and wrapping the primitive CPU operations in left/right/non-moving atomic actions (phase 1).
Ultimately, the phases are built on trusted CPU-level atomic actions, such as reading and writing roots directly:

\begin{verbatim}
procedure PrimitiveWriteRoot(i:idx, v:int)
  atomic [assert rootAddr(i); root[i] := v;]

procedure PrimitiveReadRoot(i:idx)
  returns (v:int)
  atomic [assert rootAddr(i); v := root[i];]
\end{verbatim}

We write the highest-level action specifications in terms of an abstract view of memory,
as in earlier work on sequential garbage collector verification~\cite{mccr07,hawb09}.
(Abstract memory is infinite and eternal: once allocated, an abstract object lives forever.
Deallocation is an underlying implementation detail, not exposed in the abstract interface.)
Our abstract view describes a machine as consisting of just three variables:
abstract memory memAbs:[obj][fld]obj, mapping object identifiers and fields to other objects,
abstract root values rootAbs:[idx]obj, mapping root names to objects, and
allocSet:[obj]bool, the set of objects allocated so far.
At this high layer of abstraction, we use Boogie's hiding to hide all other variables
(such as the concrete root set, ``root'', the concrete memory, ``mem'', and the colors, ``Color'', used by lower-level procedures).

All operations are relative to root names of type idx.
ReadField, for example, reads an object field from an object pointed to by root x into a root y.
The predicates rootAddr and tidOwns establish that x and y are valid root names, owned by a particular mutator tid.
(We assume that each root is private to a single mutator stack or register file;
sharing between mutator threads takes place through shared pointers to objects.)
The predicates fieldIndex(f) and memAddrAbs(o) establish that x.f is a valid field of a valid object.
Allocation establishes memAddrAbs(o) for newly allocated objects so that they may be used by subsequent ReadField and WriteField operations.
It also establishes o's unique identity by ensuring that it did not previously belong to the allocated object set.

In addition to atomic action specifications, the verification establishes invariants using assertion reasoning
(omitted from Figure~\ref{fig:VerifiedGC} to save space).
For example, Initialize establishes a global mapping toAbs:[int]obj from physical memory mem and abstract memory memAbs:

\begin{verbatim}
(forall x:int, f:fld ::
      memAddr(x)
   && toAbs[x] != nil
   && fieldIndex(f)
  ==>    toAbs[mem[x][f]]
     == memAbs[toAbs[x]][f])
\end{verbatim}

and Mark maintains the no-black-to-white invariant:

\begin{verbatim}
(forall x:int, f:fld ::
      memAddr(x)
   && Black(Color[x])
   && fieldIndex(f)
   && memAddr(mem[x][f])
  ==> Gray(Color[mem[x][f]])
   || Black(Color[mem[x][f]]))
\end{verbatim}

Finally, linearity plays a key role in establishing mutual exclusion.
The GC thread has its own thread id gcTid, and each mutator has its own thread id.
The Initialize procedure consumes gcTid (written here as ``consume'')
and borrows all the mutator thread ids (written as ``linear'', as in Section~\ref{sec:overview}),
so that it's clear that no other concurrent actions are allowed during initialization.
This allows all the internal initialization actions to be both-movers, without requiring any explicit locking.
Initialize must consume gcTid because it passes gcTid to the newly spawned GarbageCollector thread;
since gcTid is consumed, it's impossible to call Initialize twice in an attempt to spawn two parallel GC threads
(which naturally expresses how the algorithm is only safe for a single GC thread).

Because \civl's linearity is based on sets of values,
we can represent thread identifiers as sets that can be subdivided into subsets
(similar to how fractional permissions may be divided into fractions).
During root scanning, each mutator thread places a fraction of its thread id in a global variable,
and reclaims the fraction from the global variable after root scanning completes;
a collector invariant tracks that the global variable contains non-empty fractions from all mutators during root scanning.
Thus, during root scanning, \civl's rules for linearity prove that no interference occurs
between the collector and any mutator operations that require the whole mutator thread id
(``mutatorTidWhole'', used in Figure~\ref{fig:VerifiedGC}'s ReadField, WriteField, Alloc,
and most other mutator operations).

\subsection{Discussion}
We now put atomicity refinement techniques from the literature and
\civl in context by presenting an overview of our design
driver, the stepwise refinement of a garbage collector~\cite{gc-techreport}.
The refinement proof spans six levels of abstraction. 
Each of refinement proof relating two consecutive levels is made feasible by a different
blend of the techniques in \civl. 
% While other refinement techniques have also used garbage collectors as
% case studies, the refinement tasks tackled there bridge only one or two of the levels in 
% our refinement proof\footnote{A more specific discussion of this point
%   can be found in the technical report on the verification of the
%   garbage collector\cite{gc-techreport}.} and only target the refinement verification
% challenges apparent at those levels of the proof. 

The topmost-level description of the garbage collector provides an
idealized, abstract view of memory. 
At this level, none of the lowest-level implementation variables are
visible -- variable hiding has been used to project them away. 
In the top few levels of the garbage collector proof, invariant-based
non-interference reasoning was our primary tool, while reduction
simplified verification by enabling us to use coarser atomic actions and fewer
location invariants.  
Linear variables were used throughout the proof to model the distinct
thread identifiers for the garbage collector thread and mutator
threads, but were most instrumental in encoding single-threaded
execution in the initialization phase of the program. 
For these top few levels of our proof, rely-guarantee and separation-logic-based
approaches would have also performed well, as demonstrated by the
garbage collector proof of Liang et al.~\cite{LiangRGSim}, where
the atomicity of actions in the lower levels our proof is {\em assumed} but not verified.
An important distinguishing capability in \civl is being able to use location invariants rather than pure rely-guarantee reasoning.
This helped interactive proof at the top levels significantly.
For the mark phase of the garbage collector, we made critical use of
different invariants at different locations in procedure bodies. 
While the same non-interference argument could have been encoded in
rely-guarantee reasoning, as we had done ourselves in an earlier
version of our proof, 
it would have required the use of several additional auxiliary shared variables. 
Invariants, rely and guarantee conditions referring to such auxiliary
variables throughout the program made interactive invariant reasoning more difficult to manage. 

In the lower levels of the garbage collector proof, where
correctness of concurrent data structures and synchronization primitives were proven, we made
relatively little use of location invariants, and made heavier use of
linear variables and reduction. 
We also used variable hiding heavily to hide low-level implementation
variables. 
For lower-level refinement tasks, for instance, when verifying the correctness of a
lock-protected concurrently-accessed stack, ownership
arguments, separation logic, or \QED-style atomicity would have been
sufficient. 
But, at the higher levels of our proof, where non-interference
reasoning via invariants and linear variables was indispensable, 
atomicity alone, or ownership or separation logic arguments alone
would have run into difficulty. 

While existing techniques in the literature have as their
``sweet spot'' a few of the refinement proofs in our garbage collector
proof, they run into difficulty in others. 
More critically, they
do not facilitate layering refinement proofs, which is required for stepwise
refinement. 
Using a realistic top-down proof as \civl's design driver led us to
combine in one tool and consistent theory, the verification techniques
of linearity, reduction and non-interference reasoning in the service
of a modular refinement proof directed by the syntactic structure of
the imperative concurrent program. 

%%% SHAZ, I REMOVED THIS SECONDARY NOVELTY POINT, SINCE IT IS MADE
%%% ELSEWHERE IN THE PAPER TOO.
% For this purpose, we also devised
% novel ways to combine automata-theoretic and assertion-based
% verification, and encode the component techniques, e.g., linearity, in
% assertion-based verification.  
% %%%%%%%%% SHAZ PLEASE REFINE OR REMOVE THIS FINAL SENTENCE %%%%%%%%%%%


\section{Related work}
\label{sec:related}

In this section, we discuss related work on verification of concurrent systems.
We first discuss approaches based on refinement and later approaches more directly inspired 
by Floyd-Hoare reasoning.

A generic, widely-applicable approach to specifying and verifying refinement for shared-memory concurrent software is lacking. 
%For many systems, a natural way to write a full functional specification is to provide a description of the entire system at an abstract level. 
%Checking refinement in this context is a means for full functional verification. 
%Alternatively, a series of increasingly more abstract models of a system can be used to reduce the computational cost of verifying safety properties, since safety properties of higher-level models are preserved by lower-level models. 
%This use of refinement is complementary to other techniques that combat the difficulty of verification, such as Floyd-Hoare, rely-guarantee, separation logic for modularity. 

{\bf Refinement-oriented verification.}
Atomic action specifications have been explored by the
Calvin~\cite{FlanaganFQS05} verifier previously. 
Three design choices and features make \civl better suited for
carrying out a stepwise refinement proof of a realistic program
through several layers of abstraction as in our garbage collector
proof. 
First, \civl makes a distinction between preemptive and cooperative
semantics, and carries out refinement verification on a procedure body
with cooperative semantics as enabled by movers types and reduction. 
Calvin attempts to verify refinement directly on the preemptive semantics.
Second, Calvin does not support location invariants and linear variables. 
It instead requires that a two-state rely predicate modeling all possible
interference on the shared state by other threads be supplied. 
A rely predicate that is valid at each interleaving point in
preemptive semantics is difficult to construct. 
The ability to reason on cooperative semantics using location-specific
annotations lowers the complexity of annotations required for the refinement proof.
Third, \civl, unlike Calvin, supports variable hiding. Variable hiding
is an important capability when relating multiple versions of a
program at different levels of abstraction. In our gargabe collector
proof, the top-level description hides almost all implementation
variables to provide an idealized interface to application threads. 

Reduction has been used for program simplification by the
QED~\cite{ElmasQT09} verifier.
Like \civl and Calvin, QED can verify atomicity of procedures. However,
the only method to do so in QED is transforming the procedure body
into a single (yield-free) atomic action using abstraction and reduction. \civl's
support for verifying an atomic specification for a procedure subsumes
QED's but is more general. Most importantly, \civl is able to verify
an atomic specification for a procedure body that is not atomic (has
yields) and is able to leverage location invariants and linear
variables for atomicity
refinement verification as was illustrated in the write barrier
example in Section~\ref{}. QED does not support location invariants and linear variables. 

Another key distinction between \civl and QED is the fact that a proof step in QED is a small rewrite in the concurrent program
that must be justified by potentially expensive reduction and invariant reasoning.
The number of these small proof steps directly affect both programmer
and computer effort. 
By contrast, \civl supports large proof steps, in each of which the bodies of several procedures
are automatically replaced by atomic actions, thereby lowering the cost of both interaction and automation.
In QED, abstract versions of a program are obtained by transforming the initial
version by applying a proof script. 
This makes it difficult to organize a proof in QED by providing a
description of a program at several different levels of abstraction. 
The input to \civl is a single file that correlates several different
representations of the same program at different levels of
abstraction. 
The increased scalability and usability of \civl is key to the construction of proofs spanning a large abstraction gap
between specification and implementation, as is demonstrated by our
garbage collector case study. 

This paragraph needs to be improved.
TLA+~\cite{Lamport2004} has been used for performing refinement proofs between implementations and specifications of protocols.
TLA+ drops down to logic but proofs are tedious and do not scale because of lack of software structuring.

We need a paragraph on refinement proofs in hardware, citing use of SMV~\cite{McMillan00} to verify protocol processor~\cite{Eiriksson2000} 
and Mocha~\cite{AlurHMQRT98} verification of VGI processor~\cite{Henzinger1999}.

Examples are approaches to
checking linearizability of concurrent data
structures~\cite{tacasLin,aliLin}, 
verifying correctness of
concurrent data structure implementations assuming that a certain
ownership discipline is followed~\cite{TuronM11} 

A rely-guarantee rule for simulation proofs between a source and
target programs. 
Thread-modular, nothing about procedure modular, 
Rely-guarantee, manual, separation logic, justify concurrent optimizations
many manual proofs of garbage collectors (e.g.,~\cite{LiangRGSim}). 


{\bf Other approaches.}

Overview of what ``other approaches'' means: There is a single
program, or a concurrent object and its method specification. Methods
help combat complexity and structuring of proof on this flat
description. These approaches are *complementary* to what we do.

VCC~\cite{VCC} is a tool for verifying concurrent C programs.  
Chalice~\cite{LM09} is a language and modular verification tool for concurrent programs. 
VCC does not support refinement and Chalice does so only for sequential programs;
unlike these tools, \civl is designed to verify refinement for concurrent programs.  
VCC and Chalice base their invariant reasoning on objects, object ownership, and type invariants. 
Invariant reasoning in \civl is more primitive and based on assertions in yield statements. 
Although the approach in VCC and Chalice is more convenient when applicable, \civl's approach is more flexible. 
VCC and Chalice can reason sequentially about objects exclusively owned by a thread;
\civl accomplishes the same using linear variables.
Neither VCC nor Chalice support movers and reduction reasoning.

Concurrent separation logic~\cite{OHearn07} reasons about concurrency without 
explicitly checking for non-interference between threads. 
Recently, tools based on this logic that blend in explicit non-interference reasoning have been developed~\cite{SAGL,RGSep}. 
\civl's combination of interference checking and linear variables is an extreme example of this trend,
supplying very primitive abstractions and letting programmers mix and match these abstractions freely.

Is there any other POPL-style work that needs to be cited?


%% A key challenge in specifying and verifying multi-threaded software, differently from sequential software, is that procedures do not provide a clean factoring of a program into modules. 
%% For sequential software, in the popular software verification approach based on the method of Floyd~\cite{Floyd67} and Hoare~\cite{Hoare69}, 
%% the verification of a large software component is performed modularly by verifying each procedure in it separately.
%% This modularity is enabled by the use of interface specifications (preconditions and postconditions) constraining the behavior of each procedure.
%% It is well-known that the Floyd-Hoare method is difficult to generalize to reasoning about concurrent shared-memory programs,
%% primarily because the execution of a procedure may be interfered with by concurrently executing threads.
%% Several attempts have been made to extend Floyd-Hoare reasoning to deal with concurrent interference, 
%% including classical approaches such as location invariants~\cite{Ashcroft75,OwickiG76} and rely-guarantee~\cite{Jones83},
%% and other more recent approaches~\cite{OHearn07,RGSep}. 
%% In this paper, our goal is orthogonal and complementary to these approaches to combating the complexity of verifying safety properties.
%% We provide a methodology in which the {\em refinement} problem can be stated in a modular way guided by the syntactic structure of the program as in the Floyd-Hoare method. 
%% We check refinement by reasoning about the code of a single procedure at a time, in fact, one interference-free step of the implementation at a time. 



\bibliographystyle{abbrv}
\bibliography{paper}

\end{document}

