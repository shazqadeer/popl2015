\section{Examples}
\label{sec:examples}

In this section, we present a collection of examples that illustrate
specification and verification features supported by the \civl verifier.
\subsection{Reasoning about interference}
\label{sec:examples-interference}
\begin{figure}
\begin{verbatim}
var x:int;
\end{verbatim}
\begin{verbatim}
procedure p()
  requires x >= 5;
  ensures  x >= 8;
{
  yield x >= 5;  x := x + 1;
  yield x >= 6;  x := x + 1;
  yield x >= 7;  x := x + 1;
}
procedure q() { x := x + 3; }
\end{verbatim}
\caption{Program~1}
\label{fig:ex1}
\end{figure}

We present an overview of the \civl language and verifier through a sequence of examples.
Figure~\ref{fig:ex1} shows Program~1 containing a procedure {\tt p}
executing concurrently with another procedure {\tt q}. 
As explained earlier, a \civl program is verified with respect to its cooperative semantics; a thread explicitly yields control to the
scheduler via the {\tt yield} statement following which execution continues on a 
nondeterministically chosen thread.
The {\tt yield} statement has a predicate $\varphi$ attached to it.
The yielding thread must establish $\varphi$ when it yields and the execution of other threads 
must preserve $\varphi$; these two requirements in Owicki-Gries-style
reasoning~\cite{OwickiG76} are usually known as {\em sequential correctness}
and {\em non-interference}, respectively.
To check these requirements, the \civl verifier creates verification conditions, whose number is at most
quadratic in the number of yield statements in the program.
For example, in Program~1 each yield predicate in {\tt p} must be checked against the action 
{\tt x := x + 3} in {\tt q}.

\begin{figure}
\begin{verbatim}
procedure yield_x(n:int)
  requires x >= n;
  ensures  x >= n;
{
  yield x >= n;
}
procedure p()
  requires x >= 5;
  ensures  x >= 8;
{
  call yield_x(5);  x := x + 1;
  call yield_x(6);  x := x + 1;
  call yield_x(7);  x := x + 1;
}
\end{verbatim}
\caption{Program~2}
\label{fig:ex2}
\end{figure}

{\bf From quadratic to linear verification conditions.}
Figure~\ref{fig:ex2} shows Program~2, a variation of Program~1 in which the procedure {\tt yield\_x} 
contains a single yield statement and {\tt p} calls {\tt yield\_x} instead of yielding directly.
If the calls to {\tt yield\_x} are inlined in Program~2, then we will get Program~1.
Both Program~1 and~2 are verifiable in \civl but the cost of verifying Program~2 is less because it has fewer yield statements.
In fact, if it is possible to capture all interference in a concurrent program in a single yield predicate, 
then the trick in Program~2 can be used to verify the program with a linear number of verification conditions.

\begin{figure}
\begin{verbatim}
procedure yield_x()
  ensures  x >= old(x);
{
  yield x >= old(x);
}
procedure p()
  requires x >= 5;
  ensures  x >= 8;
{
  call yield_x();  x := x + 1;
  call yield_x();  x := x + 1;
  call yield_x();  x := x + 1;
}
\end{verbatim}
\caption{Program~3}
\label{fig:ex3}
\end{figure}

{\bf Encoding rely-guarantee specifications.}
Figure~\ref{fig:ex3} shows Program~3, yet another variation of Programs~1 and~2 which shows how to encode a rely-guarantee-style~\cite{Jones83} (two-state invariant)
proof using \civl's one-state yield statements. 
The standard rely-guarantee specification to prove the assertions in {\tt p} is that the environment of {\tt p} 
may only increase {\tt x}.
We can encode this in \civl by factoring out the yield statement in a separate procedure
and then referring {\tt old(x)}, the value of {\tt x} when {\tt yield\_x} is
entered. 


\begin{figure}
\begin{verbatim}
type Tid;
const nil:Tid;
procedure Allocate() 
  returns (linear tid:Tid);
  ensures tid != nil;

var a:[Tid]int;

procedure main()
{
  while (true) {
    var linear tid:Tid := Allocate();
    async call P(tid);
    yield true;
  }
}
procedure P(linear tid: Tid)
  requires tid != nil;
  ensures a[tid] == old(a)[tid] + 1;
{
  var t:int := a[tid];
  yield t == a[tid];
  a[tid] := t + 1;
}
\end{verbatim}
\caption{Program 4}
\label{fig:ex5}
\end{figure}

\subsection{Linear variables}

Program~4 in Figure~\ref{fig:ex5} introduces linear variables, a feature of \civl 
that is useful for encoding disjointness among values contained in 
different variables.  
This example uses this feature for encoding the concept of an identifier 
that is unique to each thread.
Program~4 contains a shared global array {\tt a} indexed by an uninterpreted type {\tt Tid} 
representing the set of thread identifiers.
A collection of threads are executing procedure {\tt P} concurrently.
The identifier of the thread executing {\tt P} is passed in as the parameter {\tt tid}.
A thread with identifier {\tt tid} owns {\tt a[tid]} and can increment it without danger of interference.
The yield predicate {\tt t == a[tid]} in {\tt P} indicates this expectation, yet it is not possible to prove it 
unless the reasoning engine knows that the value of {\tt tid} in one thread is distinct 
its value in a different thread.

Instead of building a notion of thread identifiers into \civl, we provide a more 
primitive and general notion of linear variables.
The \civl type system ensures that values contained in linear variables cannot be duplicated.
Consequently, the parameter {\tt tid} of distinct concurrent calls to {\tt P} are known to be distinct;
the \civl verifier exploits this invariant while checking for non-interference.

\begin{figure}
\begin{verbatim}
type lmap = LMap(dom:[int]bool,
                 map:[int]int);
const empty:lmap;
axiom empty.dom == {};
\end{verbatim}
\begin{verbatim}
procedure Load(linear l:lmap, i:int) 
  returns(v:int);
  requires l.dom[i];
  ensures v == l.map[i];
\end{verbatim}
\begin{verbatim}
procedure Store(linear_in l_in:lmap,
                i:int, v:int) 
  returns(linear l:lmap);
  requires l_in.dom[i];
  ensures l.dom == l_in.dom;
  ensures l.map == l_in.map[i := v];
\end{verbatim}
\begin{verbatim}
procedure Move(linear_in l1_in:lmap) 
  returns(linear l1:lmap, linear l2:lmap);
  ensures l1 == empty && l2 == l1_in;
\end{verbatim}
\begin{verbatim}
procedure Transfer(linear_in l1_in:lmap, 
                   linear_in l2_in:lmap,
                   i:int) 
  returns(linear l1:lmap, linear l2:lmap);
  ensures l1.dom == l1_in.dom - {i};
  ensures l1.map == l1_in.map;
  ensures l2.dom == l2_in.dom + {i};
  ensures l2.map == l2_in.map[i:=l1.map[i]];
\end{verbatim}
\caption{Encoding linear maps}
\label{fig:linear-maps}
\end{figure}

\begin{figure}
\begin{verbatim}
var linear g:lmap;
var b:bool;
\end{verbatim}
\begin{verbatim}
procedure P()
{
  var t: int;
  var linear l:lmap;
  while (b) { call Yield(); }
                b := true;
                call g, l := Move(g);
  call Yield(); call t := Load(l, p);
  call Yield(); call l := Store(l, p, t+1);
  call Yield(); call t := Load(l, p+4);
  call Yield(); call l := Store(l, p+4, t+1);
  call Yield(); call l, g := Move(l);
                b := false;
  call Yield();
}
\end{verbatim}
\begin{verbatim}
procedure Yield() 
{
  yield b || (g.dom == {p,p+4} && 
              g.map[p] == g.map[p+4]);
}
\end{verbatim}
\caption{Program 6}
\label{fig:ex6}
\end{figure}

{\bf Linear maps.}
Figure~\ref{fig:linear-maps} illustrates how linear maps~\cite{LahiriQW11} can be encoded in \civl.
A linear map {\tt l} is a pair comprising {\tt l.map)}, an array of integers, 
and {\tt l.dom}, a set of integers representing the locations where it is legal to access {\tt l.map}.
In the parlance of separation logic, a linear map can be thought of as a collection of
{\em points-to} facts partitioned by separating conjunction; 
furthermore, there is an implicit separating conjuction between the values contained in two distinct linear maps.
Figure~\ref{fig:linear-maps} shows the primitive operations on linear maps.
(The notation {\tt linear\_in} indicates that a procedure consumes an argument,
making it unavailable to the caller after the call,
while a parameter marked {\tt linear} is only borrowed from the caller and remains available to the caller after the call.)
If a program is written using only these primitive operations, it is guaranteed all 
occurrences of {\tt Load} and {\tt Store} can be erased to loads and stores, respectively, 
to a single global memory.  
In addition, all occurrences of {\tt Move} and {\tt Transfer} can be erased completely.
Thus, linear maps provide a mechanism to program functionally yet execute imperatively.
At the same time, as is evident from the specifications of the primitive operations,
linear maps can be encoded using classical logic and verification of programs using them can benefit 
seamlessly from advances in first-order automated theorem proving.

Program~6 uses a linear map {\tt g} to represent two adjacent memory location starting at offset {\tt p}.
The variable {\tt g} is protected by a lock encoded with a boolean variable {\tt b}.
The procedure {\tt P} acquires the lock {\tt b} using atomic test-and-set, moves the content of {\tt g} into a local 
linear map {\tt l}, increments both memory locations in {\tt l}, and then moves {\tt l} back into {\tt g}.
The yield statement is encapsulated in the {\tt Yield} procedure; 
it amounts to a simple global invariant asserting that whenever the lock {\tt b} is not held, the domain of {\tt g} must
contain the two adjacent address {\tt p} and {\tt p+4} and the values stored in these addresses are identical.
This program captures the essence of programming with monitors~\cite{Hoare74} and demonstrates that this reasoning 
can be encoded easily in \civl.

{\bf Permissions.}
Because linear variables cannot be duplicated,
linear variables can safely express exclusive access to a resource, such as a section of memory.
Boyland~\cite{boyland:03fractions} generalized linear type systems with a notion of fractional permissions:
although linear variables still cannot be duplicated,
fractions of a linear resource may be shared among multiple linear variables,
as long as the fractions are all greater than 0 and always add up to 100\%.
For example, two variables could possess 50\% of a resource,
or one variable could possess 100\% of a resource.
The latter case (100\%) corresponds to the exclusive access in traditional linear type systems.
Fractional permissions are useful for distinguishing exclusive access,
such as exclusive read-write access to memory,
from shared access, such as shared read-only access to memory.

\civl's linear variables can encode fractional permissions.
As an example of this encoding,
Figure~\ref{fig:fracperm} extends Figure~\ref{fig:linear-maps}'s linear map type
with a fraction indicating partial access to the linear map.
For soundness, the {\tt Store} operation still requires exclusive access ({\tt fraction} = 100\%),
as in Figure~\ref{fig:linear-maps},
but {\tt Load} now allows shared access ({\tt fraction} may be anything),
since multiple concurrent loads do not interfere with each other.
The {\tt Split} procedure splits a single linear map into two subfractions,
while the {\tt Combine} procedures combines two fractions into a single linear map.

\begin{figure}
\begin{verbatim}
type lmap = LMap(dom:[int]bool,
                 map:[int]int,
                 fraction:real);

procedure Split(linear_in l:lmap, f1:real)
  returns(linear l1:lmap, linear l2:lmap)
  requires 0.0 < f1 && f1 < l.fraction;
  ensures l1.dom == l.dom && l1.map == l.map;
  ensures l2.dom == l.dom && l2.map == l.map;
  ensures l1.fraction == f1;
  ensures l2.fraction == l.fraction - f1;

procedure Combine(linear_in l1:lmap,
                  linear_in l2:lmap)
  returns (linear l:lmap)
  requires l1.dom == l2.dom;
  ensures l.dom == l1.dom && l.map == l1.map;
  ensures l.fraction == l1.fraction
                      + l2.fraction;

procedure Load(linear l:lmap, i:int) 
  ...

procedure Store(linear_in l_in:lmap,
                i:int, v:int) 
  ...
  requires l_in.fraction == 1.0;
  ...

procedure Move(linear_in l1_in:lmap) 
  ...

procedure Transfer(linear_in l1_in:lmap, 
                   linear_in l2_in:lmap,
                   i:int) 
  ...
  requires l1_in.fraction == l2_in.fraction;
  ...
\end{verbatim}
\caption{Fractional permissions}
\label{fig:fracperm}
\end{figure}

Rather than using real-number or rational-number fractions,
as in \cite{boyland:03fractions},
we can also use sets in place of fractions,
as shown in Figure~\ref{fig:subperm}.
Set union then replaces numerical addition,
and set difference replaces numerical subtraction.
The type of the set doesn't matter much;
Figure~\ref{fig:subperm} uses sets of real numbers.
With sets of real numbers,
we can use the real interval [0, 1) to indicate exclusive access,
in analogy to the 100\% fraction, and subintervals like [0, 0.5) and [0.5, 1)
to indicate shared access, in analogy to 50\% fractions.

This extra level of detail allows \civl to exploit disjointness of the subintervals:
we know that two threads can't both contain a linear map with the same domain and the same subinterval
(one thread might have subinterval [0, 0.5) and another [0.5, 1), but they can't both have [0, 0.5)).
Section~\ref{sec:gc-verify} describes how the GC verification uses this strategy
to transfer fractions of thread identifiers between threads and global variables
in order to aid verification of non-interference.

\begin{figure}
\begin{verbatim}
type lmap = LMap(dom:[int]bool,
                 map:[int]int,
                 fraction:[real]bool);

procedure Split(linear_in l:lmap, f1:set)
  returns(linear l1:lmap, linear l2:lmap)
  requires f1 != {} && f1 != l.fraction;
  requires f1 isSubsetOf l.fraction;
  ensures l1.dom == l.dom && l1.map == l.map;
  ensures l2.dom == l.dom && l2.map == l.map;
  ensures l1.fraction == f1;
  ensures l2.fraction == l.fraction - f1;

procedure Combine(linear_in l1:lmap,
                  linear_in l2:lmap)
  returns (linear l:lmap)
  requires l1.dom == l2.dom;
  ensures l.dom == l1.dom && l.map == l1.map;
  ensures l.fraction == l1.fraction
                      + l2.fraction;
\end{verbatim}
\caption{Subset permissions}
\label{fig:subperm}
\end{figure}
