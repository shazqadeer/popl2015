\section{Examples}
\label{sec:examples}

In this section, we present a collection of examples that illustrate
specification and verification features supported by the \civl verifier.
\subsection{Reasoning about interference}
\begin{figure}
\begin{verbatim}
var x:int;
\end{verbatim}
\begin{verbatim}
procedure p()
  requires x >= 5;
  ensures  x >= 8;
{
  yield x >= 5;  x := x + 1;
  yield x >= 6;  x := x + 1;
  yield x >= 7;  x := x + 1;
}
procedure q() { x := x + 3; }
\end{verbatim}
\caption{Program~1}
\label{fig:ex1}
\end{figure}

We present an overview of the \civl language and verifier through a sequence of examples.
Figure~\ref{fig:ex1} shows Program~1 containing a procedure {\tt p}
executing concurrently with another procedure {\tt q}. 
An execution of a \civl program is non-preemptive; a thread explicitly yields control to the
scheduler via the {\tt yield} statement following which execution continues on a 
nondeterministically chosen thread.
The {\tt yield} statement has an assertion $\varphi$ attached to it.
The yielding thread must establish $\varphi$ when it yields and the execution of other threads 
must preserve $\varphi$; these two requirements in Owicki-Gries-style
reasoning~\cite{OwickiG76} are usually known as {\em sequential correctness}
and {\em non-interference}, respectively.
To check these requirements, the \civl verifier creates verification conditions, whose number is at most
quadratic in the number of yield statements in the program.
For example, in Program~1 each yield predicate in {\tt p} must be checked against the action 
{\tt x := x + 3} in {\tt q}.

\begin{figure}
\begin{verbatim}
procedure yield_x(n:int)
  requires x >= n;
  ensures  x >= n;
{
  yield x >= n;
}
procedure p()
  requires x >= 5;
  ensures  x >= 8;
{
  call yield_x(5);  x := x + 1;
  call yield_x(6);  x := x + 1;
  call yield_x(7);  x := x + 1;
}
\end{verbatim}
\caption{Program~2}
\label{fig:ex2}
\end{figure}

{\bf From quadratic to linear verification conditions.}
Figure~\ref{fig:ex2} shows Program~2, a variation of Program~1 in which the procedure {\tt yield\_x} 
contains a single yield statement and {\tt p} calls {\tt yield\_x} instead of yielding directly.
If the calls to {\tt yield\_x} are inlined in Program~2, then we will get Program~1.
Both Program~1 and~2 are verifiable in \civl but the cost of verifying Program~2 is less because it has fewer yield statements.
In fact, if it is possible to capture all interference in a concurrent program in a single yield predicate, 
then the trick in Program~2 can be used to verify the program with a linear number of verification conditions.

\begin{figure}
\begin{verbatim}
procedure yield_x()
  ensures  x >= old(x);
{
  yield x >= old(x);
}
procedure p()
  requires x >= 5;
  ensures  x >= 8;
{
  call yield_x();  x := x + 1;
  call yield_x();  x := x + 1;
  call yield_x();  x := x + 1;
}
\end{verbatim}
\caption{Program~3}
\label{fig:ex3}
\end{figure}

{\bf Encoding rely-guarantee specifications.}
Figure~\ref{fig:ex3} shows Program~3, yet another variation of Programs~1 and~2 which shows how to encode a rely-guarantee-style~\cite{Jones83} (two-state invariant)
proof using \civl's one-state yield statements. 
The standard rely-guarantee specification to prove the assertions in {\tt p} is that the environment of {\tt p} 
may only increase {\tt x}.
We can encode this in \civl by factoring out the yield statement in a separate procedure
and then referring {\tt old(x)}, the value of {\tt x} when {\tt yield\_x} is
entered. 


\begin{figure}
\begin{verbatim}
type Tid;
const nil: Tid;
procedure Allocate() 
  returns (linear tid:Tid);
  ensures tid != nil;

var a:[Tid]int;

procedure main()
{
  while (true) {
    var linear tid:Tid := Allocate();
    async call P(tid);
    yield true;
  }
}
procedure P(linear tid: Tid)
  requires tid != nil;
  ensures a[tid] == old(a)[tid] + 1;
{
  var t:int := a[tid];
  yield t == a[tid];
  a[tid] := t + 1;
}
\end{verbatim}
\caption{Program 4}
\label{fig:ex5}
\end{figure}

\subsection{Linear variables}

Program~4 in Figure~\ref{fig:ex5} introduces linear variables, a feature of \civl 
that is useful for encoding disjointness among values contained in 
different variables.  
This example uses this feature for encoding the concept of an identifier 
that is unique to each thread.
Program~4 contains a shared global array {\tt a} indexed by an uninterpreted type {\tt Tid} 
representing the set of thread identifiers.
A collection of threads are executing procedure {\tt P} concurrently.
The identifier of the thread executing {\tt P} is passed in as the parameter {\tt tid}.
A thread with identifier {\tt tid} owns {\tt a[tid]} and can increment it without danger of interference.
The yield assertion {\tt t == a[tid]} in {\tt P} indicates this expectation, yet it is not possible to prove it 
unless the reasoning engine knows that the value of {\tt tid} in one thread is distinct 
its value in a different thread.

Instead of building a notion of thread identifiers into \civl, we provide a more 
primitive and general notion of linear variables.
The \civl type system ensures that values contained in linear variables cannot be duplicated.
Consequently, the parameter {\tt tid} of distinct concurrent calls to {\tt P} are known to be distinct;
the \civl verifier exploits this invariant while checking for non-interference.

\begin{figure}
\begin{verbatim}
type lmap;
function dom(lmap): [int]bool;
function map(lmap): [int]int;
const empty: lmap;
axiom domain(empty) == {};
\end{verbatim}
\begin{verbatim}
procedure Load(linear l_in:lmap, i:int) 
  returns(linear l:lmap, v:int);
  requires dom(l_in)[i];
  ensures l == l_in && v == map(l_in)[i];
\end{verbatim}
\begin{verbatim}
procedure Store(linear l_in:lmap, i:int, v:int) 
  returns(linear l:lmap);
  requires dom(l_in)[i];
  ensures dom(l) == dom(l_in) 
          && map(l) == map(l_in)[i := v];
\end{verbatim}
\begin{verbatim}
procedure Move(linear l1_in:lmap) 
  returns(linear l1:lmap, linear l2:lmap);
  ensures l1 == empty && l2 == l1_in;
\end{verbatim}
\begin{verbatim}
procedure Transfer(linear l1_in:lmap, 
                   linear l2_in:lmap, i:int) 
  returns(linear l1:lmap, linear l2:lmap);
  ensures dom(l1) == dom(l1_in) - {i};
  ensures map(l1) == map(l1_in);
  ensures dom(l2) == dom(l2_in) + {i};
  ensures map(l2) == map(l2_in)[i := map(l1)[i]];
\end{verbatim}
\caption{Encoding linear maps}
\label{fig:linear-maps}
\end{figure}

\begin{figure}
\begin{verbatim}
var linear g: lmap;
var b: bool;
\end{verbatim}
\begin{verbatim}
procedure P()
{
  var t: int;
  var linear l: lmap;
  while (b) { call Yield(); }
                b := true;
                call g, l := Move(g);
  call Yield(); call l, t := Load(l, p);
  call Yield(); call l := Store(l, p, t+1);
  call Yield(); call l, t := Load(l, p+4);
  call Yield(); call l := Store(l, p+4, t+1);
  call Yield(); call l, g := Move(l);
                b := false;
  call Yield();
}
\end{verbatim}
\begin{verbatim}
procedure Yield() 
{
  yield b || (dom(g) == {p,p+4} && 
              map(g)[p] == map(g)[p+4]);
}
\end{verbatim}
\caption{Program 6}
\label{fig:ex6}
\end{figure}

{\bf Linear maps.}
Figure~\ref{fig:linear-maps} illustrates how linear maps~\cite{LahiriQW11} can be encoded in \civl.
A linear map {\tt l} is a pair comprising {\tt map(l)}, an array of integers, 
and {\tt dom(l)}, a set of integers representing the locations where it is legal to access {\tt map(l)}.
In the parlance of separation logic, a linear map can be thought of as a collection of
{\em points-to} facts partitioned by separating conjunction; 
furthermore, there is an implicit separating conjuction between the values contained in two distinct linear maps.
Figure~\ref{fig:linear-maps} shows the primitive operations on linear maps.
If a program is written using only these primitive operations, it is guaranteed all 
occurrences of {\tt Load} and {\tt Store} can be erased to loads and stores, respectively, 
to a single global memory.  
In addition, all occurrences of {\tt Move} and {\tt Transfer} can be erased completely.
Thus, linear maps provide a mechanism to program functionally yet execute imperatively.
At the same time, as is evident from the specifications of the primitive operations,
linear maps can be encoded using classical logic and verification of programs using them can benefit 
seamlessly from advances in first-order automated theorem proving.

Program~6 uses a linear map {\tt g} to represent two adjacent memory location starting at offset {\tt p}.
The variable {\tt g} is protected by a lock encoded with a boolean variable {\tt b}.
The procedure {\tt P} acquires the lock {\tt b} using atomic test-and-set, moves the content of {\tt g} into a local 
linear map {\tt l}, increments both memory locations in {\tt l}, and then moves {\tt l} back into {\tt g}.
The yield assertion is encapsulated in the {\tt Yield} procedure; 
it amounts to a simple global invariant asserting that whenever the lock {\tt b} is not held, the domain of {\tt g} must
contain the two adjacent address {\tt p} and {\tt p+4} and the values stored in these addresses are identical.
This program captures the essence of programming with monitors~\cite{Hoare74} and demonstrates that this reasoning 
can be encoded easily in \civl.

{\bf Permissions.}


