\begin{figure}
\begin{verbatim}
var x:int;
\end{verbatim}
\begin{verbatim}
procedure p()
  requires x >= 5;
  ensures  x >= 8;
{
  yield x >= 5;  x := x + 1;
  yield x >= 6;  x := x + 1;
  yield x >= 7;  x := x + 1;
}
\end{verbatim}
\begin{verbatim}
procedure q() modifies x; { x := x + 3; }
\end{verbatim}
\caption{Program~1}
\label{fig:ex1}
\end{figure}

We present an overview of the \civl language and verifier through a sequence of examples.
Figure~\ref{fig:ex1} shows Program~1 containing a procedure {\tt p}
executing concurrently with another procedure {\tt q}. 
An execution of a \civl program is non-preemptive; a thread explicitly yields control to the
scheduler via the {\tt yield} statement following which execution continues on a 
nondeterministically chosen thread.
The {\tt yield} statement has an assertion $\varphi$ attached to it.
The yielding thread must establish $\varphi$ when it yields and the execution of other threads 
must preserve $\varphi$; these two requirements in Owicki-Gries-style
reasoning~\cite{OwickiG76} are usually known as {\em sequential correctness}
and {\em non-interference}, respectively.
To check these requirements, the \civl verifier creates verification conditions, whose number is at most
quadratic in the number of yield statements in the program.
For example, in Program~1 each yield predicate in {\tt p} must be checked against the action 
{\tt x := x + 3} in {\tt q}.

%modifies

\begin{figure}
\begin{verbatim}
var x:int;
\end{verbatim}
\begin{verbatim}
procedure yield_x(n:int)
  requires x >= n;
  ensures  x >= n;
{
  yield x >= n;
}
\end{verbatim}
\begin{verbatim}
procedure p()
  requires x >= 5;
  ensures  x >= 8;
{
  call yield_x(5);  x := x + 1;
  call yield_x(6);  x := x + 1;
  call yield_x(7);  x := x + 1;
}
\end{verbatim}
\caption{Program~2}
\label{fig:ex2}
\end{figure}

{\bf From quadratic to linear verification conditions.}
Figure~\ref{fig:ex2} shows Program~2, a variation of Program~1 in which the procedure {\tt yield\_x} 
contains a single yield statement and {\tt p} calls {\tt yield\_x} instead of yielding directly.
If the calls to {\tt yield\_x} are inlined in Program~2, then we will get Program~1.
Both Program~1 and~2 are verifiable in \civl but the cost of verifying Program~2 is less because it has fewer yield statements.
In fact, if it is possible to capture all interference in a concurrent program in a single yield predicate, 
then the trick in Program~2 can be used to verify the program with a linear number of verification conditions.

\begin{figure}
\begin{verbatim}
procedure yield_x()
  ensures  x >= old(x);
{
  yield x >= old(x);
}
\end{verbatim}
\begin{verbatim}
procedure p()
  requires x >= 5;
  ensures  x >= 8;
{
  call yield_x();  x := x + 1;
  call yield_x();  x := x + 1;
  call yield_x();  x := x + 1;
}
\end{verbatim}
\caption{Program~3}
\label{fig:ex3}
\end{figure}

{\bf Encoding rely-guarantee specifications.}
Figure~\ref{fig:ex3} shows Program~3, yet another variation of Programs~1 and~2 which shows how to encode a rely-guarantee-style~\cite{Jones83} (two-state invariant)
proof using \civl's one-state yield statements. 
The standard rely-guarantee specification to prove the assertions in {\tt p} is that the environment of {\tt p} 
may only increase {\tt x}.
We can encode this in \civl by factoring out the yield statement in a separate procedure
and then referring {\tt old(x)}, the value of {\tt x} when {\tt yield\_x} is
entered. 


\begin{figure}
\begin{verbatim}
type Tid;
var linear alloc:[Tid]bool;
const nil: Tid;
procedure Allocate() returns (linear tid:Tid);
  modifies alloc;
  ensures tid != nil;
\end{verbatim}
\begin{verbatim}
var a:[Tid]int;
\end{verbatim}
\begin{verbatim}
procedure main()
{
  while (true) {
    var linear tid:Tid := Allocate();
    async call P(tid);
    yield true;
  }
}
\end{verbatim}
\begin{verbatim}
procedure P(linear tid: Tid)
  requires tid != nil;
  ensures a[tid] == old(a)[tid] + 1;
{
  var t:int := a[tid];
  yield t == a[tid];
  a[tid] := t + 1;
}
\end{verbatim}
\caption{Program 4}
\label{fig:ex5}
\end{figure}

{\bf Linear variables.}
Program~4 in Figure~\ref{fig:ex5} introduces linear variables, a feature of \civl 
that is useful for encoding disjointness among values contained in 
different variables.  
This example uses this feature for encoding the concept of an identifier 
that is unique to each thread.
Program~4 contains a shared global array {\tt a} indexed by an uninterpreted type {\tt Tid} 
representing the set of thread identifiers.
A collection of threads are executing procedure {\tt P} concurrently.
The identifier of the thread executing {\tt P} is passed in as the parameter {\tt tid}.
A thread with identifier {\tt tid} owns {\tt a[tid]} and can increment it without danger of interference.
The yield assertion {\tt t == a[tid]} in {\tt P} indicates this expectation, yet it is not possible to prove it 
unless the reasoning engine knows that the value of {\tt tid} in one thread is distinct 
its value in a different thread.

Instead of building a notion of thread identifiers into \civl, we provide a more 
primitive and general notion of linear variables.
The \civl type system ensures that values contained in linear variables cannot be duplicated.
Consequently, the parameter {\tt tid} of distinct concurrent calls to {\tt P} are known to be distinct;
the \civl verifier exploits this invariant while checking for non-interference.

\begin{figure}
\begin{verbatim}
var Color:[int]int; // WHITE=1, GRAY=2, BLACK=3
\end{verbatim}
\begin{verbatim}
procedure TopWB(linear tid:Tid, addr:int)
  atomic [if (Color[addr] == WHITE) { 
              Color[addr] := GRAY;  }];
{
  var cNoLock:int := GetColorNoLock();
  call YieldColorsGetDarker(); 
  if (cNoLock == WHITE) 
    call MidWB(tid, addr);
}
\end{verbatim}
\begin{verbatim}
procedure MidWB(linear tid:Tid, addr:int)
atomic [if (Color[addr] == WHITE) {
            Color[addr] := GRAY;  }];
{
  call AcquireLock(tid);
  var cLock:int := GetColorLocked(tid);
  if (cLock == WHITE) 
    call SetColorLocked(tid, GRAY);
  call ReleaseLock(tid);
}
\end{verbatim}
\begin{verbatim}
procedure YieldColorsGetDarker()
  ensures forall a:: Color[a] >= old(Color[a]);
{
  yield forall a:: Color[a] >= old(Color[a]);
}
\end{verbatim}
\begin{verbatim}
procedure AcquireLock() right [...];
procedure ReleaseLock() left [...];
procedure SetColorLocked() atomic [...];
procedure GetColorLocked() returns (cl:int)
  both [...];
\end{verbatim}
\caption{Program 5}
\label{fig:reft}
\end{figure}

{\bf Refinement.} 
\civl supports the verification of refinement of atomic action specifications for procedures. 
We illustrate this technique on Program 5 (Figure~\ref{fig:reft}),
a simplified version of a programming pattern in our concurrent garbage collector (GC).
Mutator threads entering a write barrier for an address \exC{addr} check
if the global variable \exC{Color[addr]} is \exC{WHITE}
and set it to \exC{GRAY}, indicating that the object at this address
and objects reachable from it should not be garbage collected. 

Procedure \exC{TopWB} implements the write barrier.
To avoid the cost of locking for each address encountered, \exC{TopWB} reads \exC{Color[addr]} without holding a lock.
If the color is {\tt WHITE}, it calls the more expensive procedure \exC{MidWB} 
that re-examines and possibly updates \exC{Color[addr]} while holding the lock.
\civl simplifies reasoning about \exC{TopWB} and \exC{MidWB} by allowing us to 
compactly express their specification as the following atomic action:
\begin{verbatim}
[if (Color[addr] == WHITE) Color[addr] := GRAY;]
\end{verbatim}
This specification indicates that regardless of the different implementations of 
\exC{TopWB} and \exC{MidWB} and regardless of how the environment interferes
with their execution, to their respective callers it appears as if they atomically execute the above code.

The correctness of \exC{topWB} is not obvious.
Consider the following potential scenario. 
\exC{TopWB}, not holding a lock, reads {\tt Color[addr]} and
sets \exC{cNoLock} to \exC{GRAY} and then yields inside the call to
\exC{YieldColorsGetDarker}. Another thread sets {\tt Color[addr]} to
{\tt WHITE}. \exC{TopWB} resumes, but does nothing and exits,
because the procedure-local variable \exC{cNoLock} is \exC{GRAY}. In this scenario, the atomic action
specification of \exC{TopWB} would not be satisfied. However, in the GC this
scenario is not possible. 
The yield predicate in \exC{YieldColorsGetDarker} expresses the fact that
other threads in the environment of the thread running \exC{TopWB} can
only modify {\tt Color[addr]} to a higher (darker) value. 
\civl then verifies that, for every control path through \exC{TopWB}
procedure, exactly one {\tt yield}-to-{\tt yield} execution
fragment implements the atomic action specification and that other fragments do not modify
global state. 
This proof requires both correct modeling of environment interference in \exC{YieldColorsGetDarker}
and the atomic action specification for the called procedure \exC{MidWB}.

There are two important benefits of atomic action specifications.
First, an atomic action is often the most compact and precise way to express the specification a procedure.
Second, an atomic action specification successfully hides from the caller of a procedure $P$
the potentially numerous yields inside the body of $P$.
This capability provides modularity akin to that provided by the frame rule for sequential programs.
Without this capability, any postcondition $\phi$ needed by the caller of $P$ must be made explicit in the precondition 
and yield predicates of $P$ and all procedures recursively called inside it.
With this capability, the caller achieves the same effect by writing $\yield{\phi}$ before and after the call, 
provided that $\phi$ is preserved by the atomic action specification of $P$.

{\bf Yield elimination.} 
We now turn our attention to the verification of \exC{MidWB} against
its atomic action specification.
Just as the verification of \exC{TopWB} builds on the specification of \exC{MidWB},
the verification of \exC{MidWB} builds on other refinement proofs (not shown) 
of the procedures called in \exC{MidWB};
these called procedures are shown at the bottom of the figure. 
Observe that in \exC{MidWB} there is no {\tt yield} between
statements. Since threads explicitly yield control, the
entire body of \exC{MidWB} is executed atomically. In
a real execution, control can switch between threads at any point in
the code. The absence of {\tt yield}s is justified by reasoning about
mover types and reduction. The procedures called in
\exC{MidWB} have the mover types claimed in their
declarations and verified by \civl. 
For example, the mover type of \exC{Acquirelock} is {\tt right} which indicates 
that it commutes later in time against concurrently executing environment actions.
\exC{ReleaseLock} has mover type {\tt left} and it commutes earlier in time;
\exC{SetColorLocked} has mover type {\tt atomic} and it does not commute;
\exC{GetColorLocked} has mover type {\tt both} and it commutes both earlier and later in time.
Given the mover types of 
atomic action specifications, the \civl verifier ensures
that the absence of yields is justified using reduction~\cite{Lipton75}.  
\civl then verifies that the body of \exC{MidWB}, without any yield statements, satisfies
its atomic specification. Yield elimination is useful both for the
user, since no yield predicates need to be provided, and for \civl,
as no predicates need to be checked for sequential correctness and
non-interference.

{\bf Variable hiding.} The atomic action specification of \exC{MidWB}
makes no reference to the lock variable, although its implementation
involves a lock. When verifying refinement for \exC{MidWB}, the lock
variable has been hidden. Hiding variables at refinement steps and
providing different {\tt yield} predicates at different phases of a
proof are important novel proof organization capabilities in \civl
that facilitate proofs spanning large abstraction gaps.

{\bf Putting it all together.} 
Refinement reasoning requires powerful and
flexible mechanisms for reasoning with invariants.  
As illustrated in Program~5, refinement verification for \exC{TopWB} required 
(1)~rely-guarantee reasoning using a stable yield predicate, and 
(2)~verification of atomicity for \exC{MidWB}, which, in turn, required reduction. 
The combination of features in \civl, such as assertion reasoning, linear variables, and yield elimination,
is designed to support the goal of organizing a
refinement proof in an efficient and natural way.
The GC verification effort, described in Section~\ref{sec:gc}, provides a compelling demonstration of \civl in action.


%To force commit.

%%% Local Variables: 
%%% mode: latex
%%% TeX-master: "paper"
%%% End: 
