\section{Experience}
\label{sec:experience}

The \civl verifier has been under development for around two years.  
Over that period, we have developed a collection of 32 benchmarks, 
ranging in size from 17 to 539 LOC, to illustrate various features of
\civl and for regression testing as we evolved the verifier.
In addition to microbenchmarks, this collection also includes
standard benchmarks from the literature such as a multiset implementation~\cite{ElmasTQ05}, 
the ticket algorithm~\cite{FarzanKP14}, 
Treiber stack~\cite{Herlihy2008}, work-stealing queue~\cite{Blumofe1999},
device cache~\cite{ElmasQT09}, and lock-protected increment~\cite{FlanaganQ03}. 
The \civl verifier is fast; the entire benchmark set verifies in 20 seconds on a standard 4-core Windows PC 
with no benchmark requiring more than a few seconds.

\subsection{Garbage collector}
We have used \civl to design and verify a realistic concurrent garbage collection (GC) algorithm.  
In particular, although our algorithm is based on an earlier algorithm by Dijkstra et al~\cite{dijk78}, 
it extends the earlier algorithm with various modern optimizations and embellishments to improve generality and performance.  
These extensions include lower write barrier overhead, phase-based synchronization and handshaking, 
and coordination between the GC and mutator threads during root scanning; our use of linearity aids the proof of root scanning, 
while our rely-guarantee encoding aids management of colors inside the write barrier.  
Furthermore, our encoding of the algorithm in \civl spans a wide range of abstraction, 
from low-level memory operations all the way up to high-level specifications; 
we used six levels of refinement to help hide low-level details from the high-level portions of the verification.
We believe that \civl's combination of features makes practical, for the first time, verification across such a wide range of abstraction.

The GC verification takes 60 seconds on the same 4-core Windows PC, generating and verifying 667 proof obligations. 
The bulk of this time, 54 seconds, is taken by the verification of the refinement checks from Section~\ref{sec:refinement}.
The linear type checking, the yield safety checks, and the commutativity checks take the rest of the time and are insignificant in comparison.

\subsection{Discussion of the proof}
We now put atomicity refinement techniques from the literature and
\civl in context by presenting an overview of our design
driver, the stepwise refinement of a garbage collector~\cite{gc-techreport}.
The refinement proof spans six levels of abstraction. 
Each of refinement proof relating two consecutive levels is made feasible by a different
blend of the techniques in \civl. 
% While other refinement techniques have also used garbage collectors as
% case studies, the refinement tasks tackled there bridge only one or two of the levels in 
% our refinement proof\footnote{A more specific discussion of this point
%   can be found in the technical report on the verification of the
%   garbage collector\cite{gc-techreport}.} and only target the refinement verification
% challenges apparent at those levels of the proof. 

The topmost-level description of the garbage collector provides an
idealized, abstract view of memory. 
At this level, none of the lowest-level implementation variables are
visible -- variable hiding has been used to project them away. 
In the top few levels of the garbage collector proof, invariant-based
non-interference reasoning was our primary tool, while reduction
simplified verification by enabling us to use coarser atomic actions and fewer
location invariants.  
Linear variables were used throughout the proof to model the distinct
thread identifiers for the garbage collector thread and mutator
threads, but were most instrumental in encoding single-threaded
execution in the initialization phase of the program. 
For these top few levels of our proof, rely-guarantee and separation-logic-based
approaches would have also performed well, as demonstrated by the
garbage collector proof of Liang et al.~\cite{LiangRGSim}, where
the atomicity of actions in the lower levels our proof is {\em assumed} but not verified.
An important distinguishing capability in \civl is being able to use location invariants rather than pure rely-guarantee reasoning.
This helped interactive proof at the top levels significantly.
For the mark phase of the garbage collector, we made critical use of
different invariants at different locations in procedure bodies. 
While the same non-interference argument could have been encoded in
rely-guarantee reasoning, as we had done ourselves in an earlier
version of our proof, 
it would have required the use of several additional auxiliary shared variables. 
Invariants, rely and guarantee conditions referring to such auxiliary
variables throughout the program made interactive invariant reasoning more difficult to manage. 

In the lower levels of the garbage collector proof, where
correctness of concurrent data structures and synchronization primitives were proven, we made
relatively little use of location invariants, and made heavier use of
linear variables and reduction. 
We also used variable hiding heavily to hide low-level implementation
variables. 
For lower-level refinement tasks, for instance, when verifying the correctness of a
lock-protected concurrently-accessed stack, ownership
arguments, separation logic, or \QED-style atomicity would have been
sufficient. 
But, at the higher levels of our proof, where non-interference
reasoning via invariants and linear variables was indispensable, 
atomicity alone, or ownership or separation logic arguments alone
would have run into difficulty. 

While existing techniques in the literature have as their
``sweet spot'' a few of the refinement proofs in our garbage collector
proof, they run into difficulty in others. 
More critically, they
do not facilitate layering refinement proofs, which is required for stepwise
refinement. 
Using a realistic top-down proof as \civl's design driver led us to
combine in one tool and consistent theory, the verification techniques
of linearity, reduction and non-interference reasoning in the service
of a modular refinement proof directed by the syntactic structure of
the imperative concurrent program. 

%%% SHAZ, I REMOVED THIS SECONDARY NOVELTY POINT, SINCE IT IS MADE
%%% ELSEWHERE IN THE PAPER TOO.
% For this purpose, we also devised
% novel ways to combine automata-theoretic and assertion-based
% verification, and encode the component techniques, e.g., linearity, in
% assertion-based verification.  
% %%%%%%%%% SHAZ PLEASE REFINE OR REMOVE THIS FINAL SENTENCE %%%%%%%%%%%
