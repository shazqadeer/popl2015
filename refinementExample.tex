\begin{figure}
\begin{small}
\begin{verbatim}
procedure Alloc() returns addr:int
  ensures {:atomic} {assume Mem[addr] == free;  
                     Mem[addr] := taken;}
{
  Reserve();         
  ptr := lo;
  while (ptr < hi && !found)
    invariant LoopInv(tid, addr, freeSpace, AtOrAfter);
  { 
    found := AllocAddrIfFree(addr);
    if (!found) 
      addr := addr + 1;
  }
}
\end{verbatim}
\begin{verbatim}
procedure Free(addr:int)
  ensures {:atomic} {Mem[addr] := free;}
{
  FreeAddr(addr);  
  IncFreeSpace(); 
}
\end{verbatim}
\begin{verbatim}
procedure AllocAddrIfFree(addr:int) returns res:bool
  ensures {:atomic} { res := false;
                      if (Mem[addr] == free) {
                         Mem[addr] := taken; 
                         res := true;
                      }  
                    }
{
  acquire(allocLock);
  if (Mem[addr] == free)
    Mem[addr] := taken;   
  release(allocLock);
}
\end{verbatim}
\begin{verbatim}
procedure FreeAddr(addr)
  ensures {:atomic} {Mem[addr] := free;}
{
  acquire(allocLock);
  Mem[addr] := free;   
  release(allocLock);
}
\end{verbatim}
\begin{verbatim}
procedure Reserve() 
  ensures {:atomic} {assume freeSpace >= 1; 
                     freeSpace := freeSpace - 1;
                     AtOrAfter[tid][lo] := true;}
{
  while (true) {
    tmp := freeSpace;
    if (tmp>0 && CAS(freeSpace,tmp,tmp-1))  break;
  }
}
\end{verbatim}
\begin{verbatim}
procedure IncFreeSpace() 
  ensures {:left} {freeSpace := freeSpace + 1;}
{
  while (true) {
    tmp := freeSpace;
    if (CAS(freeSpace, tmp, tmp+1))  break;
  }
}
\end{verbatim}
\begin{verbatim}
LoopInv(tid, addr, freeSpace, AtOrAfter) = 
  0 <= freeSpace && lo <= addr <= hi &&
  Size(AtOrAfter[lo]) + freeSpace == #Free[lo,hi] && 
  (forall u,v :: lo <= u <= v <= hi ==> 
         Subset(AtOrAfter[v], AtOrAfter[u])) &&
  (forall u :: 
         Size(AtOrAfter[u]) <= #Free[u,hi]) &&
         AtOrAfter[u][tid] <==> lo <= u && u <= addr)
\end{verbatim}
\end{small}
\caption{An example of multiple phases of refinement}
\label{fig:refinement}
\end{figure}

The \exC{Alloc} procedure is possibly blocking and returns when it
finds an address \exC{addr} such that \exC{Mem[addr]} is
\exC{free}. \exC{Free(addr)} simply frees the address
\exC{addr}. 
Before \exC{Alloc} scans the range of addresses
\exC{[lo,hi]}, it calls \exC{Reserve}, which continuously checks if
\exC{freeSpace > 0} and tries to atomically decrement \exC{freeSpace}
and reserves a \exC{free} address for later allocation by
\exC{Alloc}'s \exC{while} loop. It is non-trivial to verify that this
obviously-terminating \exC{while} loop succeeds in finding a
\exC{free} address. The variable \exC{freeSpace} keeps track of the
number of \exC{free} addresses that have not been reserved in this
manner. 

This example makes use of refinement at several levels. At the lowest
level, the atomic action specifications of \exC{Reserve} and
\exC{IncrementFreeSpace}, and the fact that the latter is a left mover
are verified. This is sufficient to show that \exC{Free} is
atomic. Verifying refinement and atomicity for \exC{Alloc} makes use
of the fact that \exC{Reserve} is atomic. To prove that \exC{Alloc}
satisfies its atomic action specification, we make use of an auxiliary
variable \exC{AtOrAfter}. \exC{AtOrAfter[u][tid] == true} iff thread
\exC{tid} has successfully completed executing \exC{Reserve} but has
not returned, and its local variable \exC{addr >= u}. We also make use
of the function \exC{\#Free} where \exC{\#Free[u,v]} returns the number
of addresses \exC{addr} in the interval \exC{[u,v]} such that
\exC{Mem[addr] == free}. 
\exC{Reserve}
establishes the following important facts when \exC{Alloc}'s
\exC{while} loop is about to be entered: \exC{AtOrAfter[lo][tid]} and 
\begin{verbatim}
  Size(AtOrAfter[lo]) + freeSpace == #Free[lo,hi] 
\end{verbatim}


%
%
%
%
%Text:
%   (forall u: int :: memAddr(u) || NumFreeAtOrAfter[u] == 0) &&
%    (forall u: int :: {memAddr(u)} memAddr(u) ==> NumFreeAtOrAfter[u] == (NumFreeAtOrAfter[u+1] + (if Free[u] then 1 else 0)))
%
% Invariant in text

%%% Local Variables: 
%%% mode: latex
%%% TeX-master: "paper"
%%% End: 

