We have chosen to demonstrate the proposed verification methodology and tool on a realistic modern concurrent garbage collector. To this end, we designed a garbage collector that extends the concurrent collector of Dijkstra et. al. \cite{dijk78}. The goal in this design is to get a collector that is on one hand easy to verify and on the other hand highly performant in practice. Most modern concurrent collectors are snapshot-oriented  \cite{doli93,doli94,doma00,azat03}, and as such may require complex claims on snapshot times and reachability. Other popular concurrent collectors and in particular the {\em mostly concurrent} garbage collector \cite{boeh91,prin00a,bara05} consist of different complex phases and complex interaction between objects that reside on specific {\em cards}. We preferred a collector whose invariants are simple and hold continuously as much as possible throughout the execution. This means that a program execution can move from one thread to another and then to the garbage collector while all relevant invariants continue to hold at all times. 

Dijkstra's collector seems to be a good candidate, but it cannot be considered a modern performant collector. On the positive side, its write-barrier maintains simple invariants continuously. However, this collector becomes incorrect in the presence of more than one program thread (mutator) and it requires activating the write-barrier on writes to root pointers, which means write barrier overhead on the runtime stack accesses and register accesses. The first issue means that the collector is unacceptable for use with modern multicore platforms. The second issue implies bad performance. Concurrent garbage collectors today are expected to work correctly without adding a write-barrier overhead to local accesses. 

We therefore extended and modified Dijkstra's collector to make it work with parallel programs and also not require applying a write-barrier on root modifications. As far as we know, this garbage collector's algorithm has not been proposes previously in the literature.  



