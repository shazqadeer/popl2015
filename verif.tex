\subsection{Verification}
\label{sec:verification}

Section~\ref{sec:overview} briefly described the verification process for \civl, based on generating verification conditions for sequential correctness and non-interference.
If a program is well-typed and all verification conditions are satisfied, then the program is guaranteed to run without getting stuck (in particular, with no assertion failures).

To avoid introducing a separate formal language for sequential programs, we reuse the syntax from Figure~\ref{fig:syntax} to represent sequential programs, where each sequential program contains exactly one thread.
The generated sequential programs only use ordinary sequential statements:
$\skipstmt$, $w := \locExpr$, $\assert{}$, $\call{}$, $\ablock{}{}$, $\ite{}{}{}$, and $\while{}{}{}$;
they do not use the concurrency-related statements $\yield$ and $\async{}$, .

Sequential programs make use of additional statements  $\assume{e}$ and $\havoc{\vec{x}}$ with the semantics that $\assume{e}$ steps to $\skipstmt$ only if the program state satisfies $e$, and $\havoc{\vec{x}}$ steps to $\skipstmt$ and non-deterministically updates the values of the variables in $\vec{x}$. 

In addition, the generated sequential programs use only ordinary variables, never linear variables.
As a result of these properties, the sequential programs can be verified with off-the-shelf automated sequential verification tools like Boogie, as described further in Section~\ref{sec:civl}.
(In practice, a sequential verification tool requires hints in the form of loop annotations, preconditions, and postconditions.
We omit such hints in the formalization; Section~\ref{sec:civl} describes how the real \civl tool translates these annotations from the concurrent program into the sequential programs.)


\subsection{Verifying Sequential Correctness}
For $m \in \Vstore$, let $ \vec{x}(m) = \{ x  |   x : t = v \in m \} $. 


The $\yield{e}$ and $\ablock{e,r}{s}$ annotations must have the following property during the execution of a thread. Starting from any state satisfying the $\yield{e_1}$ statement, all $e$s in atomic blocks $\ablock{e,r}{s}$ executed until the next yield statement $\yield{e_2}$ is reached, and $e_2$ must be satisfied. All of these checks can be encoded into a single sequential program by modifying the original program such that every $\yield{e}$ statement is replaced with one that sets the state non-deterministically to any state that satisfies $e$. More precisely, given the set of procedure names $\ProcName$ and $\bodies$, we verify the sequential program $\SeqVer(\bodies)$.   

$ \SeqVer(s_1; s_2) = \SeqVer(s_1); \SeqVer(s_2)$


\subsubsection{Verifying Non-Interference}
For a yielding thread $Y[l, \yield{e}]$, the modification of state performed by another thread $T$ as it steps from one $yield$ statement to another must preserve $e$. To facilitate modular checking, we instead check a sufficient condition for this. Instead of verifying that the entire sequence of statements executed by $T$ from one $yield$ statement to another, we check that all atomic blocks $\ablock{e,r}{s}$ in the program preserve $e$.

For every $(l,\yield{e_1}) \in \Yields l \bodies$ and $\ablock{e_2,r}{s} \in \Ablocks l$, we verify that $e$ remains preserved under 
$$\ablock{e_2[\vec{x} \rightarrow \vec{x'}],  r}{s[\vec{x} \rightarrow \vec{x'}]}$$
This can be expressed as the Hoare triple 
$$\{ e_1 \vee e_2[\vec{x} \rightarrow \vec{x'}] \} \; s[\vec{x} \rightarrow \vec{x'}] \; \{ e_1 \}$$
Hoare triples can be verified by sequential verification tools in a straightforward manner. 

\subsubsection{Verifying Refinement}
Consider an abstract-atomic procedure $P_{A}$ with an atomic action specification $(m, \tau, \alpha)$, i.e., $$P\ \mathit{modifies}\ \overrightarrow{w}\ \mathit{requires}\ \phi\ \mathit{ensures}\ \beta\ \mathit{refines}\ (m, \tau, \alpha)$$
In this section, we describe the verification conditions generated to verify that $\bodies(P)$ is a correct refinement. Intuitively speaking, we generate verification checks that ensure that every sequence of state transitions performed by $\bodies(P)$ refines an execution of the {\em specification automaton} in Figure~\ref{fig:refChkAut}. In a concurrent execution, state transitions performed by a particular execution instance of the statement $\bodies{P}$ are interleaved with actions performed by other threads. This fact is taken into account in the checks below in the atomic block expressions $e$ labeling atomic blocks $\ablock{e,r}{s}$. The refinement check type $r$ indicates which state transition in the {\em specification automaton} the atomic block $\ablock{e,r}{s}$ maps to. In our tool implementation, the $r$ annotation is not needed. It is included in our formalization for ease of exposition. 

%% Draw two-state automaton with state and edge labels. 

{\bf Explain:} Thus, the check performed is not only thread-modular, but also procedure-modular. 

In the following, $\hiddenVars$ represents the variables hidden.

$$
\begin{array}{l}
\Ablocks(bs)   =   \bigcup_{P \in \ProcName} \Ablocks(bs, \bodies(P)) \\
\Ablocks(bs, \procName, s_1; s_2)   =  \Ablocks(bs, \procName, s_1) \cup 
                                                                                         \Ablocks(bs, \procName, s_2) \\
 \Ablocks(bs, \procName, \while{\locExpr}{e}{s} )   =   \Ablocks(bs,\procName,  s) \\
\Ablocks(bs, \procName, \termwhile{\locExpr}{e}{s} )   =   \Ablocks(bs, \procName,  s) \\
\Ablocks(bs, \procName, \ite{\locExpr}{s_1}{s_2})   =   \\
\hspace*{2.5cm} \Ablocks(bs, \procName, s_1) \cup \Ablocks(bs, \procName, s_2) \\
\Ablocks(bs, \procName, \call{P'})   =   \Ablocks(bs, P', \bodies(P') ) \\ 
\Ablocks(bs, \procName, \async{P'})   =   \Ablocks(bs, P', \bodies(P') ) \\ 
\Ablocks(bs, \procName \ablock{e,r}{s})   =   \{(\procName, \ablock{e,r}{s}) \} \\ 
\Ablocks(bs, \procName, \skipstmt)   =   \emptyset \\
\Ablocks(bs, \procName, \assert{le})  =   \emptyset \\
\Ablocks(bs, \procName, w := \locExpr)   =   \emptyset \\
\Ablocks(bs, \procName, \call{A})   =   \emptyset \\
\end{array}
$$

For two statements $s_1$ and $s_2$, we say $s_1$ {\em refines} $s_2$
{\em subject to $e$}, written $\impl{e}{s_1}{s_2}$, iff for each global
stores $\varsG$ and $\varsG'$ and local stores $\varsL$ and $\varsL'$,
if $\varsG \cdot \varsL \vdash e$ and $(\varsG \cdot \varsL, s_1)
\trans (\varsG' \cdot \varsL', s_1')$ for some $s_1'$, then $(\varsG \cdot \varsL, s_2)
\trans (\varsG' \cdot \varsL', s_2')$ for some $s_2'$. Whether $\impl{e}{s_1}{s_2}$ can be
expressed in a straightforward manner a sequential verification
problem.

For each $(P_{A}, \ablock{e,r}{s}) \in \Ablocks(\Prog)$, we perform a refinement check as described below. Here 
$$P_{A}\ \mathit{modifies}\
\vec{w}\ \mathit{requires}\ \phi\ \mathit{ensures}\ \beta\
\mathit{refines}\ (m, \tau, \alpha)$$
\begin{itemize}
\item If $r = \mathit{act}$, then this atomic block is meant to refine the atomic specification of the procedure it belongs to. We then check that 
(i) $e \implies \tau$ and 
(ii) $\impl{e}{s}{\alpha}$
\item If $r = \mathit{pre}$, then this atomic block maps to a self loop before the atomic specification action and we check $\impl{e}{s}{\mathit{havoc }\ \vec{w}, \hiddenVars}$. 
\item If $r = \mathit{post}$, then this atomic block maps to a self loop before the atomic specification action and we check $\impl{e}{s}{\mathit{havoc }\ \hiddenVars}$. 
\end{itemize}

\subsubsection{Yield Type Checking}

{\bf Termination Assumption:}
% For a statement s, let s' denote the recursive computation that substitutes each occurrence of
% 1. yield e and call P by assume false,
% 2. async P by skip
% 3. ablock {e} s by s'
% 4. twhile {I} e s by while {I} e s'
% 5. assert le by assume le
% 5. usual recursion for everything else

\noindent
$\Visit(s_1; s_2) = \\ \hspace*{0.7cm} (\Visit(s_1).\stm; \Visit(s_2).\stm, \Visit(s_1).\coll \cup \Visit(s_2).\coll)$\\
$\Visit(\ite{e}{s_1}{ s_2}) = $  \\
\hspace*{1.3cm}$( \ite{e}{\Visit(s_1).\stm}{\Visit(s_2).\stm}, \\
\hspace*{3.3cm}\Visit(s_1).\coll \cup \Visit(s_2).\coll)$\\
$\Visit(\termwhile{l}{e}{s}) = ( \while{l}{e}{\Visit(s).\stm}, \{\Visit(s).\stm\} )$\\
$\Visit(\while{l}{e}{s}) = ( \while{l}{e}{\Visit(s).\stm}, \emptyset)$\\
$\Visit(\assert{le}) = (\assume{le}, \emptyset)$\\
$\Visit(\yield{e}) = (\assume{false}, \emptyset)$\\
$\Visit(\call{P}) = (\assume{false}, \emptyset)$\\


Then we require that the s' collected by applying the above to each occurrence of twhile {I} e s is terminating starting from a state satisfying I.  The requirement that we know nothing about the start state except that it satisfies the loop invariant is important.

You can say informally what it means for a while loop annotated with a loop invariant to be terminating (there is a measure function blah blah) and then invoke this definition for each s' collected as above.  Note that the s' collected above are sequential code fragments.  

Note also that there is redundancy when twhile loops are nested.  In that case, you really just need to do it for the top-level but you can just ignore this redundancy.


%%% Local Variables: 
%%% mode: latex
%%% TeX-master: "paper"
%%% End: 

