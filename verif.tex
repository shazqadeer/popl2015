\subsection{Verification}

Consider an abstract-atomic procedure $P_{A}$ with an atomic action specification $(m, \tau, \alpha)$, i.e., $$P\ \mathit{modifies}\ \overrightarrow{w}\ \mathit{requires}\ \phi\ \mathit{ensures}\ \beta\ \mathit{refines}\ (m, \tau, \alpha)$$
In this section, we describe the verification conditions generated to verify that $\bodies(P)$ is a correct refinement. Intuitively speaking, we generate verification checks that ensure that every sequence of state transitions performed by $\bodies(P)$ refines an execution of the {\em specification automaton} in Figure~\ref{fig:refChkAut}. In a concurrent execution, state transitions performed by a particular execution instance of the statement $\bodies{P}$ are interleaved with actions performed by other threads. This fact is taken into account in the checks below in the atomic block expressions $e$ labeling atomic blocks $\ablock{e,r}{s}$. The refinement check type $r$ indicates which state transition in the {\em specification automaton} the atomic block $\ablock{e,r}{s}$ maps to. In our tool implementation, the $r$ annotation is not needed. It is included in our formalization for ease of exposition. 

%% Draw two-state automaton with state and edge labels. 

{\bf Explain:} Thus, the check performed is not only thread-modular, but also procedure-modular. 

In the following, $\hiddenVars$ represents the variables hidden.

$$
\begin{array}{l}
\Ablocks(bs)   =   \bigcup_{P \in \ProcName} \Ablocks(bs, \bodies(P)) \\
\Ablocks(bs, \procName, s_1; s_2)   =  \Ablocks(bs, \procName, s_1) \cup 
                                                                                         \Ablocks(bs, \procName, s_2) \\
 \Ablocks(bs, \procName, \while{\locExpr}{e}{s} )   =   \Ablocks(bs,\procName,  s) \\
\Ablocks(bs, \procName, \termwhile{\locExpr}{e}{s} )   =   \Ablocks(bs, \procName,  s) \\
\Ablocks(bs, \procName, \ite{\locExpr}{s_1}{s_2})   =   \\
\hspace*{2.5cm} \Ablocks(bs, \procName, s_1) \cup \Ablocks(bs, \procName, s_2) \\
\Ablocks(bs, \procName, \call{P'})   =   \Ablocks(bs, P', \bodies(P') ) \\ 
\Ablocks(bs, \procName, \async{P'})   =   \Ablocks(bs, P', \bodies(P') ) \\ 
\Ablocks(bs, \procName \ablock{e,r}{s})   =   \{(\procName, \ablock{e,r}{s}) \} \\ 
\Ablocks(bs, \procName, \skipstmt)   =   \emptyset \\
\Ablocks(bs, \procName, \assert{le})  =   \emptyset \\
\Ablocks(bs, \procName, w := \locExpr)   =   \emptyset \\
\Ablocks(bs, \procName, \call{A})   =   \emptyset \\
\end{array}
$$

For two statements $s_1$ and $s_2$, we say $s_1$ {\em refines} $s_2$
{\em subject to $e$}, written $\impl{e}{s_1}{s_2}$, iff for each global
stores $\varsG$ and $\varsG'$ and local stores $\varsL$ and $\varsL'$,
if $\varsG \cdot \varsL \vdash e$ and $(\varsG \cdot \varsL, s_1)
\trans (\varsG' \cdot \varsL', s_1')$ for some $s_1'$, then $(\varsG \cdot \varsL, s_2)
\trans (\varsG' \cdot \varsL', s_2')$ for some $s_2'$. Whether $\impl{e}{s_1}{s_2}$ can be
expressed in a straightforward manner a sequential verification
problem.

For each $(P_{A}, \ablock{e,r}{s}) \in \Ablocks(\Prog)$, we perform a refinement check as described below. Here 
$$P_{A}\ \mathit{modifies}\
\vec{w}\ \mathit{requires}\ \phi\ \mathit{ensures}\ \beta\
\mathit{refines}\ (m, \tau, \alpha)$$
\begin{itemize}
\item If $r = \mathit{act}$, then this atomic block is meant to refine the atomic specification of the procedure it belongs to. We then check that 
(i) $e \implies \tau$ and 
(ii) $\impl{e}{s}{\alpha}$
\item If $r = \mathit{pre}$, then this atomic block maps to a self loop before the atomic specification action and we check $\impl{e}{s}{\mathit{havoc }\ \vec{w}, \hiddenVars}$. 
\item If $r = \mathit{post}$, then this atomic block maps to a self loop before the atomic specification action and we check $\impl{e}{s}{\mathit{havoc }\ \hiddenVars}$. 
\end{itemize}

%%% Local Variables: 
%%% mode: latex
%%% TeX-master: "paper"
%%% End: 

