\section{A concurrent programming language}
\label{sec:language}

We present our verification method on a core concurrent programming language, \civl.
Figure~\ref{fig:syntax} shows the syntax;  the operational semantics is described informally below.
A \civl program is comprised of threads each of which has a stack, global variables shared across threads, 
thread-local variables that are shared across procedures, and local variables inside a procedure.
Thus, a program contains all information to represent not only the static program 
written by the programmer but also the dynamic state of the program as it executes.
The mutually disjoint sets $\Global$, $\ThreadLocal$, and $\Local$ are the names of global, thread-local, and procedure-local variables 
respectively, and $\Var$ denotes their union.
\begin{wrapfigure}[14]{l}{0.6\textwidth}
\vspace*{-0.5cm}
\setlength{\tabcolsep}{3pt}
\scriptsize{
\begin{tabular}{rclcl}
$\stmt \in \Stmt$ &::= & $\skipstmt \mid \yield{e,\lins} \mid \call{A} \mid$ \\
                  & & $\call{P} \mid \async{P} \mid$ \\
                  & & $\ablock{e,\lins}{\stmt} \mid \stmt;\;\stmt \mid$\\
                  & & $\ite{\locExpr}{\stmt}{\stmt} \mid$ \\
                  & & $\while{e,\alpha}{\locExpr}{\stmt}$ \\
$\Frame \in \mathit{Frame}$ &::= & $(P, \varsL, \stmt)$ \\
$T \in \mathit{Thread}$ &::= &$(\varsTL, \Frames)$ \\
$\StmtCtxt \in \mathit{StmtCtxt}$ &::= &$[]_{\Stmt} \mid \StmtCtxt;\stmt$ \\
$\FrameCtxt \in \mathit{FrameCtxt}$ &::= & $(P, []_{\StoreLocal}, \StmtCtxt)$ \\
$\ThreadCtxt \in \mathit{ThreadCtxt}$ &::= &$([]_{\StoreThreadLocal}, \FrameCtxt\cdot\Frames)$ \\
$\ProgCtxt \in \mathit{ProgCtxt}$ &::= & $(\procs, \actions, \ProcLins, []_{\StoreGlobal}, \TS \cdot \ThreadCtxt \cdot \TS)$ \\
$\Prog \in \Program$ &::= & $\ProgCtxt[\varsG][\varsTL][\varsL][\stmt]$ \\
$\YieldingThread \in \mathit{YieldingThread}$ &::= &$\ThreadCtxt[\varsTL][\varsL][\yield{e,\lins}]$ \\
$\YProgram$ &::= & $(\procs, \actions, \ProcLins, \varsG, \YieldingThreads \cdot T \cdot \YieldingThreads)$ \\
$\CProgram$ &::= & $(\procs, \actions, \ProcLins, \varsG, \YieldingThreads)$
\end{tabular}
}
\caption{Syntax}
\label{fig:syntax}
\end{wrapfigure}
There is an uninterpreted set $\Value$ of values that may be stored in these variables.
$\Store$, $\StoreGlobal$, $\StoreThreadLocal$, and $\StoreLocal$ are all sets of maps into $\Value$
from $\Var$, $\Global$, $\ThreadLocal$, and $\Local$ respectively.
We denote elements of $\Store$, $\StoreGlobal$, $\StoreThreadLocal$, and $\StoreLocal$
by $\sigma$, $\varsG$, $\varsTL$, and $\varsL$ respectively.
Formally, a \civl program is a tuple $(\procs, \actions, \ProcLins, G, \TS)$ with the following components.

\noindent
{\bf Procedures.}
$\procs$ maps a procedure name $P \in \ProcName$ to a tuple $(\phi, \mods, \psi, \stmt)$, 
where $\phi$ is the precondition of $P$, $\mods$ is the set of thread-local variables potentially modified by $P$, 
$\psi$ is the postcondition of $P$, and $\stmt$ is the body of $P$.
The predicates $\phi$ and $\psi$ cannot refer to procedure-local variables.
Procedures do not have parameters, but parameters may be modeled using thread-local variables.

\noindent
{\bf Actions.}
$\actions$ maps an action name $A \in \ActionName$ to a tuple $(\rho,\alpha,m)$.
Actions are used inside procedure bodies to access global and thread-local variables.
The predicate $\rho$ over $\Store$ is the gate and the predicate $\alpha$ over $\Store \times \Store$ 
is the transition relation of $A$.
If the gate $\rho$ does not hold on the current state, the program fails;
otherwise, the program makes progress by executing the action $A$ based on the transition relation $\alpha$.
Finally, $m$ is one of four values in $\{B,R,L,N\}$;
it denotes the commutativity type of the action, $B$ for both mover, $R$ for right mover, $L$ for left mover, 
and $N$ for non mover~\cite{FlanaganFLQ08}. 

\noindent
{\bf Linear interfaces.}
$\ProcLins$ maps each procedure and action name in $\ProcName \cup \ActionName$ to a linear interface~\cite{Wadler90lineartypes}
$(\lins,\lins')$, where each of $\lins$ and $\lins'$ is a set comprising global and thread-local variables.
To exploit linearity during verification, the programmer provides a function $\Perm$ from $\Value$ to $2^\Value$.
The set $\Perm(v)$ is the set of {\em permissions\/} associated with a value $v \in \Value$.
The \civl type checker computes a set of {\em available\/} linear variables at each control location such that
the multiset union of the permissions associated with the available variables is guaranteed to be a set, i.e., 
permissions are never duplicated among linear variables.
The linear interface $(\lins, \lins')$ of a procedure (or action) indicates the available sets
at its beginning ($\lins$) and end ($\lins'$), respectively.

\noindent
{\bf Global store.}
$\varsG$ is the global store of the program.

\noindent
{\bf Threads.}
$\TS$ is a sequence of threads.
Each thread $T$ in $\TS$ is a pair $(\varsTL, \Frames)$ where
$\varsTL$ is the thread-local store and $\Frames$ is a stack of frames comprising the continuation of $T$.  
Each frame $(P, \varsL, \stmt)$ in $\Frames$ comprises a procedure $P$, its procedure-local store $\varsL$, 
and a statement $\stmt$.

We formalize the preemptive operational semantics of \civl as a relation $\trans$ with domain $\Program$ and 
codomain $\Program \cup \{\error\}$.
We use four contexts---$\StmtCtxt$, $\FrameCtxt$, $\ThreadCtxt$, and $\ProgCtxt$---to formalize $\trans$
as a structured operational semantics~\cite{WrightF94}.
In this semantics, a thread is chosen nondeterministically to execute a single statement.
The statement $\skipstmt$ has no side effect.
The statement $\yield{e,\lins}$ contains a predicate $e$ and a set $\lins$ comprising global and thread-local variables.
The yield predicate $e$ is expected to be established by the executing thread and preserved by other threads;
the variables $\lins$ are expected to be available at the yield statement.
The yield statement is the mechanism by which the programmer specifies the cooperative semantics (described later) over which 
the refinement check is performed.
In \civl, yield predicates, preconditions, and postconditions are the three sources of location invariants~\cite{OwickiG76}.
The statement $\call{A}$ invokes the action $A$.
Suppose $\actions(A) = (\rho, \alpha, m)$.
If $\rho$ does not hold, the statement $\call{A}$ fails and the program evolves to $\error$, 
otherwise the store is updated according to the transition relation $\alpha$.
The statement $\call{P}$ invokes the procedure $P$.
Suppose $\procs(P) = (\phi, \mods, \psi, \stmt)$.
It is expected that the precondition $\phi$ holds, then statement $\stmt$ executes, and finally the postcondition $\psi$ holds.
The predicates $\phi$ and $\psi$ are expected to be preserved by other threads in the environment. 
The statement $\async{P}$ creates a new thread that begins by invoking the procedure $P$.
The statement $\ablock{e,\lins}{\stmt}$ is useful only for compactly encoding non-interference checks;
the predicate $e$ is expected to hold and the set $\lins$ is expected to be available at the beginning of $\stmt$.
Finally, we have the standard primitives for sequencing, conditional execution, and looping.

We now define the cooperative operational semantics induced by the presence of yield statements in the program.
A {\em yielding thread\/} ($\mathit{YieldingThread}$ in Figure~\ref{fig:syntax}) is one that is waiting to execute at a yield statement.
A program in $\YProgram$ is one in which all threads except at most one thread are yielding threads.
A program in $\CProgram$ is one in which all threads are yielding threads.
Thus, we have $\CProgram \subseteq \YProgram \subseteq \Program$.
Let $\ytrans$ be the relation obtained by restricting the domain and codomain of $\trans$ to $\YProgram$.
In a $\ytrans$-execution, the one (possibly) non-yielding thread is the only one allowed to execute until it reaches 
a yield statement and becomes a yielding thread; at that point, all threads are yielding and any one can be picked for execution.
We define the relation $\ctrans$ with domain $\CProgram$ and codomain $\CProgram \cup \{\error\}$ as follows:
\begin{enumerate}
\item 
$\Prog \ctrans \Prog'$ if there is a $\ytrans$-execution from $\Prog$ to $\Prog'$ such that every program on the execution 
except for $\Prog$ and $\Prog'$ is not in $\CProgram$.
\item
$\Prog \ctrans \error$ if there is a $\ytrans$-execution from $\Prog$ to $\error$ such that every program on the execution 
except for $\Prog$ is not in $\CProgram$.
\end{enumerate}

Consider a program $\Prog \in \CProgram$.
We write $\Safe(\Prog)$ if it is not the case that $\Prog \trans \error$.
We write $\Safe^*(\Prog)$ if $\Safe(\Prog')$ for all $\Prog'$ such that $\Prog \trans^* \Prog'$.
We write $\CSafe(\Prog)$ if it is not the case that $\Prog \ctrans \error$.
We write $\CSafe^*(\Prog)$ if $\CSafe(\Prog')$ for all $\Prog'$ such that $\Prog \ctrans^* \Prog'$.

We write $\Cooperative(\Prog)$ if for each infinite $\ytrans$-execution $\Prog \ytrans \Prog_0 \ytrans \Prog_1 \ytrans \cdots$ 
starting from $\Prog$, there is some $i \geq 0$ such that $\Prog_i \in \CProgram$.
We write $\Cooperative^*(\Prog)$ if $\Cooperative(\Prog')$ for all $\Prog' \in \CProgram$ such that $\Prog \ctrans^* \Prog'$.
Our verification method, described in the next section, is applicable to a program $\Prog$ only if $\Cooperative^*(\Prog)$ holds.
Verifying this condition is similar to proving termination and is orthogonal to the contribution of this paper.
