\section{Modules}
\label{sec:modules}

This section describes a simple module system built on \civl that allows separate verification of modules,
allowing programmers to check a large program by breaking it into smaller pieces and checking the pieces independently.
A key challenge for modular verification in \civl is the $\CommutativitySafe$ and $\InterferenceSafe$ judgments.
Naively, these are whole-program judgments,
quadratically checking all pairs of actions or all pairs of yields and atomic blocks from an entire program.
To check these judgments on a per-module basis rather for a whole-program,
we observe that $\CommutativitySafe$ and $\InterferenceSafe$ are trivially true for operations that act on disjoint sets of global variables.
If an atomic block modifies only variables $g_1$ and $g_2$, it will not interfere with a yield that refers only to variables $g_3$ and $g_4$.
Suppose each module $M$ owns a set of global variables, such that each global variable is owned by exactly one module.
Then we can decree that only $M$'s procedures and actions can access $M$'s global variables.
Formally, define ownership for global variables, procedures and actions as:

\begin{tabular}{rclcl}
$M$ & $\in$ & $\Module$ \\
$\own_{\Global}$ & $\in$ & $\Global \rightarrow \Module$ \\
$\own_{\ProcName}$ & $\in$ & $\ProcName \rightarrow \Module$ \\
$\own_{\ActionName}$ & $\in$ & $\ActionName \rightarrow \Module$ \\
\end{tabular}

If $\own_{\ProcName}(P) = M$, then $P$'s preconditions, postconditions, and statements only refer to global $g$ where $\own_{\Global}(g) = M$.
Similarly, if $\own_{\ProcName}(A) = M$, then $A$'s gate and transition relation can depend only on $g$ where $\own_{\Global}(g) = M$.
With this restriction, $\CommutativitySafe$ and $\InterferenceSafe$ can be checked for each module $M$ in isolation;
no other module $M'$ could interfere with any of the global variables owned by $M$.
Note that $\own_{\Global}$, $\own_{\ProcName}$, and $\own_{\ActionName}$ can change across a series of refinement layers.
For example, a library module implementing locks may define a variable to represent the abstract state of a lock;
after the lock module is verified at a low layer,
a client module can take ownership of the lock variable in a higher layer
(see \cite{gc-techreport} for a detailed example).
