
\section{Related work}
\label{sec:related}

Our work is the first to provide tool and theory to support automated,
modular whole-program refinement through multiple layers, as distinct from existing work on
single-layer atomicity refinement between procedure implementations and
specifications. 
While many of the verification techniques used within \civl appear in
the literature, \civl is the first to make sound, joint use of them to
decompose the refinement task following the syntactic structure of a
program. In the following, we first contrast \civl with refinement
verification techniques. We then contrast \civl with tools and
techniques for reasoning about concurrent programs in general. 

\subsection{Refinement-oriented verification}
Atomic action specifications have been explored by the
\calvin~\cite{FlanaganFQS05,FreundQ04} verifier previously. 
\civl makes a distinction between preemptive and cooperative
semantics, and carries out refinement verification on a procedure body
with cooperative semantics as enabled by movers types and reduction.
\calvin attempts to verify refinement directly on the preemptive
semantics, making only limited use of movers at the lowest-level
representation. 
\calvin, unlike \civl, does not support location invariants and linear
variables but incorporates rely-guarantee reasoning. 
The same non-interference reasoning can be carried out using location
invariants or rely-guarantee reasoning, and \civl supports both.  
However, in certain cases, rely-guarantee reasoning
requires use of auxiliary (shared) variables and makes interactive
proofs difficult as was the case in our GC proof. We find location
invariants to be a powerful verification device.  

\QED~\cite{ElmasQT09} is a simplifier for concurrent programs and is close in spirit to the 
refinement-oriented approach of \civl.
A key distinction between \civl and \QED is the fact that a proof step in \QED is a small rewrite in the concurrent program
that must be justified by potentially expensive reduction and invariant reasoning.
In \QED, abstract versions of a program are obtained by transforming the initial
version by applying a proof script. 
This makes it difficult to organize a proof in \QED by providing a
description of a program at several different levels of abstraction. 
Further, the number of small proof steps directly affect both programmer
and computer effort. 
By contrast, \civl supports large proof steps, in each of which the bodies of several procedures
are automatically replaced by atomic actions, thereby lowering the cost of both interaction and automation.
The non-interference reasoning in \QED is even more limited than \calvin.
\QED supports only global invariants and does not support rely-guarantee reasoning or linear variables.

Liang et al.~\cite{LiangRGSim} present a method for verifying that procedure
bodies refine atomic specifications
The key verification approach is
rely-guarantee reasoning and the refinement (simulation) relation between
a procedure and its specification is constrained so it is preserved under
parallel composition. 
No tool support is provided. 
Authors present a (paper) GC proof, which is limited in scope compared
to ours, as their proof corresponds to a few layers of our proof. In particular,
the GC is not refined down to individual atomic memory accesses. 
Since this work uses different languages to describe the high-level
and low-level programs, it is not immediately possible to carry out a
multi-level stepwise refinement proof. 

Turon and Wand~\cite{TuronM11} use ownership disciplines and
separation logic to verify refinement of atomic specifications by 
concurrent data structure implementations. 
Rely-guarantee reasoning is
supported to provide compositionality and non-interference
arguments. 
This work targets a single refinement step between atomic
specifications for methods and their implementations. 
No tool support for this verification method is provided. 

Verifying linearizability of concurrent data structures (see, e.g.,
\cite{tacasLin,aliLin}) can be viewed as an instance of one-level of
refinement in our setting. 
\civl can be used for mechanical
verification of linearizability, as we did for the Treiber stack. 
Tools and techniques specific to verifying linearizability
cannot, however, be easily generalized for stepwise refinement proofs
through multiple levels. 

Refinement proofs
between implementations and specifications of protocols have been
investigated using the TLA+~\cite{Lamport2004} specification
language. 
Compositional proofs between specifications and
implementations consisting of modules~\cite{AbadiAssumeGuarantee} have
been investigated in this context. 
Modular refinement proofs for hardware systems have been investigated extensively
(e.g.,~\cite{Henzinger1999,Eiriksson2000}) using the SMV~\cite{McMillan00} and Mocha~\cite{AlurHMQRT98} 
model checking tools.
To verify a concurrent, shared-memory program using Mocha, SMV or
TLA+, one must encode
the program semantics as a state-transition system and express
verification goals in terms of this system. 
For concurrent, shared-memory
software, there is a cognitive gap between the program's state-transition relation
and the structured, imperative multithreaded program text.
\civl enables users to reason on the latter.
 

\subsection{Reasoning about concurrency}
In this section, we briefly discuss foundational techniques
for combating the complexity of concurrent program verification. 
\civl as well as atomicity refinement verification discussed in the previous
section have common ideas with tools and formalisms
discussed in this section, however, the latter primarily target verification of a {\em single} concurrent program rather than refinement. 
Refinement in \civl is orthogonal to these techniques and can potentially be combined with them.
Conversely, these techniques can be aided by \civl's 
ability to connect a complex concurrent program to a simpler abstraction.

VCC~\cite{VCC} is a tool for verifying concurrent C programs.  
Chalice~\cite{LM09} is a language and modular verification tool for concurrent programs. 
VCC does not support refinement and Chalice does so only for sequential programs.  
VCC and Chalice base their invariant reasoning on objects, object ownership, and type invariants. 
Invariant reasoning in \civl is more primitive and based on predicates in yield statements. 
Although the approach in VCC and Chalice is more convenient when applicable, \civl's approach is more flexible. 
VCC and Chalice can reason sequentially about objects exclusively owned by a thread;
\civl accomplishes the same using linear variables.
Neither VCC nor Chalice support movers and reduction reasoning.

Concurrent separation logic~\cite{OHearn07} reasons about concurrency without 
explicitly checking for non-interference between threads. 
Recently, tools based on this logic that blend in explicit non-interference reasoning have been developed~\cite{SAGL,RGSep}. 
\civl's combination of interference checking and linear variables is
an extreme example of this trend, is very general and technique-agnostic. 
We supply very primitive abstractions and let programmers mix and
match these abstractions freely to encode the non-interference reasoning style of their choice. 



