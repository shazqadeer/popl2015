\section{Introduction}
\label{sec:introduction}

We present a technique for verifying a refinement relation between two concurrent, shared-memory multithreaded programs. 
Our work is inspired by stepwise refinement~\cite{Wirth1971}, an approach in which a high-level description is systematically refined, 
potentially via several intermediate descriptions, down to a detailed implementation. 
Refinement checking is a classical problem in verification and has been investigated in many contexts, 
including hardware verification~\cite{Eiriksson2000} and verification of cache-coherence protocols and distributed algorithms~\cite{Lamport2004}.
In the realm of sequential software, notable successes using the refinement approach include the work of Abrial et al.~\cite{AbrialBHHMV10} 
and the proof of full functional correctness of the seL4 microkernel~\cite{KleinAEMSKH14}. 
This paper presents the first general and automated proof system for refinement verification of shared-memory multithreaded software. 

We present our verification approach in the context of \civl, an idealized concurrent programming language.
In \civl, a program is described as a collection of procedures whose implementation 
can use the standard features such as assignment, conditionals, loops, procedure calls, and thread creation. 
Each procedure accesses shared global variables only through invocations of atomic actions.
A subset of the atomic actions may be refined by new procedures and a new program is 
obtained by replacing the invocation of an atomic action by a call to the corresponding procedure refining the action.
Several layers of refinement may be performed until all atomic actions in the final program are directly implementable primitives.
Unlike classical program verifiers based on Floyd-Hoare reasoning~\cite{Floyd67,Hoare69} that manipulate a program and annotations, 
the \civl verifier manipulates multiple operational descriptions of a program, i.e., 
several layers of refinement are specified and verified at once. 

To prove refinement in \civl, a simulation relation between a program and its abstraction is inferred 
from checks on each procedure with an atomic specification.
The computation inside each such procedure must be partitioned into ``steps''
such that one step behaves like the atomic specification and all other steps have no effect on the visible state.
To carry out this reasoning, the \civl verifier needs to address two problems.
First, the notion of a ``step'' in the implementation must be defined.
The definition of a step can deeply affect the number of checks that need to be performed and the number of user annotations.
Second, it is typically not possible to show the correctness of a step from an arbitrary state.
A precondition for the step in terms of shared variables must be supplied by the programmer and mechanically checked by the verifier.

To address the first problem, \civl lets the programmer define the granularity of a step.
A {\em cooperative\/} semantics for the program is explicitly introduced by the programmer 
through the use of a new primitive {\em yield\/} statement;
in this semantics a thread can be scheduled out only when it is about to execute a yield statement.
The {\em preemptive\/} semantics of the program is sequentially consistent execution; all threads are imagined
to execute on a single processor and preemption, which causes a thread to be scheduled out and a nondeterministically chosen thread to 
be scheduled in, may occur before any instruction.\footnote{In this paper, 
we focus our attention on sequential consistency and leave consideration of weak memory models to future work.}
Given a program $P$, \civl verifies that the safety of the cooperative semantics of $P$ implies the safety of the preemptive semantics of $P$.
This verification is done by computing an automata-theoretic simulation check~\cite{HenzingerHK95} 
on an abstraction of $P$ in which each atomic action of $P$ is represented by only its mover type~\cite{Lipton75,FlanaganFLQ08}. 
The mover types themselves are verified separately and automatically using an automated theorem prover~\cite{MouraB08}.

To address the second problem that refinement verification typically requires invariants about the program execution, 
\civl allows the programmer to specify location invariants, attached either to a yield statement
or to a procedure as its precondition or postcondition.
Each location invariant must be correct for all executions and must continue to hold in spite of potential interference from concurrently executing threads.
We build upon classical work~\cite{OwickiG76,Jones83} on reasoning about non-interference
with two distinct innovations.
First, we do not require the annotations to be strong enough to prove program correctness but only strong enough to provide the context for refinement checking. 
Program correctness is established via a sequence of refinement layers from an abstract program that cannot fail.
Second, to establish a postcondition of a procedure, we do not need to propagate a precondition through all the yield annotations in the procedure body. 
The correctness of an atomic action specification gives us a simple frame rule---the precondition only needs to be propagated across the atomic action specification. 
\civl further simplifies the manual annotations required for logical non-interference checking
by providing a linear type system~\cite{Wadler90lineartypes} that enables logical encoding of thread identifiers, permissions~\cite{boyland:03fractions}, 
and disjoint memory~\cite{LahiriQW11}.

\begin{wrapfigure}{l}{0.5\textwidth}
{\scriptsize
\begin{verbatim}
var Color: int; // WHITE=1, GRAY=2, BLACK=3
procedure WB(linear tid:Tid)
atomic [if (Color == WHITE) Color := GRAY];
{
  var cNoLock:int;
  cNoLock := GetColorNoLock(tid);
  yield Color >= cNoLock;
  if (cNoLock == WHITE) 
    call WBSlow(tid);
}
procedure WBSlow(linear tid:Tid)
atomic [if (Color == WHITE) Color := GRAY];
{
  var cLock:int;
  call AcquireLock(tid);
  cLock := GetColorLocked(tid);
  if (cLock == WHITE) 
    call SetColorLocked(tid, GRAY);
  call ReleaseLock(tid);
}
procedure GetColorNoLock(linear tid:Tid) 
  returns (cl:int) atomic [...];
procedure AcquireLock(linear tid:Tid) 
  right [...];
procedure ReleaseLock(linear tid:Tid) 
  left [...];
procedure GetColorLocked(linear tid:Tid) 
  returns (cl:int) both [...];
procedure SetColorLocked(linear tid:Tid, 
  cl: int) atomic [...];
\end{verbatim}
}
\caption{Write barrier}
\label{fig:reft}
\end{wrapfigure}

Finally, \civl provides a simple module system. Modules areverified separately, soundly doing away with checks that pertain to cross-module interactions. 
This feature is significant since commutativity checks and non-interference checks for location invariants are quadratic, whole program checks involving all pairs of yield locations and atomic blocks, or all pairs of actions from a program. 
Using the module system, the number of checks is reduced; they become quadratic in the number of yields and atomic blocks {\em within each module} rather than the entire program. 

We have implemented \civl as a conservative extension of the \boogie verifier.  
We have used it to verify a collection of microbenchmarks and benchmarks from the 
literature~\cite{Blumofe1999,ElmasQT09,ElmasTQ05,FarzanKP14,FlanaganQ03,Herlihy2008}. 
The most challenging case study with \civl was carried out concurrently with \civl's development and served as a design driver. 
We verified a concurrent garbage collector, through six levels of refinement, 
down to atomic actions corresponding to individual memory accesses. 
The level of granularity of the lowest-level implementation distinguishes this verification effort, 
detailed in a technical report~\cite{gc-techreport}, from previous attempts in the literature. 

In conclusion, \civl is the first automated verifier for shared-memory multithreaded programs that 
provides the capability to establish a multi-layered refinement proof.
This novel capability is enabled by two important innovations in core verification techniques
for reducing the complexity of invariants supplied by the programmer and the verification conditions solved by the prover.
\begin{itemize}
\item
Reasoning about preemptive semantics is replaced by simpler reasoning about cooperative operational semantics by exploiting 
automata-theoretic simulation checking.
We are not aware of any other verifier that combines automata-based and logic-based reasoning in this style.
\item 
A linear type system establishes invariants about disjointness of permission sets associated with values contained in program variables.
These invariants, communicated to the prover as free assumptions, significantly reduce the overhead of program annotations.
We are not aware of any other verifier that combines type-based and logic-based reasoning in this style.
\end{itemize}
